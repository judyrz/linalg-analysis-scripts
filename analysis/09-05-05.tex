\begin{Bew}
  Fall 1:$f$ reellwertig\\
  Sei 
  \[\phi(x,y):=f(a+xe_i+ye_j)\]
%  \begin{enumerate}
%    \item $\implies$ $\exists$ Umgebung $V$ von $(0,0)\in\mb{R}^2$, wo folgende partielle Ableitungen existieren:
%      \begin{align*}
%        \partial_x\phi&=\partial_if\\
%        \partial_y\phi&=\partial_jf\\
%        \partial_x\partial_y&=\partial_j\partial_if
%      \end{align*}
%    \item $\implies$ $\partial_y\partial_x\phi$ ist stetig in $(0,0)$
%  \end{enumerate}
  Zu zeigen:
  \[\underbrace{\partial_x\partial_y\phi(0)}_{=\partial_i\partial_jf(a)}=\underbrace{\partial_y\partial_x\phi(0)}_{=\partial_j\partial_if(a)}\]
  $\forall \varepsilon >0$ $\exists$ Umgebung $V'$ von $(0,0)$ mit $V'\subset V$
  \begin{align*}
    \Abs{\partial_y\partial_x\phi(x,y)-\partial_y\partial_x\phi(0,0)}<\varepsilon &&\forall (x,y)\in V'
  \end{align*}
  Sei $Q:=\left( 0;h \right)\times\left( 0;k \right)$ wobei $h,k>0$ so dass $Q\subset V$
  \begin{Lem}
    $\exists\ (\xi,\eta)\in Q$
    \[\frac{D_Q\phi}{hk}=\partial_y\partial_x\phi(\xi,\eta)\]
    (Mittelwertsatz)
  \end{Lem}
  \begin{gather*}
    Q\subset V'\implies\Abs{\frac{D_Q\phi}{hk}-\partial_i\partial_x\phi(0,0)}<\varepsilon
  \end{gather*}
  \begin{gather*}
    \frac{D_Q\phi}{hk}:=\frac{\phi(h,k)-\phi(h,0)-\phi(0,k)+\phi(0,0)}{hk}=\frac{1}{h}\left( \frac{ \phi(h,k)+\phi(0,0)}{k}-\frac{\phi(h,0)-\phi(0,k)}{k}\right)\\
    \Limo{k}\frac{D_Q\phi}{hk}=\frac{\partial_y\phi(h,0)-\partial_y\phi(0,0)}{h}\\
    \Abs{\ }<\varepsilon\implies\Limo{k}\Abs{\ }=\Abs{\Limo{k}\ }\leq \varepsilon\\
    \Abs{\frac{\partial_y\phi(h,0)-\partial_y\phi(0,0)}{h}-\partial_y\partial_x\phi(0,0)}\leq \varepsilon
  \end{gather*}
  d.h. $\forall\varepsilon>0$ $\exists \delta>0$ s.d. $\forall h:\Abs{h}<\partial$ gilt die Ungleichung. D.h.
  \[\Limo{h}\frac{\partial_y\phi(h,0)-\partial_y\phi(0,0)}{h}=\partial_y\partial_x\phi(0,0)=\partial_x\partial_y\phi(0,0)\]
  Fall2: $f$ komplexwertig\\
  Re $f$, Im $f$ erfüllen die Voraussetzung von Fall 1.
\end{Bew}
\begin{Def}{$k$-mal stetig differenzierbar}
  Sei $f:\underbrace{U}_{\subset\mb{R}}\to \mb{C}$  $f$ heisst $k$-mal stetig differenzierbar ($k\geq 1$) wenn \underline{alle} partiellen Ableitungen $k$-ter Ordnung
  \[\left( \partial_{i_1}\partial_{i_2}\cdots\partial_{i_k}f,\forall \left( i_1,\cdots,i_k \right)\in\left\{ 1,\cdots,n \right\}^k \right)\]
  auf $U$ existieren und stetig sind.
\end{Def}
\begin{Def}
  \[\mathcal{C}^k(U)=\left\{ \text{$k$-mal stetig differenzierbare Funktionen auf $U$} \right\}\]
\end{Def}
\begin{Bem}
  \begin{itemize}
    \item $\mathcal{C}$ Vektorraum
    \item $\mathcal{C}^{k+l}\in\mathcal{C}$ $\forall l\geq 0$
  \end{itemize}
\end{Bem}
\begin{Def}
  \[\mathcal{C}^\infty(U)=\cap_{k=1}^\infty\mathcal{C}^k(U)\]
  beliebig of stetig differenzierbare Funktionen auf $U$ (auch ein Vektorraum)
\end{Def}
\begin{Not}
  Genauere Bezeichnungen:
  \begin{itemize}
    \item $\mathcal{C}^k(U,\mb{R})$ reellwertig
    \item $\mathcal{C}^k(U,\mb{C})$ komplexwertig
  \end{itemize}
\end{Not}
\begin{Def}{Zweite Ableitung}
  Sei $f\in\mathcal{C}^2(U)$, $U\subset\mb{R}^n$ Seien $u,v\in\mb{R}^n$, $a\in U$
  \[\md^2f_a(u,v):=\partial_u\partial_vf(a)\]
\end{Def}
\begin{Bem}
  \begin{align*}
    \partial_vf(a)&=\sum^n_{i=1}\partial_if(a)v_i\\
    \partial_u(\partial_vf(a))&=\sum^n_{i=1}\partial_u(\partial_vf(a))v_i\\
    =\md^{(2)}f_a(u,v)&=\sum_{j=1}^n\sum^n_{i=1}\partial_j\partial_if(a)v_iu_j
  \end{align*}
  \[\md^{(2)}f_a:\mb{R}^n\times\mb{R}^n\to\mb{C}\]
  ist bilinear.
\end{Bem}
\begin{Bem}
  Schwarz:
  \begin{align*}
    f\in\mathcal{C}^2\implies\partial_i\partial_jf(a)=\partial_j\partial_if(a)&& \forall\\
    \implies \md^{(2)}f_a\ \text{symmetrisch}
  \end{align*}
  \begin{align*}
    \md^{(2)}f_a(u,v)=\md^{(2)}=\md^{(2)}f_a(v,u)&&\forall u,v
  \end{align*}
\end{Bem}
\begin{Bem}
  Die darstellende Matrix von $\md^{(2)}f_a$ ist
  \[f''(a):=\left( \partial_i\partial_jf(a) \right)\]
  2. Ableitung von $f$ im Punkt $a$. Andere Bezeichnung:
  \[H_f(a):=f''(a)\]
  Hesse-Matrix von $f$ im Punkt $a$.
\end{Bem}
\begin{Bem}
  Sei $f\in\mathcal{C}^2(U)$, $a\in U$
  \[H_f(a):=\left( \partial_i\partial_jf(a)j \right)\]
  \begin{itemize}
    \item $H_f(a)$ symmetrische Matrix
    \item \[\md^{(2)}f(a)\left( u,v \right)=u^tH_f(a)v=\sum_{i,j=1}^nf_{ij}''(a)u_iv_j\]
  \end{itemize}
\end{Bem}
\begin{Bem}
  Die Spur der Hesse-Matrix von $f$:
  \[\Delta f(a):=\Spur H_f(a)=\sum_{i=1}^n\partial^2_if(a)\]
  \[\Delta:=\sum^n_{i=1}\partial^2_i\]
  $\Delta$ Laplace-Operator
\end{Bem}
\begin{Lem}
  Für jede Orthonormalbasis $\left( v_1,\cdots,v_n \right)$ von $\mb{R}^n$ gilt
  \[\Delta f=\sum^n_{i=1}\partial_{v_i}^2f\]
\end{Lem}
\begin{Def}{Differential $p$-ter Ordnung}
  Sei $f\subset\mathcal{C}^p(U)$, $U\subset \mb{R}^n$. Sei $a\in U$. Seien $v^1,v^2,\cdots,v^p\in\mb{R}^n$
  \[\md^{(p)}f_a(v^1,\cdots,b^p):=\partial_{v^1}\partial_{v^2}\cdots\partial_{v^p}f(a)\]
  \[\md^{(p)}f_a(v^1,\cdots,b^p):=\sum^n_{i_1=1}\cdots\sum^n_{i_p=1}\partial_{i^1}\partial_{i^2}\cdots\partial_{i^p}f(a)v_{i_1}v_{i_2}^2\cdots v_{i_p}^p\]
  $f\in \phi^p$ und Schwarz $\implies$
  \[\partial_{i_1}\cdots\partial_{ip}f(a)=\partial_{i_{\sigma(1)}}\cdots\partial_{i_{\sigma(p)}}f(a)\ \forall \sigma:\left\{ 1,\cdots,p \right\}\to\left\{ 1,\cdots,p \right\}\]
  \[\md^{(p)}f_a(v^1,\cdots,v^p)=\md^{(p)}f_a(v^{\sigma(1)},\cdots,v^{\sigma(p)})\]
\end{Def}
\subsection{Taylorapproximation}
\begin{Bem}
  Sei $f\in\mathcal{C}^{p+1}(U,\mb{R})$, $U\subset\mb{R}^n$. Seien $a,x\in U$ s.d.
  \[[a;x]\subset U:=\left\{ a+t(x-a),t\in[0;1] \right\}\]
  \begin{align*}
    F:\left[ 0;1 \right]&\to\mb{R}\\
    t&\mapsto f(a+th) && h:=x-a
  \end{align*}
  \begin{itemize}
    \item 
      $F$ $p+1$-mal stetig differenzierbar (Kettenregel)
      \[F'(t)=\sum_{i=1}^n\partial_if(a+th)h_i=\md f_{a_{i_h}}\]
      \begin{align*}
        F''(t)=\sum_{i,j=1}^n\partial_j\partial_if(a+h)h_ih_j&&=\md^{(2)}f_a(h,h)\\
        \cdots\\
        F^{(k)}(t)=\sum^n_{i_1,\cdots,i_k=1}\partial_{i_1}\cdots\partial_{i_k}f(a+h)h_{i_1}\cdots h_{i_k} && =\md^{(k)}f_a(h,\cdots,h)
      \end{align*}
      Abkürzung: $V\in\mb{R}^n$
      \[\md^{(k)}f(a)v^k:=\md^{(k)}f(a)(v,\cdots,v)\]
      \begin{enumerate}
        \item $F$ reellwertig und $p+1$ stetig differenzierbar
        \item $F^{(k)}(t)=\md^{(k)}f(a+h)h^k$
      \end{enumerate}
      1) $\implies$
      \begin{gather*}
        F(1)=T_pF(1;0)+R_{p+1}\\
        T_pF(1;0)=\sum^p_{k=0}\frac{1}{k!}F^{(k)}(0)1^k\\
        R_{p+1}=\frac{1}{\left( p+1 \right)!}F^{(p+1)}(\tau), \tau\in [0;1]
      \end{gather*}
    \item 
      \begin{gather*}
        F(1)=f(a+h)=f(x)\\
        T_pF(1;0)=\sum^p_{k=0}\frac{1}{k!}\md^{(k)}f_ah^k=:T_pf(x,y)
      \end{gather*}
      Taylorapproximation der Ordnung $p$ von $f$ im Punkt $a$
    \item
      $x=a+\tau h$
      \[\exists\xi\in[a;x]:T_{p+1}=\frac{1}{(p+1)!}\md^{(p+1)}f(\xi)h^{p+1}=:R(x;a;\xi)\]
      Rest
  \end{itemize}
\end{Bem}
\begin{Def}{Taylorsatz mit Rest}
  Sei $f\in\mathcal{C}^{p+1}(U,\mb{R})$. Seien $a,x\in U$ mit $[a;x]\subset U$. Dann $\exists\xi\in [a;x]:$
  \[f(x)=T_pf(x;a)+R_{p+1}\left( x;a;\xi \right)\]
  wobei
  \[T_pf(x;a)=\sum^p_{k=0}\frac{1}{k!}\md^{(k)}f(a)(x-a)^k\]
  \[R_{p+1}f(x;a;\xi)=\frac{1}{\left( p+1 \right)}!\md^{(p+1)}f(a)(x-a)^{\phi+1}\]
\end{Def}
\begin{Kor}{Qualitative Taylorformel}
  Sei $f\in\mathcal{C}^p(U)$ (möglicherweise komplexwewrtig). Dann $\forall a\in U$
  \[f(x)=T_pf(x,a)+0\left( \Norm{x-a}^p \right), x\to a\]
  d.h.
  \[\lim_{x\to a}\frac{f(x)-T_pf(x;a)}{\Norm{x-a}^p}=0\]
\end{Kor}
\begin{Def}{Taylorreihe von $f$ im Punkt $a$}
  Sei $f\in\mathcal{C}^\infty(U)$, $a\in U$
  \[Tf(x;a):=\sum^\infty_{k=0}\frac{1}{k!}\md^{(k)}f(a)(x-a)^k\]
\end{Def}
\begin{Def}{reell-analytisch}
  Besitzt jeder Punkt von $U$ eine Umgebung, wo die Taylorreihe von $f$ gegen $f$ konvergiert, so heisst $f$ reell-analytisch.  
\end{Def}
\begin{Lem}
  Sei $f\in\mathcal{C}^\infty(U)$, $a\in U$, $r>0$ $K_r(a)\subset U$ $\forall k$ sei $P_k$ homogenes Polynom von Grad $k$ s.d.
  \begin{align*}
    f(x)=\sum\infty_{k=0}P_k(x-a)&&\forall x\in K_r(a)
  \end{align*}
  dann ist
  \[Tf(x;a)=\sum P_k(x-a)\]
\end{Lem}
\subsubsection{Geometrische Auffassung}
\begin{Def}{Tangentialhyperebene}
  Sei $f:\underbrace{U}_{\subset\mb{R}}\to \mb{C}$ und $f\in\mathcal{C}^1$. Der Graph des Taylorpolynomes 1. Ordnung
  \[\left\{ (x,z)\in\mb{R}^{n+1}:z=T_1f(x,a):=f(a)+\md f_a(x) \right\}\]
  heisst Tangentialhyperebene von $f$ im Punkt $a$.
\end{Def}
\begin{Def}{Schmiegquadrik}
  Sei $f\in\mathcal{C}^2(U)$. Der Graph des Taylorpolynomes 2. Ordnung
  \[\left\{ (x,z(\in\mb{R}^{n+1}:z=T_2f(x;a) \right\}\]
  heisst Schmiegquadrik an den Graphen von $f$ in $\left( a,f(a) \right)$
  \[z=f(a)+f'(a)(x-a)+\frac{1}{2}(x-a)^tf''(a)(x-a)\]
  \begin{gather*}
    \tilde x:=x-a\\
    \tilde z:=z-f(a)-f'(a)(x-a)\\
    \tilde z=frac{1}{2}\tilde x^tf''(a)\tilde x
  \end{gather*}
  Graph eine quadratische Funktion =: Quadrik
\end{Def}
\begin{Bsp}
  n=2
  \begin{table}
    \centering
    \begin{tabular}{c|c|c}
      Funktion&Name&(0,0)\\
      \hline
      $z=x^2+y^2$& elliptisches Paraboloid&Minimum\\
      $z=-(x^2+y^2)$&&Maximum\\
      \hline
      $z=x^2-y^2$&hyperbolisches Paraboloid&Sattelpunkt\\
      \hline
      $z=x^2$&parabolischer Zylinder&Minimum
      
    \end{tabular}
    \caption{<+Caption text+>}
    \label{tab:<+label+>}
  \end{table}<++>
\end{Bsp}<++>
