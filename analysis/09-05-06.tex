\subsection{Minima und Maxima}
\begin{Bem}
  Sei $A$ $n\times n$ symmetrische Matrix. Sei 
  \begin{align*}
    Q(x):=\frac{1}{2} x^tAx,x\in\mb{R}^n
  \end{align*}
  \begin{table}[htb]
    \centering
    \begin{tabular}{l|c|c|c|l}
      Name&Bedingung&Bezeichnung&Eigenwerte&(0,0)\\
      \hline
      positiv definit&$Q(x)>0$ $\forall x\neq 0$&$Q>0$& alle $>0$&isoliertes Minimum\\
      negativ definit&$Q(x)<0$ $\forall x\neq 0$&$Q<0$& alle $<0$&isoliertes Maximum\\
      positiv semidefinit&$Q(x)\geq0$ $\forall x$&$Q\geq0$& alle $\geq0$&Minimum\\
      negativ semidefinit&$Q(x)\leq0$ $\forall x$&$Q\leq0$& alle $\leq0$&Maximum\\
      indefinit&$\exists x:Q(x)>0$ $\exists <:Q(x)<0$& $Q\gtrless 0$&$\exists \lambda>0$ $\exists \mu<0$&kein Extremum
    \end{tabular}
  \end{table}
  $Q$ definit, wenn $Q>0$ oder $Q<0$\\
  $n=2$
  \begin{align*}
    Q>0 &\iff \det A>0\ \text{und}\ a>0\\
    Q<0 &\iff \det A<0\ \text{und}\ a>0\\
    Q\geq0 \text{oder}\ Q\leq 0&\iff \det A\geq0\\
    Q\gtrless 0&\iff \det A<0\\
  \end{align*}
\end{Bem}
\begin{Bew}
  \begin{gather*}
    A\sim\Mx{\lambda_1 0&\\0&\lambda_2}\\
    \det A=\lambda_1\lambda_2\\
    \Spur A:= a+c = \lambda_1\lambda_2\\
    \det A=ac-b^2\\
    \det A >0\implies ac > 0
  \end{gather*}
\end{Bew}
\begin{Def}
  Sei $f\in\mathcal{C}^2(U,\mb{R})$, sei $A=f''(a)$ und $Q(x)=\frac{1}{2}\md^{(2)}f(a)x^2$
  \begin{table}[htb]
    \centering
    \begin{tabular}{l|l|c}
      $f$ heisst& wenn $\md^2 f_a$\\
      \hline
      elliptisch&definit&$\md^2f>0$ oder $\md^2<0$\\
      hyperbolisch&nicht indefinit&$\md^2f\gtrless 0$\\
      flach&$\md^2f=0$
    \end{tabular}
  \end{table}
\end{Def}
\begin{Def}
  Sei $f:\underbrace{U}_{\subset\mb{R}}\to \mb{R}$. $f$ hat in $a\in X$ ein
  \begin{description}
    \item[lokales Maximum] wenn es eine Umgebung $V$ von $a$ gibt, so dass $f(x)\leq f(a)$ $\forall x\in V$
    \item[lokales Minimum] wenn es eine Umgebung $V$ von $a$ gibt, so dass $f(x)\geq f(a)$ $\forall x\in V$
    \item[isoliertes lokales Maximum] wenn es eine Umgebung $V$ von $a$ gibt, so dass $f(x)<f(a)$ $\forall x\in V\setminus \left\{ a \right\}$
    \item[isoliertes lokales Minimum] wenn es eine Umgebung $V$ von $a$ gibt, so dass $f(x)>f(a)$ $\forall x\in V\setminus\left\{ a \right\}$
  \end{description}
\end{Def}
\begin{Sat}{Notwendiges Kriterium}
  Sei $f:\underbrace{U}_{\subset\mb{R}}\to \mb{R}$. Ist $a\in U$ ein lokales Extremum von $f$ und ist $f$ partiell differenzierbar in $a$, so gilt
  \[\partial_1f(a)=\partial_2f(a)=\cdots=\partial_nf(a)=0\]
\end{Sat}
\begin{Bem}
  Ist $f$ differenzierbar in $a$, dann gilt $\md f_a=0$
\end{Bem}
\begin{Bew}
  $k\in \left\{ 1,\cdots,n \right\}$. Sei $F(t)=f(a+te_k)$. $a$ lokales Extremum von $f$ $\implies$ $t=0$ lokales Extremum von $F$ $f$ partiell differenzierbar in $a$ $\implies$ $F$ differenzierbnar in 0 $\implies$ $F'(0)=0$\\
  Kettenregel $F'(0)=\partial_kf(a)$
\end{Bew}
\begin{Def}{stationäre Stelle}
  Sei $f$ differenzierbar in $a$ und es gelte $\md f_a=$. Dann heisst $f$ stationär in $a$ und $a$ heisst stationäre Stelle in $f$ oder kritischer Punkt.
\end{Def}
\begin{Sat}{hinreichendes Kriterium}
  Sei $f\in\mathcal{C}^2(U,\mb{R})$, sei $\md f_a=0$ Dann
  \begin{enumerate}
    \item $\md^2f_a>0$ $\implies$ $f$ hat in $a$ ein isoliertes lokales Maximum
    \item $\md^2f_a<0$ $\implies$ $f$ hat in $a$ ein isoliertes lokales Minimum
    \item $\md^2f_a\gtrless0$ $\implies$ $f$ hat in $a$ kein Extremum
  \end{enumerate}
\end{Sat}
\begin{Bsp}
  $f(x,y)=x^4+y^4$ $(0,0)$ ist ein isoliertes lokales Minimum
  \begin{gather*}
    \md f_{(0,0)}=0\\
    \md^2 f_{(0,0)}=0
  \end{gather*}
\end{Bsp}
\begin{Bew}
  \begin{gather*}
    \md f_a=0\implies T_2f(x;a)=f(a)+\frac{1}{2}\md^2f(a)(x-a)^2\\
    f(a+h)=f(a)+\frac{1}{2}\md^2f(a)h^2+R_2(h)\\
    \frac{R_2(h)}{\Norm{h}^2}\to 0
  \end{gather*}
  \begin{enumerate}
    \item $\md ^2f_a>0$ Sei
      \begin{align*}
        \phi:S^{n-1}&\to\mb{R}\\
        h&\mapsto\md^2f(a)h^2
      \end{align*}
      $S^{n-1}$ kompakt $\implies$ $\phi$ nimmt ein Minimum $m$. $\phi>0\implies m>0$
      \begin{align*}
        \md^2f_a(h)^2=\md^2f_a\left( \Norm{h}\frac{h}{\Norm{h}} \right)^2=\Norm{h}^2\phi\left( \frac{h}{\Norm{h}} \right)\geq \Norm{h}^2m &&\forall h\in\mb{R}^h\setminus\left\{ 0 \right\}
      \end{align*}
      Sei $\delta>0$:
      \begin{enumerate}
        \item $K_\delta(a)\subset U$
        \item $\Abs{R_2(h)}\leq \frac{1}{4}m\Norm{h}^2$ $\forall h:\Norm{h}<\delta$
      \end{enumerate}
      Für $h$: $\Norm{h}<\delta$
      \begin{gather*}
        f(a+h)=f(a)+\underbrace{\frac{1}{2}\md^2f_ah^2}_{\geq \frac{m}{2}\Norm{h}}+R_2(h)\geq f(a)+\frac{m}{4}\Norm{h}^2\\
        \xRightarrow{m>0} f(a+h)> f(a)\ \forall h\in K_\delta(a)\setminus\left\{ 0 \right\}
      \end{gather*}
      $a$ ist ein isoliertes lokales Minimum
    \item $\md^2f_a<0$ Sei $\phi:=-f$. $\md^2g>0$ $\implies$ $g$ heisst in $a$ ein isoliertes lokales Minimum $\implies$ $f$ ist in $a$ in isoliertes lokales Maximum
    \item Sei $\md^2f_a \gtrless 0$. D.h. $\exists \md^2 f_av^2>0$ und $\exists \md^2 f_aw^2<0$ Sei $F_v(t):=f(a+tv)$ und $F_w(t):=f(a+tw)$
      \begin{gather*}
        \dot F_v(0)=\dot F_w(0)=0\\
        F_v''(0)=\md^2f(a)v^2>0
      \end{gather*}
      $\implies$ $F_v$ nimmt in $t=0$ ein Minimum an
      \begin{gather*}
        F_w''(0)=\md^2f(a)w^2<0
      \end{gather*}
      $\implies$ $F_v$ nimmt in $t=0$ ein Maximum an. Deshalb nimmt $f$ in $a$ kein Extremum an.
  \end{enumerate}
\end{Bew}
\begin{Sat}
  Sei $f\in\mathcal{C}^2(U,\mb{R})$, $a\in U$.
  \[a\ \text{lokales Maximum} \implies \md^2f\leq 0\]
  \[a\ \text{lokales Minimum} \implies \md^2f\geq 0\]
\end{Sat}
\begin{Bem}{Methode}
  Sei $f:U\to\mb{R}\left( \mathcal{C}^2 \right)$ und $U$ offen. Wir wollen alle Extrema finden
  \begin{enumerate}
    \item Man findet alle stationären Stellen d.h.
      \[a\in U:\md f_a=0\]
    \item Man studiert $\md^2f_a$ für $a$ stationär.
    \item beten (z.B. Taylor weiterentwickeln)
  \end{enumerate}
  Ist $f$ auf $\bar U$ definiert, so muss man Extrema auf $\bar U$ finden
  \begin{enumerate}
    \item Man sucht nach Extrema auf $f|_{\partial U}$
    \item Man verifiziert, ob solche Extrema eigentliche Extrema von $f$ auf $\bar U$ sind.
  \end{enumerate}
\end{Bem}
\subsection{harmonische Funktionen}
\begin{Def}{harmonische Funktionen}
  Sei $f\in\mathcal{C}^2(U)$ $f$ heisst harmonisch, falls
  \begin{align*}
    \Delta f(x)=0&&\forall x\in U
  \end{align*}
\end{Def}
\begin{Sat}{Schwaches Maximums-Prinzip für harmonische Funktionen} Sei $U\subset\mb{R}^n$ offen und beschränkt und sei $f:\bar U\to\mb{R}^n$ stetig s.d. $f|_U$ harmonisch ist. Dann nimmt $f$ ihr Maximum und ihr Minimum auf $\partial U$ an.
\end{Sat}
\begin{Bew}{Widerspruchsbeweis} Wir nehmen an:
  \begin{itemize}
    \item $f|_U$ harmonisch
    \item $f$ nimmt ihr Maximum nicht auf $\partial U$ an.
  \end{itemize}
  Seien $M:=\max_{x \in U} f(x)<\infty$ und $\mu:=\max_{x\in \partial x} f(x)<\infty$ denn $f$ ist stetig und $U$ beschränkt. Aus der 2. Annahme folgt, dass $\mu <M$
  \begin{gather*}
    \lambda:=\max_{x\in \partial U}\left( x_1^2+x_2^2+\cdots+x_n^2 \right)<\infty
  \end{gather*}
  Sei $\varepsilon>0:\mu+\varepsilon\lambda<M$. Sei $f_\varepsilon(x):=f(x)+\varepsilon\left( x_1^2+\cdots+x_n^2 \right)$.
  \begin{gather*}
    M_\varepsilon:=\max_{x\in \bar U}f_\varepsilon(x),\mu_\varepsilon:=\max_{x\in \partial U}f_\varepsilon (x)\\
    f_\varepsilon(x)>f(x)\ \forall x\\
    M_\varepsilon\geq M\\
    \mu_\varepsilon\leq \mu +\varepsilon\lambda<M\\
    \implies \mu_\varepsilon<M_\varepsilon
  \end{gather*}
  $\implies$ $f_\varepsilon$ hat ein Maximum $M_\varepsilon$ an einem Punkt $a\in U$
  \[\implies \md^2f_\varepsilon\leq 0\]
  \begin{gather*}
    \Delta f_\varepsilon(a)=\Spur f_\varepsilon''(a)\leq 0
  \end{gather*}
  Aber
  \begin{gather*}
    \Delta f_\varepsilon(a)=\underbrace{\Delta f(a)}_{=0}+\underbrace{\varepsilon\Delta \left( x_1^2+\cdots+x_n^2 \right)}_{2n\varepsilon>0}
  \end{gather*}
  $\implies$ $\Delta f_\varepsilon(a)$. Dies ist ein Widerspruch zu $\Delta f_\varepsilon(a)\leq 0$.
\end{Bew}
\subsection{Konvexität von Funktionen}
\begin{Def}
  Sei $F:U\to \mb{R}$, $U\subset\mb{R}^n$ $f$ heisst konvex, wenn $\forall a,b\in U$ $\forall t\in (0;1)$
  \[f\left( \left( 1-t \right)a+tb \right)\leq \left( 1-t \right)f(a)+t f(b)\]
  \begin{itemize}
    \item konvex: $\leq$
    \item konkav: $\geq$
    \item streng konvex $<$
    \item streng konkav $>$
  \end{itemize}
\end{Def}
\begin{Sat}
  Sei $f\in\mathcal{C}^2(U,\mb{R})$, $U$ konvex und offen
  \begin{align*}
    f\ \text{konvex}\iff \md^2f_a\geq 0&&\forall a\in U\\
    \md^2f_a>0\implies f\ \text{streng konvex}&&\forall a\in U
  \end{align*}
\end{Sat}
\begin{Bew}
  \[F(t)=f\left( \left( 1-t \right)a+tb \right)\]
  1-dimensional
\end{Bew}
\subsection{Parameterabhängige Integrale}
\begin{Def}
  Sei $U\subset\mb{R}^n$ und sei $f:U\times [a;b]\to\mb{C}$ s.d. $\forall x\in U:t\mapsto f(x,t)$ stetig auf $[a;b]$ $\implies$ Regelfunktion
  \[F(x):=\int_a^bf(x,t)\md t\]
  $f$ stetig $\implies$ $F$ stetig
\end{Def}
\subsubsection{Differentiationssatz}
\begin{Sat}
  Es gelte zusätzlich
  \begin{itemize}
    \item $\forall t\in [a;b]$ ist $f$ nach $x_i$ differenzierbar
    \item $\partial_if:U\times [a;b]\to\mb{C}$ stetig
  \end{itemize}
  Dann ist $F$ nach $x_i$ differenzierbar und 
  \[\partial_i F(x)=\int_a^b\partial_if(x,t)\md t\]
  d.h.
  \[\partial\int=\int\partial\]
\end{Sat}
\begin{Bem}
  Mit den Lebesgne-Integralen gilt das unter viel schwächeren Bedingungen.
\end{Bem}
