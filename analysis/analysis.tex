% headers by Alexander Berthold van der Bourg / Pirmin Weigele 

%= Document-Class ==================================================================================
\documentclass[10pt,a4paper]{article}

%= Packages ========================================================================================
\usepackage[utf8]{inputenc}
\usepackage{ngerman,amsmath,amssymb,amsfonts,mathrsfs}
\usepackage{amsthm}
\usepackage{bbm}
\usepackage{ulsy}
\usepackage{epic,eepic,pstricks,pst-node,pst-plot}
\usepackage{pstricks}
\usepackage{colortbl}
\usepackage{graphicx}
\usepackage{makeidx}
\usepackage{fancyhdr}
\usepackage{latexsym}
\usepackage{psfrag}
\usepackage{enumerate}
\usepackage{float}
%\usepackage{mathtext}
\usepackage[all, knot, poly]{xy}
\usepackage{dsfont}
\pagestyle{fancy}
\usepackage{multirow, bigdelim, bigstrut}
\usepackage{rotating}
\usepackage{ifthen}
\usepackage{boxedminipage}
\usepackage{mathtools}
\usepackage{ulsy}
\usepackage{trfsigns}

%= Seiten-Layout =========================================================================
\voffset-22mm \textheight715pt 

%Seitenbreite==============================================================

%\oddsidemargin=-0.2in
%\evensidemargin=-0.4in
%\textwidth=5.2in
%\headwidth=5.2in

%= Index-Befehle ========================================================================
\renewcommand{\indexname}{Stichwortverzeichnis}
\makeindex

%= Befehl-Overwriting =======================================================================
\makeatletter
\makeatother

%= Strings ================================================================
\newcommand{\mainfold}{.}
\newcommand{\prefix}{A1-}

%= Eigene Befehle ==========================================================================
\DeclareMathOperator{\id}{Id}
\DeclareMathOperator{\arccot}{arccot}
\DeclareMathOperator{\arcsinh}{arcsinh}
\DeclareMathOperator{\arccosh}{arccosh}
\DeclareMathOperator{\arctanh}{arctanh}
\DeclareMathOperator{\md}{d}
\DeclareMathOperator{\Grad}{grad}
\DeclareMathOperator{\Spur}{Spur}
\DeclareMathOperator{\Graph}{Graph}
\DeclareMathOperator{\sign}{sign}
\DeclareMathOperator{\Hom}{Hom}
\DeclareMathOperator{\rot}{rot}

\newcommand{\Diff}[2]{\displaystyle\frac{\mathrm{d}#1}{\mathrm{d}#2}}
\newcommand{\End}{\hfill{\hbox{$\Box$}}\par\vspace{2mm}}
\newcommand{\eps}{\varepsilon}
\newcommand{\ePic}[1]{\input{\mainfold/graphics/\prefix#1.eepic}}
\newcommand{\pst}[1]{\input{\mainfold/graphics/\prefix#1.pst}}
\newcommand{\pic}[1]{\input{\mainfold/graphics/\prefix#1.pic}}
\newcommand{\Mx}[1]{\begin{pmatrix}#1\end{pmatrix}}
%\newcommand{\im}[1]{\operatorname{Im}(#1)}
%\newcommand{\Include}[4]{\rhead{#2.#3.20#4}\input{\mainfold/lectures/#1-#4-#3-#2.tex}}
\newcommand{\Index}[1]{\emph{#1}\index{#1}}
\newcommand{\Int}[4]{\displaystyle\int\limits_{#1}^{#2}#3\,\mathrm{d}#4}
\newcommand{\diff}[1]{\operatorname{d}\!#1}
\newcommand{\Limi}[1]{\displaystyle\lim_{#1\rightarrow\infty}}
\newcommand{\Limo}[1]{\displaystyle\lim_{#1\rightarrow0}}
\newcommand{\mb}[1]{\mathbb{#1}}
\newcommand{\ds}{\displaystyle}
\newcommand{\ol}[1]{\overline{#1}}
\newcommand{\Part}[2]{\dfrac{\partial #1}{\partial #2}}
\newcommand{\QED}{\hfill{\hbox{(QED)}}\par\vspace{2mm}}
\newcommand{\re}[1]{\operatorname{Re}(#1)}
\newcommand{\s}{\hspace{2mm}}
\newcommand{\vsa}{\vspace{1mm} \\}
\newcommand{\vsb}{\vspace{2mm} \\}
\newcommand{\vsc}{\vspace{3mm} \\}
% \newcommand{\tr}[1]{\textrm{#1}}
\newcommand{\tr}[1]{\text{#1}}
\newcommand{\ra}{\rightarrow}
\newcommand{\Ra}{\Rightarrow}
\newcommand{\Lra}{\Leftrightarrow}
\newcommand{\La}{\Leftarrow}
\newcommand{\ul}[1]{\underline{#1}}
\newcommand{\rsa}{\rightsquigarrow}
\newcommand{\ara}[2]{\autorightarrow{\ensuremath{#1}}{\ensuremath{#2}}}

%\newcommand{\detmx}{\left| \begin{array} #1 \end{array} \right|}

\newcommand{\grad}[1]{\Grad(#1)}
\newcommand{\fr}[2]{\displaystyle\frac{#1}{#2}} % fertiger bullshit, daf�r gibts \dfrac{}{}
\renewcommand{\Re}{\operatorname{Re}}
\renewcommand{\Im}{\operatorname{Im}}

% ---- DELIMITER PAIRS ----
\def\floor#1{\lfloor #1 \rfloor}
\def\ceil#1{\lceil #1 \rceil}
\def\seq#1{\langle #1 \rangle}
\def\set#1{\{ #1 \}}
\def\abs#1{\mathopen| #1 \mathclose|}	% use instead of $|x|$ 
\def\norm#1{\mathopen\| #1 \mathclose\|}% use instead of $\|x\|$ 

% --- Self-scaling delmiter pairs ---
\def\Floor#1{\left\lfloor #1 \right\rfloor}
\def\Ceil#1{\left\lceil #1 \right\rceil}
\def\Seq#1{\left\langle #1 \right\rangle}
\def\Set#1{\left\{ #1 \right\}}
\def\Abs#1{\left| #1 \right|}
\def\Norm#1{\left\| #1 \right\|}

%Adrians Abbildungs-Environment ==============================================

\newcommand{\Sidein}{\begin{rotate}{90}\small$\in$\end{rotate}}

\newcommand{\Abb}[5][]{\ensuremath{
    \begin{array}{lc}
      \ifthenelse{\equal{#1}{}}{}{#1:}\;\; & 
      \begin{xy}
        \xymatrixrowsep{1em}\xymatrixcolsep{2em}%
        \xymatrix{ #2 \ar[r] \ar@{}[d]^<<<<{\hspace{0.001em} \Sidein}
          & #3  \ar@{}[d]^<<<<{\hspace{0.001em} \Sidein} \\
          #4 \ar@{|->}[r] & #5} \end{xy}
    \end{array}
  }%
}

%= Environments ========================================================================
\def\thechapter{\Roman{chapter}}
\def\thesection{\arabic{section}}
\newtheorem{Bew}{Beweis}
\newtheorem{Lem}{Lemma}
\newtheorem{Kor}{Korollar}
\newtheorem{Sat}{Satz}
\newtheorem{Prop}{Proposition}
\theoremstyle{definition}
\newtheorem{Bsp}{Beispiel}
\newtheorem{Def}{Definition}
\newtheorem{Prob}{Problem}
\theoremstyle{remark}
\newtheorem{Bem}{Bemerkung}
\newtheorem{Eig}{Eigenschaften}
\newtheorem{Not}{Notation}

\def\pstexInput#1{%
  \begin{center}
    \begin{picture}(0,0)%
      \special{psfile=\mainfold/graphics/A2-#1.pstex}%
    \end{picture}%
    \input{\mainfold/graphics/A2-#1.pstex_t}%
  \end{center}
}

%= Titelseite ===========================================================================
\begin{document}
\headheight15pt
\begin{titlepage}
\hfill
\vspace{20mm}
\pagenumbering{roman}
\begin{center}
{\LARGE Analysis I - Vorlesungs-Script} \vskip 3em {\large Prof.
Alberto Cattaneo} \vskip 1.5em
{\large Basisjahr 08/09 Semester II}\vspace{30mm}\\
{\large {\bf Mitschrift:} \vspace{2mm}\\
Simon Hafner}\vspace{5mm}\\ %30mm
%{\large {\bf Graphics:} \vspace{2mm}\\
%Pirmin Weigele }\vspace{30mm}\\ %30mm
\author{Simon Hafner}

\end{center}
\vfill

\end{titlepage}


%= Inhaltsverzeichnis ==========================================================================
\lhead{}
\rhead{}
\tableofcontents
\newpage
\pagenumbering{arabic}
\setcounter{page}{1}

%= Vorlesung-Skripts ==========================================================================
\cfoot{\thepage}
\fancyhead[L]{\nouppercase{\leftmark}}
\newpage

%= Analysis I & & II ==========================================================================

%Analysis I
\input{09-02-18}
\input{09-02-24}
\input{09-02-25}
\input{09-03-03}
\input{09-03-04}
\input{09-03-10}
\input{09-03-11}
\input{09-03-17}
\input{09-03-18}
\input{09-03-24}
\input{09-03-25}
\input{09-03-31}
\input{09-04-01}
\begin{Def}{folgenkompakt}
  Ein metrischer Raum $X$ heisst folgenkompakt, wenn jede Folge in $X$ eine konvergente Teilfolge besitzt. [Bolzano-Weierstrass-Eigenschaft]
\end{Def}
\begin{Def}
  Eine Teilmenge eines metrischen Raumes heisst folgenkompakt, wenn sie bezüglich der Spurmetrik folgenkompakt ist.
\end{Def}
\begin{Bsp}
    $\mb{R}$ ist nicht folgenkompakt (Folgen, die gegen $\infty$ konvergieren)
\end{Bsp}
\begin{Bsp}
    $[a;b]$ ist folgenkompakt (Satz von Bolzano-Weierstrass)
\end{Bsp}
\subsection{Überdeckung}
\begin{Def}{Überdeckung}
  Sei $X$ ein Menge, sei $I$ eine Indexmenge und sei $\left\{ U_i \right\}_{i\in I}$ Familie von Teilmengen von $X$. $\left\{ U_i \right\}_{i\in I}$ heisst Überdeckung von $X$, wenn $X=\bigcup_{i\in I}U_i$ d.h.
  \[\forall x\in X\ \exists i\in I:x\in U_i\]
  Sei $X$ ein metrischer Raum. Dann heisst eine Überdeckung $\left\{ U_i \right\}_{i\in I}$ offen, wenn $U_i$ offen $\forall i$ ist.
\end{Def}
\begin{Bsp}
  $x=[0;1]$
  \[\left\{ \left[0;\frac{2}{3}\right),\left(\frac{1}{3};1\right] \right\}\ \text{Überdeckung}\]
  offen bezüglich der Spurtopologie
\end{Bsp}
\begin{Bsp}
  $x=(0;1)$
  \[\left\{ \left( \frac{1}{n};1 \right) \right\}_{n\in\mb{N}_x}\ \text{offen Überdeckt}\]
\end{Bsp}
\begin{Bsp}
  $x=[0;1]$
  \begin{align*}
    U_n:=\left( \frac{1}{n};1 \right]&& n>0\\
    U_0:=\left[ 0;\frac{1}{2} \right]
  \end{align*}
  $\left\{ U_n \right\}_{n\geq 0}$ offene Überdeckung von $X$
\end{Bsp}
\begin{Def}{endliche Überdeckung}
  eine Überdeckung $\left\{ U_i \right\}_{i\in I}$ heisst endlich, wenn $I$ eine endliche Menge ist.
\end{Def}
\begin{Def}{kompakter metrischer Raum}
  Ein metrischer Raum $X$ heisst kompakt , wenn aus \underline{jeder} offenen Überdeckung von $X$ eine endliche Überdeckung ausgewählt werden kann. d.h.
  \begin{align*}
  \forall \left\{ U_i \right\}_{i\in I}\ x=\bigcup_{i\in I}U_i\ \text{offen}\\
  \exists n\in \mb{N}\ \text{und}\ \exists i_1,i_2,\cdots,i_n\in I\\
  \text{s.d.}\ X=U_{i_1}\cup U_{i_2}\cup\cdots\cup U_{i_n}=\cup_{j=1}U_{i_j}
  \end{align*}
\end{Def}
\begin{Def}{kompakte Teilmenge}
  Eine Teilmenge eines metrischen Raumes heisst kompakt, wenn sie bezüglich der Spurmetrik so ist.
\end{Def}
\begin{Sat}
  \[X\ \text{kompakt}\ \Lra\ X\ \text{folgenkompakt}\]
\end{Sat}
\begin{Bew}
  $\Ra$ Sei $(a_k)$ Folge in $X$. Zu zeigen: $(a_k)$ besitzt eine konvergente Teilfolge.
  \[A:=\left\{ a_k,k\in\mb{N} \right\}\]
  \subparagraph{Fall 1}$A$ ist endlich $\implies$ $(a_k)$ besitzt eine konstante Teilfolge.
  \subparagraph{Fall 2}$A$ unendlich
  \begin{Lem}
    $A$ besitzt einen Häufungspunkt.\\
    $A$ besitzt keinen Häufungspunkt.
    \[\forall x\in X\ \exists U(x)\ \text{Umgebung von $x$}\]
    s.d.
    \[U(x)\cap A=\begin{cases}
      \varnothing&x\not\in A\\
      \left\{ x \right\}&x\in A
    \end{cases}\]
    Zudem:
    \begin{gather*}
    \forall x\in U(x)\ \bigcup_{x\in A}U(x)=X\\
    \left\{ U(x) \right\}_{x\in X}\ \text{ist eine offene Überdeckung von $X$}\\
    \xRightarrow{\text{$X$ kompakt}}\\
    \exists n:\exists x_1,\cdots,x_n\in X\ \text{s.d.}\ X=U(x_1)\cup\cdots\cup U(x_n)\\
    A=X\cap A=\left( U(x_1)\cup\cdots\cup U(x_n) \right)\cap A=\left\{ x_i:x_i\in A \right\}\subset \left\{ x_i \right\}\\
    \implies A\ \text{endlich}\implies \text{Widerspruch!}
    \end{gather*}
  \end{Lem}
  Sei $a$ Häufungspunkt von $A$ $\implies$
  \begin{gather*}
    \forall\mu\in \mb{N}:K_{\frac{1}{\mu}}(a)\ni a_{k_\mu}\in A\setminus\left\{ a \right\}\\
    (a_{k_\mu})\ \text{Teilfolge}, (a_{k_\mu})\in K \implies \Limi{\mu}a_{k_\mu}=a
  \end{gather*}
\end{Bew}
\begin{Def}{beschränkt}
  Sei $X$ ein metrischer Raum, $\mb{K}\subset X$. $\mb{K}$ heisst beschränkt, wenn 
  \[\exists x\in X\ \exists r>0:\mb{K}\subset K_r(x)\]
\end{Def}
\begin{Lem}
  Sei $X$ ein metrischer Raum, $\mb{K}\subset X$
  \[\mb{K}\ \text{folgenkompakt} \implies\text{$\mb{K}$ beschränkt und abgeschlossen}\]
\end{Lem}
\begin{Bew}
  Sei $\mb{K}$ nicht beschränkt oder nicht abgeschlossen.
  \subparagraph{Fall 1}$\mb{K}$ nicht beschränkt\\
  Sei $x\in \mb{K}$. Da $\mb{K}$ nicht beschränkt
  \[\forall k\exists x_k\in \mb{K}:d(x_k,x)>k\]
  (sonst wäre $\mb{K}\subset K_k(x)$)
  $(x_k)$ besitzt keine konvergente Teilfolge. Sonst:
  \[x_{k_i}\xrightarrow{i\to\infty}x\implies d(x_k,x)\to 0\]
  (was aber nicht möglich ist, da der Abstand immer grösser wird)
  \subparagraph{Fall 2}$\mb{K}$ nicht abgeschlossen
  \[\exists (x_k),x_k\in \mb{K}\forall k\ \text{und}\ x_k\in x\not\in X\]
  $\implies$ jede Teilfolge von $(x_k)$ konvergiert gegen $x\in X$.
\end{Bew}
\begin{Bem}
  $\mb{K}$ folgenkompakt $\implies$ $\mb{K}$ abgeschlossen und beschränkt.\\
  Im allgemeinen $\not\La$
\end{Bem}
\begin{Bsp}
  $X=\mathcal{C}\left( [0;\pi],\mb{C} \right)$ mit Supremumsnorm
  \[\mb{K}=K_1(0)=\left\{ f\in X:\overbrace{\Norm{f}}^{\sup\Abs{f}}\leq 1 \right\}\]
  $\mb{K}$ ist abgeschlossen
  \[\overline{K_1(0)}\subset K_2(0)\]
  \ldots und beschränkt.
  \begin{align*}
    e_k(x):=e^{ikx}&&\\
    e_k\in \mb{K}\ \forall k&&\\
    \Norm{e_k-e_l}=2\ \forall k,l
  \end{align*}
  \begin{Bew}
    \begin{gather*}
      \Abs{e_k(x)-e_l(x)}^2=\left( e^{-ikx}-e^{ilx} \right)\left( e^{ikx}-e^{ilx} \right)=\\
      =1-e^{i(l-k)x}-e^{i(k-l)x}+1 = 2\left( 1-cos(k-l) \right)
    \end{gather*}
    Maximum 4 wenn $\cos = -1$, $\sup\Abs{e_k-e_l}=2$ $\implies$ jede Teilfolge $e_k$
    \[\Norm{e_{ki}-e_{kj}}=2\ \forall i,j\]
    keine Cauchyfolge. Keine Teilfolge ist Cauchy. $\implies$ keine Teilfolge konvergiert
  \end{Bew}
\end{Bsp}
\begin{Sat}
  Sei $V$ ein \underline{endlichdimensionaler} normierter Vektorraum, sei $\mb{K}\subset V$. Dann sind folgende Aussagen equivalent:
  \begin{enumerate}
    \item $\mb{K}$ ist beschränkt und abgeschlossen
    \item $\mb{K}$ kompakt
    \item $\mb{K}$ ist folgenkompakt
  \end{enumerate}
  zu zeigen: $1.\implies 2.$
\end{Sat}
\begin{Sat}
  Sei $X$ kompakt und $A\subset X$ abgeschlossen. Dann ist $A$ kompakt.
\end{Sat}
\begin{Bew}
  Sei $\left\{ U_i \right\}_{i\in I}$ offene Überdeckung von $A$.
  \begin{gather*}
    U_i\ \text{offen in}\ A\implies \exists V_i\subset X\ text{offen, mit}\ U_i=A\cap U_i\\
    \bigcup_{i\in I}U_i=A \implies \bigcup_{i\in A}V_i\supset A\\
    X=X\setminus A\cup \bigcup_{i\in I}V_i\\
    X\setminus A,V_i\ \text{Überdeckung von $X$}\\
    A\ \text{abgeschlossen}\implies X\setminus A\ \text{offen}\\
    X\setminus A,V_i\ \text{offene Überdeckung}\\
    X\ \text{kompakt}\implies\ \exists n:i_1,\cdots,i_n: X=X\setminus A\cup V_{i_1}\cup V_{i_1}\cup\cdots\cup V_{i_n}\\
    \implies U_{i_1},\cdots,U_{i_n}\ \text{Überdeckung von $A$}
  \end{gather*}
\end{Bew}
\subsection{Existenz von Maxima und Minima}
\begin{Sat}
  Sei $f:X\to Y$ stetig ($X,Y$ metrische Räume)
  \[X\ \text{kompakt}\implies f(x)\ \text{kompakt}\]
\end{Sat}
\begin{Bew}
  Sei $\left\{ U_i \right\}_{i\in I}$ eine offene Überdeckung von $f(x)$ $V_i:=f^{-1}(U_i)$ $\implies$ $\left\{ V_i \right\}_{i\in I}$ offene Überdeckung von $X$.
  \[\implies\exists n:i_1,\cdots,i_n\in I\ X=v_{i_1}\cup\cdots\cup V_{i_n}\implies f(x)=U_{i_1}\cup\cdots\cup U_{i_n}\]
\end{Bew}
\begin{Sat}{von Maxima und Minima}
  Sei $f:x\to\mb{R}$ stetig und $X$ kompakt. Dann nimmt $f$ ein Maximum und ein Minimum an.
\end{Sat}
\begin{Bew}
  $f$ stetig und $X$ kompakt $\implies$ $f(x)\subset\mb{R}$ kompakt. $\implies$ $f(x)$ beschränkt und abgeschlossen.\\
  beschränkt $\implies$ $f(x)$ besitzt ein Supremum und ein Infimum\\
  abgeschlossen $\implies$ $\sup, \inf f\in f(x)$
\end{Bew}
\begin{Def}{gleichmässig stetig}
  $f:X\to Y$, ($X,Y$ metrische Räume) heisst gleichmässig stetig, wenn
  \[\forall\varepsilon>0\ \exists \delta>0:\forall x_1,x_2\subset X\ \text{mit}\ d_x(x_1,x_2)<\delta\]
  gilt
  \[d_y\left( f(x_1),f(x_2) \right)<\varepsilon\]
\end{Def}
\begin{Bem}
  $f$ gleichmässig stetig $\implies$ $f$ stetig
\end{Bem}
\begin{Sat}
  Sei $f:X\to Y$ $X$ kompakt
  \[f\ \text{stetig}\implies f\ \text{gleichmässig stetig}\]
\end{Sat}
\begin{Bew}
  Wie im Falle $X\subset\mb{R}$  
\end{Bew}
\begin{Lem}{Tubenlemma}
  Sei $X$ ein metrischer Raum, $\mb{K}$ ein kompakter Raum, $x_0\in X$, $W\subset X\times \mb{K}$ offen mit $\left\{ x_0 \right\}\times\mb{K}\subset W$.\\
  Dann $\exists$ Umgebung von $x_0$ in $X$ s.d. \[U\times\mb{K}\subset W\]
\end{Lem}
\begin{Bew}
  $W$ offen in der Produkttopologie.
  \[\forall y,x\in \mb{K}, \left( x_0,y \right)\in W\]
  $\exists$ Umgebung von $U_y$ von $x_0$ in $X$
  $\exists$ Umgebung von $V_y$ von $x_0$ in $\mb{K}$
  mit $U_y\times V_y\subset W$
  \begin{gather*}
    \bigcup_{y\in\mb{K}}V_j=\mb{K}\\
    y\in V_y\ \forall y\\
    \left\{ V_y \right\}_{y\in \mb{K}}\ \text{offene Überdecktung von $\mb{K}$}\ \text{$\mb{K}$ kompakt}\\
    \implies \forall n,u_1,\cdots,u_n\in\mb{K},\ \mb{K}=V_{y_1}\cup\cdots\cup V_{y_n}\\
    U:=U_{y_1}\cap U_{y_2}\cap\cdots\cap U_{y_n}\\
    U\ni x_0\\
    U\times\mb{K}\subset W\\
    U\ \text{\underline{offen}}
  \end{gather*}
\end{Bew}
\begin{Kor}
  $\mb{K}$ kompakt und $L$ kompakt $\implies$ $\mb{K}\times L$ kompakt
\end{Kor}

\input{09-04-08}
\section{Differenzierbare Funktionen (Kap 2)}
\begin{Bem}
  Sei $F:\underbrace{U}_{\subset\mb{R}^n}\to\mb{C}$ oder $\mb{R}$. $U$ offen. Die lineare Approximation von $f$ im Punkt $a\in U$ ist eine Funktion der Form
  \[Tf(x;a):=f(a)+L(x-a)\]
  wobei $L:\mb{R}^n\to\mb{R}$ \underline{linear} s.d. der Rest ($x=a+h\in U$)
  \[R(\underbrace{h}_{\mb{R}^n}):=f(a+h)-Tf(a+h;a)\]
  erfüllt
  \[\Limo{h}\frac{R(h)}{Norm{h}}=0\]
  (wobei $\Norm{\ }$ irgendeine Norm auf $\mb{R}^n$ ist)\\
  $U$ offen $\implies$ $\exists \varepsilon>0$ s.d.
  \[K_\varepsilon(a)\subset U\]
  \[\implies a+h\in U\ \forall h:\Norm{h}<\varepsilon\]
\end{Bem}
\begin{Def}{differenzierbar in $a$}
  $f:U\to\mb{C}$, $U$ offen in $\mb{R}$ heisst differenzierbar in $a$, wenn es eine lineare Abbildung $L:\mb{R}^n\to\mb{C}$ gibt s.d.
  \[\lim\frac{R(h)}{\Norm{h}}=0\]
  d.h.
  \[\Limo{h}\frac{f(a+h)-f(a)-Lh}{\Norm{h}}=0\]
\end{Def}
\begin{Def}{Tangentialhyperebene}
  Der Graph von $Tf$
  \[\left\{ \left( x,y \right)\in \mb{R}^n\times \mb{C}:y=Tf(x;a) \right\}\]
  heisst die Tangentialhyperebene von $f$ in $(a,f(a))$
\end{Def}
\begin{Bem}
  $h=1$: euklidische Definition
\end{Bem}
\begin{Bem}
  Wichtig: $U$ offen!
\end{Bem}
\begin{Bem}
  Es spielt keine Rolle, welche Norm verwendet wird.
\end{Bem}
\begin{Bem}
  $L:\mb{R}^n\to\mb{R}$ oder $L:\mb{R}^n\to\mb{C}$, $L$ linear, d.h.
  \[L\in\Hom\left( \mb{R}^n,\mb{C} \right)\]
  $\mb{C}$ wird als reeller Vektorraum betrachtet.
  \[L\in\Hom\left( \mb{R}^n,\mb{R} \right)=:\mb{R}^{h*}\]
  d.h. Linearform
\end{Bem}
\begin{Def}{Linearisierung}
  Die lineare Abbildung $L$ heisst Linearisierung von $f$ im Punkt $a$.
\end{Def}
\begin{Lem}
  Ist $f$ differenzierbar in $a$, so ist ihre Linearisierung eindeutig bestimmt.
\end{Lem}
\begin{Bew}
  Seien $L$ und $L^*$ Linearisierungen von $f$ in $a$.
  \begin{gather*}
    \Limo{h} \frac{f(a+h)-f(a)-Lh}{\Norm{h}}=0\\
    \Limo{h} \frac{f(a+h)-f(a)-L^*h}{\Norm{h}}=0
  \end{gather*}
  Differenz:
  \[\Limo{h}\frac{(L^*-L)(h)}{\Norm{h}}=0\]
  $h=tv$, $v\in\mb{R}^n$, $t\in\mb{R}$
  \[\Limo{t}\frac{(L^*-L)(tv)}{\Abs{t}\not\Norm{v}}=0\]
  $L^*-L$ linear $\xRightarrow{\text{endlichdim}}$ $L^*-L$ stetig
  \[(L^*-L)\left( \Limo{t}\frac{\not t v}{\not t} \right)=\left( L^*-L \right)(v)\]
  \begin{gather*}
    \left( L^*-L \right)(v)=0\ \forall v\in\mb{R}^n,\ \Norm{v}=1\\
    \forall h\in\mb{R}^n:h=tv,\ \Norm{v}=1\\
    \implies \left( L^*-L \right)(h)=0\ \forall h\in\mb{R}^n\implies L^*=L
  \end{gather*}
\end{Bew}
\begin{Def}{Differenzial}
  Die Linearisierung $L$ von $f$ im Punkt $a$ bezeichnet man auch mit 
  \[\md f_a\ \text{oder}\ \md f(a)\]
  Differenzial von $f$ im Punkt $a$
  \[Tf(x;a)=f(a)+\md f_a(x-a)+R(x-a),\ \Hom\left( \mb{R}^n,\mb{C} \right)\]
\end{Def}
\begin{Bem}
  Sei $\left\{ e_1,\cdots,e_n \right\}$ die Standardbasis von $\mb{R}^n$. $\forall h\in \mb{R}$
  Sei
  \[f'(a):=\left( \md f_ae_1,\md f_ae_2,\cdots,\md f_ae_n \right)\in M\left( n\times 1,\mb{C} \right)\]
  Dann
  \[\md f_ah:=f'(a)\cdot \Mx{h_1\\h_2\\\vdots\\h_n}\]
  \[h=\sum^n_{i=1}h_ie_i,\ h\in\mb{R}^n\]
\end{Bem}
\begin{Bem}
  $n=1$ $f'(a)\in\mb{C}$ übliche Ableitung
\end{Bem}
\begin{Sat}
  Ist $f$ differenzierbar im Punkt $a$, so ist $f$ stetig im Punkt $a$.
\end{Sat}
\begin{Bew}
  \begin{gather*}
    f(a+b)-f(a)=Lh+R(h)\\
    \Limo{h}\frac{R(h)}{\Norm{h}}=0\\
    \implies\Limo{h}R(h)=0\\
    L\ \text{linear $n\infty$} \implies L\ \text{stetig}\\
    \implies \Limo{h}Lh=0\\
    \implies\Limo{h}\left( f(a+h)-f(a) \right)=0\\
    \implies f\ \text{stetig in}\ a
  \end{gather*}
\end{Bew}
\begin{Def}{differenzierbar auf $U$}
  $f:U\to\mb{C}$, $U$ offen in $\mb{R}^n$ heisst differenzierbar auf $U$, wenn Sie $\forall a\in U$ differenzierbar ist. In diesem Falle:
  \begin{align*}
    \md f:U&\to\Hom\left( \mb{R}^n,\mb{C} \right)\\
    a&\mapsto\md f_a
  \end{align*}
\end{Def}
\begin{Bsp}
  Sei $A\in M(n\times 1,\mb{R})$, sei $b\in\mb{R}^n$, $f:\mb{R}^n\to\mb{R}$
  \[f(\underbrace{x}_{\in\mb{R}^n}):=Ax+b\]
  $f$ ist auf ganz $\mb{R}$ differenzierbar und 
  \begin{gather*}
    f'(a)=A\ \forall a\in\mb{R}^n
    f(a+h)-f(a)=A(x+h)-Ax=ah
  \end{gather*}
  \[\Limo{h}\frac{f(a+h9-f(a)-Ah}{\Norm{h}}=0\]
  $L(h)=Ah$
\end{Bsp}
\begin{Bsp}
  Sei $A\in M(n\times n,\mb{R})$, $f:\mb{R}^n\to\mb{R}$
  \[f(x):=x^TAx\]
  \begin{gather*}
    f(x+h)-f(x)=(x+h)^TA(x+h)-x^TAx=x^TAx+x^TAh+h^Ax+h^TAh-x^TAx=\\
    =\underbrace{x^TAh+h^Ax}_{\text{linear in $h$}}+\underbrace{h^TAh}_{R(h)}\\
    L(h)=x^tAh+h^TAx
  \end{gather*}
  Zu zeigen
  \[\lim\frac{R(n)}{\Norm{h}}=0\]
  Sei $\sigma=\max_{i,j}\Abs{a_{ij}}$
  \begin{gather*}
    \Abs{h^TAh}=\Abs{\sum_{i,j}h_ia_{ij}h_j}\leq\sum_{ij}\Abs{h_i}\Abs{a_ij}\Abs{h_j}\leq\\
    \leq \sigma\sum_i\Abs{h_i}\sum_j\Abs{h_j}=\sigma\Norm{h}^2_1\\
    \frac{\Abs{R(h)}}{\Norm{h}_1}\leq\sigma\ \Norm{h}_1\to 0
  \end{gather*}
  $\implies$ $f$ differenzierbar
  \begin{gather*}
    L(h)=x^TAh+x^TA^Th=x^T(A+A^T)h\\
    \implies f'(x)=x^T(A+A^T)
  \end{gather*}
  Ist $A$ symmetrisch (d.h. $A^T=A$), so
  \[f'(x)=2x^TA\]
\end{Bsp}
\subsection{Berechnung von Ableitungen}
\begin{Def}{differenzierbar in Richtung eines Vektors}
  Sei $f:U\to\mb{C}$, $U$ offen in $\mb{R}^n$, ($f$ nicht notwendigerweise differenzierbar), sei $h\in\mb{R}^n$. $f$ heisst differenzierbar im Punkt $a$ in Richtung des Vektors $h$ wenn der Grenzwert
  \[\Limo{t}\frac{f(a+th)-f(a)}{t}\]
  existiert.
\end{Def}
\begin{Def}{Ableitung in Richtung eines Vektors}
  In diesem Falle heisst dieser Grenzwert die Ableitung von $f$ im Punkt $a$ in Richtung des Vektors $h$
\end{Def}
\begin{Not}{Richtungsableitnug}
  $\partial_hf(a)$
\end{Not}
\begin{Not}{partielle Ableitung}
  Standardbasis $\left\{ e_1,\cdots,e_n \right\}$
  \[\Part{f}{x_i}(a)=\partial_if(a):=\partial_{e_i}f(a)\]
\end{Not}
\begin{Bem}
  $\Part{f}{x}$ berechnen: $x_i$ als Variable, $x_j$, $j\neq i$ als Konstanten
\end{Bem}
\begin{Bsp}
  $f(x,y)=y^2\sin(x+y)$
  \begin{gather*}
    \Part{f}{x}=y^2\cos(x+y)\\
    \Part{f}{y}=2y\sin(x+y)+y^2\cos(x+y)
  \end{gather*}
\end{Bsp}
\begin{Def}{partiell differenzierbar}
  $f$ heisst partiell differenzierbar im Punkt $a$, wenn alle partielle Ableitungen $\partial_1f(a),\cdots,\partial_nf(a)$ existieren.
\end{Def}
\begin{Sat}
  Sei $f$ differenzierbar im Punkt $a$. Dann
  \begin{enumerate}
    \item $f$ besitzt alle Richtungsableitungen in $a$
      \[\partial_hf(a)=\md f_a(h),\ \forall h\in\mb{R}^n\]
    \item $f$ ist partiell differenzierbar in $a$
    \item \[f'(a)=\left( \partial_1f(a),\cdots,\partial_nf(a) \right)\]
  \end{enumerate}
\end{Sat}
\begin{Bew}
  \[f(a+th)=f(a)+\md f_a(th)+R(th)\]
  \begin{Lem}
    \[\Limo{t}\frac{R(th)}{t}=0\]
  \end{Lem}
  \begin{Bew}
    \[\Limi{h}\frac{R(n)}{\Norm{h}}=0\implies R(0)=0\]
    $h\neq 0$, $v=tk$
    \begin{gather*}
      \lim\frac{R(v)}{\Norm{v}}=0\\
      \Limo{t}\frac{R(th)}{\Abs{t}\Norm{h}}=0\\
      \frac{1}{\Norm{h}}\Limo{t}\frac{R(th)}{\Abs{t}}=0
    \end{gather*}
  \end{Bew}
  \begin{gather*}
    \partial_nf(a)=\Limo{t}\frac{f(a+th)-f(a)}{t}=\Limo{t}\frac{\md f_a(th)+R(th)}{t}=\\
    \Limo{t}\frac{\md f_a(th)}{t}\stackrel{\text{linear}}{=}\Limo{t}\frac{\not t \md f_a(h)}{\not t}=\md f_a(h)
  \end{gather*}
  1 $\implies$ 2\\
  3
  \begin{gather*}
    \partial_if(a)=\partial_ef(a)=\md f_a(e_i)=f(a)\Mx{0\\0\\\vdots\\0\\1\\0\\\vdots\\0}=\left( f'(a) \right)_i
  \end{gather*}
\end{Bew}
\begin{Bem}
  differenzierbar $\implies$ partiell differenzierbar, im Allgemeinen gilt die Umkehrung nicht! Aber die partielle Differenzierbarkeit ist eine notwendige Bedingung für die Differenzierbarkeit.
\end{Bem}
\begin{Bem}
  $Lh=\left( \partial_1f_1,\cdots,\partial_nf \right)h$ ist der einzige Kandidat für die Linearisierung.
\end{Bem}
\begin{Bsp}
  von partiell differenzierbaren Funktionen, die \underline{nicht} differenzierbar sind
  \[f(x,y)=\begin{cases}
    \frac{xy}{x^2+y^2}&(x,y)\neq (0,0)\\
    0&(x,y)=0
  \end{cases}\]
  \[f\ \text{nicht stetig in}\ (0,0)\implies f\ \text{nicht differenzierbar in} (0,0)\]
  \begin{gather*}
    \partial_xf(0)=\Limo{t}\frac{\overbrace{f(t,0)}^{=0}-\overbrace{f(0,0)}^{=0}}{t}=0\\
    \partial_yf(0)=0
  \end{gather*}
  $f$ partiell differenzierbar in $(0,0)$
\end{Bsp}
\begin{Bem}
  $\partial_nf(0)$ existiert nicht für $h\neq\Mx{1\\0}$, $h\neq\Mx{0\\1}$
\end{Bem}


\begin{Bsp}
  $h=\Mx{h_1\\h_2}\in\mb{R}^2$
  \begin{gather*}
    \partial_nf(0,0)=\Limo{t}\frac{\left( th_1,th_2 \right)-f(0,0)}{t}\\
    =\Limo{t}\frac{1}{t}\frac{t^3h^2_1h_2}{t^2\left( h_1^2+h_2^2 \right)}=f(h_1,h_2)
  \end{gather*}
  \begin{gather*}
    \partial_xf(0,0)=f(1,0)=0\\
    \partial_gf(0,0)=f(0,1)=0\\
  \end{gather*}
  \[\Limo{n}\frac{f(h_1,h_2)-f(0,0)-\overbrace{L}^{=0}f}{\Norm{h}}=0\ ?\]
  Nein. z.Z.: $h_1=h_2=:k$
  \begin{gather*}
    f(k,k)=\frac{k^3}{2k^2}=\frac{k}{2}\\
    \Norm{\Mx{k\\k}}_\infty=\Abs{k}\\
    \frac{f(k,k)}{\Norm{\Mx{k\\k}}_\infty}=\pm\frac{1}{2}\not\to 0
  \end{gather*}
  $\implies$ $f$ nicht differenzierbar in $(0,0)$
\end{Bsp}
\subsection{Differenzierbarkeitskriterium}
\begin{Sat}{Differenzierbarkeitskriterium}
  Sei $f:D\to\mb{C}$ und sei $a\in D$.
  \begin{enumerate}
    \item Es gibt eine Umgebung von $a$, s.d. $\forall x\in U$ ist $f$ in $x$ partiell differenzierbar.
    \item Alle partiellen Ableitungen sind im Punkt $a$ stetig
  \end{enumerate}
  $\implies$ $f$ ist in $a$ differenzierbar
\end{Sat}
\begin{Bew}{Idee}
  \[\frac{f(a+h)-f(a)-Lb}{\Norm{h}}\to 0\]
  \[f(a+h)-f(a)=\left[ f(a+h)-f(a+h_1) \right]+\left[ f(a+h_1)+f(a) \right]\]
  Mithilfe des Mittelwertsatzes kann dieser Betrag abgeschätzt werden.
\end{Bew}
\begin{Def}
  Sei $f:\underbrace{U}_{\subset\mb{R}^n}\to \mb{C}$, differenzierbar auf $U$
  \begin{align*}
    \md f: U&\to \Hom(\mb{R}^n,\mb{C})\\
    x&\mapsto\md f_x
  \end{align*}
  $f$ heisst stetig differenzierbar auf $U$ falls $\md f:U\to\Hom(\mb{R}^n,\mb{C})$ stetig ist.
  \begin{align*}
    &\xrightarrow{\sim}M(n\times 1,\mb{C})\xrightarrow{\sim}\mb{C}^n\\
    &\mapsto\left( \partial_1 f(x),\cdots,\partial_n f(x) \right)\\
  \end{align*}
  \[\implies\md f:U\to\Hom(\mb{R}^n,\mb{C})\ \text{stetig}\iff f':U\to M(n\times 1,\mb{C})\iff \partial_if\ \text{stetig}\forall i\]
\end{Def}
\begin{Kor}
  \[\text{$f$ stetig diff}\iff\text{Alle partiellen Ableitungen sind stetig}\]
\end{Kor}
\begin{Kor}
  Sei $f:\underbrace{U}_{\subset\mb{R}^n}\to \mb{C}$
  \[\text{$f$ stetig differenzierbar auf $U$}\iff\ \text{Alle partiellen Ableitungen in $U$ existieren und sind stetig}\]
\end{Kor}
\begin{Def}
  \[\mathcal{C}^1(U):=\left\{ \text{stetig differenzierbar Funktionen auf $U$} \right\}\]
  Vektorraum. genauer:
  \begin{align*}
    \mathcal{C}^1(U,\mb{R})=\left\{ \text{reellwertig stetig differenzierbar $f$ auf $U$} \right\}\\
    \mathcal{C}^1(U,\mb{C})=\left\{ \text{komplexwertig stetig differenzierbar $f$ auf $U$} \right\}
  \end{align*}
\end{Def}
\begin{Bem}
  Sei $f:\underbrace{U}_{\subset\mb{R}^n}\to \mb{R}$, differenzierbar in $a\in U$
  \[\md f(a)\in\Hom(\mb{R}^n,\mb{R})=\mb{R}^{n*}\]
\end{Bem}
\begin{Bem}
  Sei $\left\langle , \right\rangle$ ein Skalarprodukt
  \begin{align*}
    \phi{\left\langle , \right\rangle}:&\mb{R}^n\to\mb{R}^{n*}&\text{linear}\\
    &v\mapsto\phi_{\left\langle , \right\rangle}(v)&\text{Isomorphimsus}
  \end{align*}
  \[\phi_{\left\langle , \right\rangle}(v)(w):=\left\langle v,w \right\rangle \]
\end{Bem}
\subsection{Gradient}
\begin{Def}{Gradient}
  Der Gradient von $f$ in $a$ bezüglich $\left\langle , \right\rangle$ ist
  \[\phi_{\left\langle , \right\rangle}^{-1}\left( \md f_a \right)=:\grad f(a)\]
  \[\grad f(a)\in\mb{R}^n\]
\end{Def}
\begin{Bem}
  \[\left\langle \grad f(a),w \right\rangle =\phi\left( \grad f(a) \right)(w)=\md f_a(w)=\partial_wf(a)\]
\end{Bem}
\begin{Bem}
  \[\partial_wf(a)=\left\langle \grad f(a),w \right\rangle \ \forall w\in\mb{R}^n\]
\end{Bem}
\begin{Bem}{Spezialfall Standardskalarprodukt}
  $\left\langle x,y \right\rangle =\sum x_iy_i$
  \begin{align*}
    \phi_{\left\langle , \right\rangle}:&\mb{R}^n\to\mb{R}^{n*}\\
    &\Mx{x_1\\\vdots\\x_n}\to\Mx{x_1,\cdots,x_n}
  \end{align*}
\end{Bem}
\begin{Not}{Gradient}
  Der Gradient bez. $\left\langle  \right\rangle_\text{Stand.}$ bezeichnet man als $\nabla f$
  \[\nabla f(a)=\Mx{\partial_1 f(a)\\\vdots\\\partial_n f(a)}\]
\end{Not}
\begin{Bem}{Zusammenfassung}
  \[\md f_1\in\mb{R}^{n*}\]
  \[\grad f(a)\in\mb{R}^n\]
  \begin{tabular}[htb]{c}
  Standardbasis $\mapsto$ Standardskalarodukt\\
  $f'(a)$ $n$-Zeilenvektor\\
  $\nabla f(a)$ $n$-Spaltenvektor
  \end{tabular}
\end{Bem}
\subsubsection{Geometrische Bedeutung des Gradienten}
\begin{Bem}
  Sei $h\in\mb{R}^n$
  \[\partial_n f(a)=\left\langle \grad f(a),h \right\rangle\]
  Cauchy-Schwarz
  \[\Abs{\partial_nf(a)}\leq\Norm{\grad f(a)}\Norm{h}\]
  \[\Norm{w}:=\sqrt{\left\langle w,w \right\rangle }\]
  die durch $\left\langle , \right\rangle$ induzierte Norm
  \[-\Norm{\grad f}\Norm{a}\leq \partial_n f\leq \Norm{\grad f}\Norm{h}\]
  \begin{align*}
    \partial_nf=\Norm{\grad f}\Norm{h}\xi&\xi\in [-1;1]
  \end{align*}
  \[\exists\phi:\xi=\cos\phi\]
  \[\partial_nf(a)=\Norm{\grad f(a)}\Norm{h}\cos \phi\]
\end{Bem}
\begin{Bem}
  Sei $h$: $\Norm{h}=1$
  \[\partial_nf(a)=\Norm{\grad f(a)}\cos \phi\]
  \[\implies\Norm{\grad f(a)}=\max\left\{ \partial_n f(a),\underbrace{\Norm{h}}_{\text{hängt von $\left\langle , \right\rangle$ ab}}=1 \right\}\]
  Sei $\grad f(a)\neq 0$ ($\iff \md f_a\neq 0$)
  \[\implies \exists !h\text{mit}\ \Norm{h}=1\text{und}\ \Norm{\grad f(a)}=\partial_n f(a)\]
  Nämlich
  \[h=\frac{\grad f(a)}{\Norm{\grad f(a)}}\]
  d.h. $\grad f(a)$ zeigt die Richunt des stärksten Anstiegs von $f$ in Punkt $a$.
\end{Bem}
\subsection{Rechenregeln}
\begin{Bem}{Rechenregeln}
  Sei $f,g:\underbrace{U}_{\subset\mb{R}^n}\to \mb{C}$, differenzierbar in $a\in U$. Dann
  \begin{enumerate}
    \item $f+g$ und $fg$ sind differenzierbar in $a$ und
      \begin{align*}
        \md(f+g)_a&=\md f_a+\md g_a\\
        \md(fg)_a&=f(a)\md g_a+g(a)\md f_a
      \end{align*}
    \item Sei zusätzlich $f(a)\neq 0$. Dann ist $\frac{1}{f}$ in $a$ differenzierbar und
      \[\md\left( \frac{1}{f}_a \right)=-\frac{\md f_a}{f(a)^2}\]
  \end{enumerate}
\end{Bem}
\begin{Bem}{Folgerung}
  Jede rationale Funktion ist in ihrem Definitionsbereich stetig differenzierbar.
\end{Bem}
\begin{Sat}{Kettenregel}
  Sei $U\subset\mb{R}$ offen. Seien
  \begin{align*}
    \gamma:I&\to U&\text{differenzierbar in}\ t_0\in I\\
    f:U&\to\mb{C}&\text{differenzierbar in}\ a:=f(t_0)
  \end{align*}
  Dann ist $f\circ \gamma:I\to\mb{C}$ differenzierbar in $t_0$ und
  \[\frac{\md\left( f\circ \gamma \right)}{\md t}=\md f_a\dot \gamma(t_0)=\sum_{i=1}^n\partial_if(a)\dot\gamma_i(t_0)\]
  Ist $\left\langle , \right\rangle$ vorhanden
  \[\frac{\md(f\circ \gamma)}{\md t}(t_0)=\left\langle \grad f(a),\dot\gamma(t_0) \right\rangle\]
\end{Sat}
\begin{Bem}
  Kettenregel für partielle Ableitung. Seien
  \begin{align*}
    f:U&\to\mb{C}&U\subset\mb{R}^n\\
    g:V\to U&V\subset\mb{R}^m
  \end{align*}
  \[F:=f\circ g:V\to\mb{C}\]
  $(x_1,\cdots,x_n)$ Basen auf $U$ und $(y_1,\cdots,y_m)$ Basen auf $V$. Wir wollen $\frac{\partial F}{\partial y_i}$. Seien $y_j$ mit $j\neq i$ festgelegt. Sei
  \begin{align*}
  g_{(i)}:y_i&\mapsto g(y_1,\cdots,y_n)=\left( g_+(y\cdots),\cdots,g_m(\cdots) \right)\\
  g_{(i)j}:y_i&\to g_j(y\cdots y)
  \end{align*}
  \begin{gather*}
    \frac{\md g_{(i)j}}{\md y_i}=\frac{\partial g_j}{\partial y_i}\\
    \frac{\partial F}{\partial y_i}=\frac{\md}{\md y_i}\left( f\circ g_{(i)} \right)=\sum^n_{j=1}\frac{\partial f}{\partial x_j}\frac{\md g_{(i)j}}{\md y_i}=\sum^n_{j=1}\frac{\partial f}{\partial x_j}\frac{\partial g_j}{y_i}
  \end{gather*}
  $F=f\circ g$
  \[\frac{\partial F}{\partial y_i}(y)=\sum^n_{j=1}\frac{\partial f}{\partial x_j}\left( g(y) \right)\frac{\partial g_j}{\partial y_i}(y)\]
\end{Bem}
\begin{Bsp}{Polarkoordinaten}
  Sei $f:\mb{R}^2\to\mb{C}$
  \[P_2(r,\phi)=\Mx{r\cos \phi\\r\sin\phi}\]
  $F=f\circ P_2$
  \[\frac{\partial F}{\partial r}\partial_xf\cos\phi+\partial_yf\sin\phi\]
  \[\frac{\partial F}{\partial \phi}-\partial_xf+\sin\phi+\partial_yf+\cos\phi\]
\end{Bsp}
\begin{Bsp}
  Sei $F:\mb{R}^+_*\to\mb{C}$ und $f:\mb{R}^n\to\mb{C}$ differenzierbare Funktionen
  \[f(x):=F\left( \Norm{x} \right)\]
  \begin{gather*}
    \frac{\partial f}{\partial x_i}=F\frac{\partial\Norm{x}}{\partial x_i}\\
    \frac{\partial}{\partial x_i}\Norm{x}=\frac{\partial}{\partial x_i}\sqrt{x_1^2+x_2^2+\cdots x_n^2}=\frac{1}{\not2\sqrt{x_1^2+\cdots+x_n^2}}\not 2x_i
  \end{gather*}
  \[\Part{\Norm{x}}{x_i}=\frac{x_i}{\Norm{x}}\]
  \[\Part{f}{x_i}=F'\frac{x_i}{\Norm{x}}\]
\end{Bsp}
\subsection{Niveaumengen}
\begin{Def}{Niveaumengen}
  Sei $f:U\to\mb{R}\ni c$. Die Fasern $f^{-1}(c)$ heissen Niveaumengen von $f$.
\end{Def}
\begin{Sat}
  \[\gamma(I)\subset f^{-1}(c)\]
  Sei $\left\langle , \right\rangle$ Skalarprodukt. Dann
  \begin{align*}
    \grad f\left( \gamma(t) \right)\perp\dot\gamma(t)&&\forall t\in I
  \end{align*}
\end{Sat}
\begin{Bew}
  $f\circ \gamma=c$ konstant
  \[\underbrace{\Part{\left( f\circ \gamma \right)}{t}}_{=0}=\left\langle \grad f,\dot\gamma \right\rangle \]
  Der Gradient steht senkrecht auf den Höhenlinien und zeigt in die Richtung des stärksten Anstiegs.
\end{Bew}

\subsubsection{Mittelwertsatz}
\begin{Sat}{Mittelwertsatz}
  Sei $f:\underbrace{U}_{\subset\mb{R}}\to \mb{R}$ differenzierbar auf $U$. Seien $a,b\in U$, die durch eine Strecke verbindbar sind. Dann $\exists \xi\in [a;b]$
  \[f(b)-f(a)=\md f_\xi(b-a)\]
\end{Sat}
\begin{Bew}
  Sei
  \begin{align*}
    \gamma:[0;1]&\to U\\
    t&\mapsto a+t(b-a)
  \end{align*}
  \begin{gather*}
    \gamma\left( [0;1] \right)=[a;b]\\
    \dot\gamma(t)=b-a\ \forall t
  \end{gather*}
  \[F:f\circ \gamma:[0;1]\to\mb{R}\]
  Kettenregel $\implies$ $F$ differenzierbar $\xi:=\gamma(\tau)$
  \begin{gather*}
  \xRightarrow{\text{MWS auf } [0;1]}\exists \tau\in[0;1]:F(1)-F(0)=\dot F(\tau)\ (1-0)\\
  F(1)-F(0)=f\left( \gamma(1) \right)-f\left( \gamma(0) \right)=f(b)-f(a)\\
  \dot F(\tau) \stackrel{\text{KR}}{=} \md f_{\gamma(\tau)} \dot\gamma (\tau)=\md f_{\gamma(\tau) } (b-a)
  \end{gather*}
\end{Bew}
\begin{Kor}
  Sei $U$ zusammenhängend und offen. Sei $f:\underbrace{U}_{\subset\mb{R}}\to \mb{C}$ differenzierbar auf $U$. Dann
  \[\md f=0\ \text{überall}\iff f\ \text{konstant}\]
\end{Kor}
\begin{Bew}
  $\La$ trival\\
  $\Ra$
  \subparagraph{Fall 1} $f$ reellwertig
  \[U\ \text{zusammenhängend}\implies \forall a,b\in U\ \exists\ \text{Streckenzug, der sie verbindet}\]
  \begin{align*}
    f(b)-f(a)=f(a_1)-f(a)+f(a_2)-f(a_1)+f(a_3)-f(a_2)+\cdots\\
    f(a_{i+1}-f(a_i)=\md f_{\xi_i}(a_{i+1}-a_i)=0 && \xi_i\in[a_i,a_{i+1}]\\
    \implies f(b)-f(a) &&\forall a,b\in U
  \end{align*}
  \subparagraph{Fall 2} $f:U\to \mb{C}$
  \[\Re f, \Im f:U\to\mb{R}\]
  differenzierbar
  \[\md f=0\implies \md \Re f=0,\md \Im f=0\]
\end{Bew}
\subsubsection{Schrankensatz}
\begin{Sat}{Schrankensatz}
  Sei $f:\underbrace{U}_{\subset\mb{R}}\to \mb{C}$ differenzierbar auf $U$. Sei $K\subset U$ kompakt und konvex. Dann ist $f|_K$ Lipschitz-stetig
  \[\Abs{f(y)-f(x)}\leq L\Norm{y-x}_\infty\]
  \[L=\Norm{f'}_K:=\max_{\xi\in K}\Norm{f'(\xi)}_1\]
  \[\Norm{f'(\xi)}_1=\Abs{\partial_1f(\xi)}+\Abs{\partial_2f(\xi)}+\cdots+\Abs{\partial_n f(\xi)}\]
\end{Sat}
\begin{Bew}
  $K$ konvex $\implies$ $\exists$ Strecke, die $x$ und $y$ verbindet
  \begin{align*}
    \gamma:[0;1]&\to K\\
    t&\mapsto x+t(y-x)
  \end{align*}
  Sei $F=f\circ \gamma$. Kettenregel $\implies$ $F$ ist stetig differenzierbar.
  \begin{gather*}
    \xRightarrow{\text{Schranke auf $[0;1]$}} \Abs{f(y)-f(x)}=\Abs{F(1)-F(0)}\leq \Norm{\cdot F}\\
    \Norm{\dot F}=\sup_{t\in[0;1]}\Abs{\dot F(t)}\\
    \text{Kettenregel:}\ \Abs{\dot F(t)}=\Abs{\sum_i\partial_if\left( \gamma(t) \right)(y_i-x_i)}\leq \sum_i\Abs{\partial_if\left( \gamma(t) \right)}\Abs{y_i-x_i}\\
    \Norm{\dot F}\leq \underbrace{\Norm{f'}_K}_{<\infty,\ \text{da $K$ kompakt}}\Norm{y-x}_\infty
  \end{gather*}
\end{Bew}
\begin{Sat}{Integraldarstellung des Funktionzuwachses}
  Sei $f:U\to\mb{C}$ stetig differenzierbar und sei $\gamma:[\alpha;\beta]\to U$ stetig differenzierbar Kurve. $a:=\gamma(\alpha)$, $b:=\gamma(\beta)$. Dann
  \[f(b)-f(a)=\int_\alpha^\beta\md f_{\gamma\left( t \right)}\dot \gamma(t)\md t=\sum_i\int^\beta_\alpha\partial_if\left( \gamma(t) \right)\dot\gamma_i(t)\md t\]
  Wenn ein Skalarprodukt vorhanden ist
  \[=\int^\beta_\alpha \left\langle \grad f\left( \gamma(t) \right), \gamma(t) \right\rangle \md t\]
\end{Sat}
\begin{Bew}
  Sei $F=f\circ \gamma$
  \begin{gather*}
    f(b)-f(a)=F(\beta)-F(\alpha)=\int^\beta_\alpha\dot F(t)\md t
  \end{gather*}
  + Kettenregel
\end{Bew}
\begin{Kor}
  Sei $f:\underbrace{U}_{\subset\mb{R}, \text{offen}}\to\mb{C}$ stetig differenzierbar. Sei $K_r(a)\subset U$, $a<U$, $r>0$. Dann gibt es $q_1,\cdots,q_n:K_r(a)\to\mb{C}$ stetige Funktionen, s.d.
  \[\forall x\in K_r(0)\ f(x)-f(a)=\sum^n_{i=1}q_i(x)(x_i)(x_a-a_i)\]
  und
  \[q_i(a)=\partial_if(a),\ \forall i\]
\end{Kor}
\begin{Bew}
  \begin{gather*}
    \gamma(t)=a+t(x-a),\ t\in[0;1]\\
    \dot\gamma(t)=x-a\ \forall t\\
    f(x)-f(a)=\int^1_0\sum_i\partial_if\left( \gamma(t) \right)(x-a)\md t =\sum^n_{i=1}\left( \int\partial_if\left( \gamma(t) \right)\md t \right)(x_i-a)\\
    q_i(x):=\int^1_0\partial_if\left( a+t(x-a) \right)\md t
  \end{gather*}
  $\partial_if$ stetig, $[0;1]$ kompakt $\implies$ $q_i$ stetig
  \begin{gather*}
    \partial_if(a)=\Limo{t}\frac{f(a+te_j-f(a)}{t}\\
    x=a+te_j
    x_i=\begin{cases}
      a_i&i\neq j\\a_i+t&i=j
    \end{cases}\\
    x_i-a_i =\begin{cases}
      0&i\neq j\\t&i=j
    \end{cases}\\
    f\left( a+te_j \right)-f(a)=\phi_j(a+te_j)t\\
    \partial_jf(a)=\Limi{t}\phi_j(a+te_j)\stackrel{q\ \text{stetig}}{=}q_j
  \end{gather*}
\end{Bew}
\section{Integrale von Differentialformen und Vektorfeldern (Kap 5.2)}
\begin{Def}{1-Differentialform}
  Sei $U\subset\mb{R}^{n*}$ offen. Eine (stetige) Abbildung $U\to\mb{R}^{n*}$ heisst (stetige) 1-Differentialform.
\end{Def}
\begin{Def}{Vektorfeld}
  Sei $U\subset\mb{R}^{n*}$ offen. Eine (stetige) Abbildung $U\to\mb{R}^n$ heisst Vektorfeld
\end{Def}
\begin{Bsp}
  $f$ stetig differenzierbar $U\to\mb{R}$
  \begin{align*}
    \md f:U\to\mb{R}^{n*}&&\text{stetige Differentialform}\\
    \grad f:U\to\mb{R}^{n}&&\text{stetige Differentialform}
  \end{align*}
\end{Bsp}
\begin{Not}
  Sei
  \begin{align*}
    \omega:U&\to\mb{R}^{n*}\\
    x&\mapsto(\omega_1(x),\cdots,\omega_n(x)
  \end{align*}
  Man schreibt
  \[\omega = \sum^n_{n=1}\omega_i\md x_i\]
  Idee: $\left\{ \md x_i \right\}$ bezeichnet die Basis von $\mb{R}^{n*}$ Dualbasis zu $\left\{ x_1,\cdots,x_n \right\}$
  \[\md f=\sum_i\partial_if\md x_i\]
  \begin{align*}
    X:U&\to\mb{R}^n\\
    x&\mapsto\Mx{x_1(x)\\\vdots\\x_n(x)}= \vec x(x)\\
    \left[ x(x)=\sum^n_{i=1}x_i(x)\Part{}{x_i} \right]
  \end{align*}
\end{Not}
\begin{Def}
  Sei $\gamma:[a;b]\to U$ stetig differenzierbar. Sei $\omega$ eine stetige Differentialform auf $U$:
  \[\int_\gamma\omega:=\int_a^b\sum_{i=1}^n\omega_i\left( \gamma(t) \right)\dot\gamma_i(t)\md t\]
\end{Def}
\begin{Bsp}
  \[\int_\gamma\md f=\int_a^b\sum_i\partial_if\gamma_i\md t\]
\end{Bsp}
\begin{Not}
  \[\md x_i=\Diff{x_i}{t}\md t=\dot\gamma_i\md t\]
\end{Not}
\begin{Def}
  Sei $X$ eine stetiges Vektorfeld auf $U$
  \[\int_\gamma \vec x\vec{\md x}=\int_\gamma\left\langle x,\md x \right\rangle :=\int\sum_{i=1}^nX_i(t)\dot\gamma_i(t)\md t\]
\end{Def}
\begin{Bsp}
  \[\int_\gamma\left\langle \grad f,\md x \right\rangle \]
  (``Werk'')
\end{Bsp}
\begin{Bem}{Integraldarstellung}
  $f$ stetig differenzierbar
  \[f(b)-f(a)=\int_\gamma\md f=\int_\gamma\left\langle \grad f,\md x \right\rangle\]
\end{Bem}
\begin{Bsp}
  \begin{align*}
    \omega:\mb{R}^2&\to\mb{R}^{2*}\\
    \Mx{x\\y}&\mapsto\left( -y,x \right)
  \end{align*}
  \begin{gather*}
    \omega=-y\md x+x\md y\\
    \int_\gamma\omega=\int_a^b\left( x(t)\dot y(t)-y(t)\dot x(t) \right)\md t= F(\gamma)
  \end{gather*}
  Sektorfläche
\end{Bsp}
\begin{Lem}
  Sei $\beta:I\to J$ eine $\mathcal{C}^1$-Parametermetrisierung. Dann
  \begin{gather*}
    \int_{\gamma\circ\beta}\omega = \pm I\int_\gamma\omega\\
    \int_{\gamma\circ\beta}\left\langle x,\md x \right\rangle = \pm I\int_\gamma\left\langle x,\md x \right\rangle \\
    +:\beta\ \text{orientierungstreu}\\
    -:\beta\ \text{orientierungsumkehrend}
  \end{gather*}
\end{Lem}
\begin{Def}{stückweise stetig differenzierbar}
  Sei $\gamma_i:[a_i;b_i]\to U$ stetig differenzierbar mit $a_{i+1}=b_i$, $i=1,\cdots,r$ Sei $\gamma:[a_1,b_\gamma]\to U$ Vereinigung d.h.
  \[\gamma(t)=\gamma_i(t)\ \text{falls}\ t\in\left[ a_i;b_i \right]\]
  Dann heisst $\gamma$ stückweise stetig differenzierbar
  \[\int_\gamma:=\sum_{j=1}^r \int_{\gamma_j}\]
\end{Def}
\begin{Bem}
  $\gamma$ stetig differenzierbar
  \[\gamma=\gamma_1\cup \gamma_2\]
  \[\int_\gamma=\int_{\gamma_1}+\int_{\gamma_2}\]
\end{Bem}
\section{Höhere Ableitungen}
\begin{Def}
  Sei $f:U\to\mb{C}$ differenzierbar in der Richtung $e_i$
  \[\partial_if:U\to\mb{C}\]
  Ist $\partial_if$ in der Richtung $e_j$ differenzierbar, so schreib wir
  \[\partial_j\partial_if:=\partial_j(\partial_if)\]
  \[\partial_j\partial_if(x)=\Limo{t}\frac{\partial_if(x+te_j)-\partial_if(x)}{t}=\Limo{r}\Limo{s}\frac{f(x+te_j+se_i)-f(x+te_j)-f(x+se_i)+f(x)}{ts}\]
  Im Allgemeinen
  \[\partial_j\partial_if\neq \partial_i\partial_jf\]
\end{Def}
\begin{Bsp}
  \begin{gather*}
    f(x,y):=\begin{cases}
      \frac{x^3y}{x^2+y^2}&\left( x,y \right)\neq (0,0)\\
      0&(x,y)=0
    \end{cases}\\
    \partial_x\partial_yf(0,0)=0\\
    \partial_y\partial_xf(0,0)=1\\
  \end{gather*}
\end{Bsp}
\subsection{Berechnen}
\begin{Bsp}
  \begin{gather*}
    f(x,y)=\sin(x^2y)\\
    \partial_xf=2xy\cos(x^2y)\\
    \partial_yf=x^2\cos x^2y\\
    \partial^2_xf:=\partial_x\partial_xf=2y\cos(x^2y)-4x^2y^2\sin(x^2y)\\
    \partial_y\partial_xf=\partial_y\left( 2xy\cos(x^2) \right)=2x\cos x^2y-2x^3y\sin(x^2y)\\
    \partial_x\partial_yf=\partial_x\left( x^2\cos x^2y \right)=2x\cos x^2y-2x^3y\sin x^2y
  \end{gather*}
  In diesem Beispiel
  \[\partial_x\partial_yf=\partial_y\partial_xf\]
\end{Bsp}
\subsection{Satz von Schwarz}
\begin{Sat}
  Sei $f:\underbrace{U}_{\ni a}\to\mb{C}$
  \begin{enumerate}
    \item Es gibt eine Umgebung von $a$, wo $\partial_if,\partial_if$ und $\partial_j\partial_if$ existieren.
    \item $\partial_i\partial_jf$ ist stetig im Punkt $a$. Dann existiert $\partial_i\partial_jf(a)$ und
      \[\partial_i\partial_jf(a)=\partial_j\partial_if(a)\]
  \end{enumerate}
\end{Sat}
\begin{Lem}
  Sei $Q:=(a;a+b)\times (b;b+b+k)$, $n,k>0$ ein Rechteck. Sei $\phi:Q\to\mb{R}$
  \[D_Q\phi:=\phi(a+h,b+k)-\phi(a,b+k)-\phi(a+b,b)-\phi(a,b)\]
  Besitzt $\phi$ auf $Q$ die Ableitungen $\partial_1\phi$ und $\partial_2\partial_1\phi$. Dann
  \[\exists(\xi,\eta)\in Q\ \text{s.d.}\ D_Q\phi=hk\partial_2\partial_1\phi(\xi,\eta)\]
\end{Lem}
\begin{Bew}
  Interierter Mittelwertsatz
\end{Bew}

\begin{Bew}
  Fall 1:$f$ reellwertig\\
  Sei 
  \[\phi(x,y):=f(a+xe_i+ye_j)\]
  \begin{enumerate}
    \item $\implies$ $\exists$ Umgebung $V$ von $(0,0)\in\mb{R}^2$, wo folgende partielle Ableitungen existieren:
      \begin{align*}
        \partial_x\phi&=\partial_if\\
        \partial_y\phi&=\partial_jf\\
        \partial_x\partial_y&=\partial_j\partial_if
      \end{align*}
    \item $\implies$ $\partial_y\partial_x\phi$ ist stetig in $(0,0)$
  \end{enumerate}
  Zu zeigen:
  \[\underbrace{\partial_x\partial_y\phi(0)}_{=\partial_i\partial_jf(a)}=\underbrace{\partial_y\partial_x\phi(0)}_{=\partial_j\partial_if(a)}\]
  $\forall \varepsilon >0$ $\exists$ Umgebung $V'$ von $(0,0)$ mit $V'\subset V$
  \begin{align*}
    \Abs{\partial_y\partial_x\phi(x,y)-\partial_y\partial_x\phi(0,0)}<\varepsilon &&\forall (x,y)\in V'
  \end{align*}
  Sei $Q:=\left( 0;h \right)\times\left( 0;k \right)$ wobei $h,k>0$ so dass $Q\subset V$
  \begin{Lem}
    $\exists\ (\xi,\eta)\in Q$
    \[\frac{D_Q\phi}{hk}=\partial_y\partial_x\phi(\xi,\eta)\]
    (Mittelwertsatz)
  \end{Lem}
  \begin{gather*}
    Q\subset V'\implies\Abs{\frac{D_Q\phi}{hk}-\partial_i\partial_x\phi(0,0)}<\varepsilon
  \end{gather*}
  \begin{gather*}
    \frac{D_Q\phi}{hk}:=\frac{\phi(h,k)-\phi(h,0)-\phi(0,k)+\phi(0,0)}{hk}=\frac{1}{h}\left( \frac{ \phi(h,k)+\phi(0,0)}{k}-\frac{\phi(h,0)-\phi(0,k)}{k}\right)\\
    \Limo{k}\frac{D_Q\phi}{hk}=\frac{\partial_y\phi(h,0)-\partial_y\phi(0,0)}{h}\\
    \Abs{\ }<\varepsilon\implies\Limo{k}\Abs{\ }=\Abs{\Limo{k}\ }\leq \varepsilon\\
    \Abs{\frac{\partial_y\phi(h,0)-\partial_y\phi(0,0)}{h}-\partial_y\partial_x\phi(0,0)}\leq \varepsilon
  \end{gather*}
  d.h. $\forall\varepsilon>0$ $\exists \delta>0$ s.d. $\forall h:\Abs{h}<\partial$ gilt die Ungleichung. D.h.
  \[\Limo{h}\frac{\partial_y\phi(h,0)-\partial_y\phi(0,0)}{h}=\partial_y\partial_x\phi(0,0)=\partial_x\partial_y\phi(0,0)\]
  Fall2: $f$ komplexwertig\\
  Re $f$, Im $f$ erfüllen die Voraussetzung von Fall 1.
\end{Bew}
\begin{Def}{$k$-mal stetig differenzierbar}
  Sei $f:\underbrace{U}_{\subset\mb{R}}\to \mb{C}$  $f$ heisst $k$-mal stetig differenzierbar ($k\geq 1$) wenn \underline{alle} partiellen Ableitungen $k$-ter Ordnung
  \[\left( \partial_{i_1}\partial_{i_2}\cdots\partial_{i_k}f,\forall \left( i_1,\cdots,i_k \right)\in\left\{ 1,\cdots,n \right\}^k \right)\]
  auf $U$ existieren und stetig sind.
\end{Def}
\begin{Def}
  \[\mathcal{C}^k(U)=\left\{ \text{$k$-mal stetig differenzierbare Funktionen auf $U$} \right\}\]
\end{Def}
\begin{Bem}
  \begin{itemize}
    \item $\mathcal{C}$ Vektorraum
    \item $\mathcal{C}^{k+l}\in\mathcal{C}$ $\forall l\geq 0$
  \end{itemize}
\end{Bem}
\begin{Def}
  \[\mathcal{C}^\infty(U)=\cap_{k=1}^\infty\mathcal{C}^k(U)\]
  beliebig of stetig differenzierbare Funktionen auf $U$ (auch ein Vektorraum)
\end{Def}
\begin{Not}
  Genauere Bezeichnungen:
  \begin{itemize}
    \item $\mathcal{C}^k(U,\mb{R})$ reellwertig
    \item $\mathcal{C}^k(U,\mb{C})$ komplexwertig
  \end{itemize}
\end{Not}
\begin{Def}{Zweite Ableitung}
  Sei $f\in\mathcal{C}^2(U)$, $U\subset\mb{R}^n$ Seien $u,v\in\mb{R}^n$, $a\in U$
  \[\md^2f_a(u,v):=\partial_u\partial_vf(a)\]
\end{Def}
\begin{Bem}
  \begin{align*}
    \partial_vf(a)&=\sum^n_{i=1}\partial_if(a)v_i\\
    \partial_u(\partial_vf(a))&=\sum^n_{i=1}\partial_u(\partial_vf(a))v_i\\
    =\md^{(2)}f_a(u,v)&=\sum_{j=1}^n\sum^n_{i=1}\partial_j\partial_if(a)v_iu_j
  \end{align*}
  \[\md^{(2)}f_a:\mb{R}^n\times\mb{R}^n\to\mb{C}\]
  ist bilinear.
\end{Bem}
\begin{Bem}
  Schwarz:
  \begin{align*}
    f\in\mathcal{C}^2\implies\partial_i\partial_jf(a)=\partial_j\partial_if(a)&& \forall\\
    \implies \md^{(2)}f_a\ \text{symmetrisch}
  \end{align*}
  \begin{align*}
    \md^{(2)}f_a(u,v)=\md^{(2)}=\md^{(2)}f_a(v,u)&&\forall u,v
  \end{align*}
\end{Bem}
\begin{Bem}
  Die darstellende Matrix von $\md^{(2)}f_a$ ist
  \[f''(a):=\left( \partial_i\partial_jf(a) \right)\]
  2. Ableitung von $f$ im Punkt $a$. Andere Bezeichnung:
  \[H_f(a):=f''(a)\]
  Hesse-Matrix von $f$ im Punkt $a$.
\end{Bem}
\begin{Bem}
  Sei $f\in\mathcal{C}^2(U)$, $a\in U$
  \[H_f(a):=\left( \partial_i\partial_jf(a)j \right)\]
  \begin{itemize}
    \item $H_f(a)$ symmetrische Matrix
    \item \[\md^{(2)}f(a)\left( u,v \right)=u^tH_f(a)v=\sum_{i,j=1}^nf_{ij}''(a)u_iv_j\]
  \end{itemize}
\end{Bem}
\begin{Bem}
  Die Spur der Hesse-Matrix von $f$:
  \[\Delta f(a):=\Spur H_f(a)=\sum_{i=1}^n\partial^2_if(a)\]
  \[\Delta:=\sum^n_{i=1}\partial^2_i\]
  $\Delta$ Laplace-Operator
\end{Bem}
\begin{Lem}
  Für jede Orthonormalbasis $\left( v_1,\cdots,v_n \right)$ von $\mb{R}^n$ gilt
  \[\Delta f=\sum^n_{i=1}\partial_{v_i}^2f\]
\end{Lem}
\begin{Def}{Differential $p$-ter Ordnung}
  Sei $f\subset\mathcal{C}^p(U)$, $U\subset \mb{R}^n$. Sei $a\in U$. Seien $v^1,v^2,\cdots,v^p\in\mb{R}^n$
  \[\md^{(p)}f_a(v^1,\cdots,b^p):=\partial_{v^1}\partial_{v^2}\cdots\partial_{v^p}f(a)\]
  \[\md^{(p)}f_a(v^1,\cdots,b^p):=\sum^n_{i_1=1}\cdots\sum^n_{i_p=1}\partial_{i^1}\partial_{i^2}\cdots\partial_{i^p}f(a)v_{i_1}v_{i_2}^2\cdots v_{i_p}^p\]
  $f\in \phi^p$ und Schwarz $\implies$
  \[\partial_{i_1}\cdots\partial_{ip}f(a)=\partial_{i_{\sigma(1)}}\cdots\partial_{i_{\sigma(p)}}f(a)\ \forall \sigma:\left\{ 1,\cdots,p \right\}\to\left\{ 1,\cdots,p \right\}\]
  \[\md^{(p)}f_a(v^1,\cdots,v^p)=\md^{(p)}f_a(v^{\sigma(1)},\cdots,v^{\sigma(p)})\]
\end{Def}
\subsection{Taylorapproximation}
\begin{Bem}
  Sei $f\in\mathcal{C}^{p+1}(U,\mb{R})$, $U\subset\mb{R}^n$. Seien $a,x\in U$ s.d.
  \[[a;x]\subset U:=\left\{ a+t(x-a),t\in[0;1] \right\}\]
  \begin{align*}
    F:\left[ 0;1 \right]&\to\mb{R}\\
    t&\mapsto f(a+th) && h:=x-a
  \end{align*}
  \begin{itemize}
    \item 
      $F$ $p+1$-mal stetig differenzierbar (Kettenregel)
      \[F'(t)=\sum_{i=1}^n\partial_if(a+th)h_i=\md f_{a_{i_h}}\]
      \begin{align*}
        F''(t)=\sum_{i,j=1}^n\partial_j\partial_if(a+h)h_ih_j&&=\md^{(2)}f_a(h,h)\\
        \cdots\\
        F^{(k)}(t)=\sum^n_{i_1,\cdots,i_k=1}\partial_{i_1}\cdots\partial_{i_k}f(a+h)h_{i_1}\cdots h_{i_k} && =\md^{(k)}f_a(h,\cdots,h)
      \end{align*}
      Abkürzung: $V\in\mb{R}^n$
      \[\md^{(k)}f(a)v^k:=\md^{(k)}f(a)(v,\cdots,v)\]
      \begin{enumerate}
        \item $F$ reellwertig und $p+1$ stetig differenzierbar
        \item $F^{(k)}(t)=\md^{(k)}f(a+h)h^k$
      \end{enumerate}
      1) $\implies$
      \begin{gather*}
        F(1)=T_pF(1;0)+R_{p+1}\\
        T_pF(1;0)=\sum^p_{k=0}\frac{1}{k!}F^{(k)}(0)1^k\\
        R_{p+1}=\frac{1}{\left( p+1 \right)!}F^{(p+1)}(\tau), \tau\in [0;1]
      \end{gather*}
    \item 
      \begin{gather*}
        F(1)=f(a+h)=f(x)\\
        T_pF(1;0)=\sum^p_{k=0}\frac{1}{k!}\md^{(k)}f_ah^k=:T_pf(x,y)
      \end{gather*}
      Taylorapproximation der Ordnung $p$ von $f$ im Punkt $a$
    \item
      $x=a+\tau h$
      \[\exists\xi\in[a;x]:T_{p+1}=\frac{1}{(p+1)!}\md^{(p+1)}f(\xi)h^{p+1}=:R(x;a;\xi)\]
      Rest
  \end{itemize}
\end{Bem}
\begin{Def}{Taylorsatz mit Rest}
  Sei $f\in\mathcal{C}^{p+1}(U,\mb{R})$. Seien $a,x\in U$ mit $[a;x]\subset U$. Dann $\exists\xi\in [a;x]:$
  \[f(x)=T_pf(x;a)+R_{p+1}\left( x;a;\xi \right)\]
  wobei
  \[T_pf(x;a)=\sum^p_{k=0}\frac{1}{k!}\md^{(k)}f(a)(x-a)^k\]
  \[R_{p+1}f(x;a;\xi)=\frac{1}{\left( p+1 \right)}!\md^{(p+1)}f(a)(x-a)^{\phi+1}\]
\end{Def}
\begin{Kor}{Qualitative Taylorformel}
  Sei $f\in\mathcal{C}^p(U)$ (möglicherweise komplexwewrtig). Dann $\forall a\in U$
  \[f(x)=T_pf(x,a)+0\left( \Norm{x-a}^p \right), x\to a\]
  d.h.
  \[\lim_{x\to a}\frac{f(x)-T_pf(x;a)}{\Norm{x-a}^p}=0\]
\end{Kor}
\begin{Def}{Taylorreihe von $f$ im Punkt $a$}
  Sei $f\in\mathcal{C}^\infty(U)$, $a\in U$
  \[Tf(x;a):=\sum^\infty_{k=0}\frac{1}{k!}\md^{(k)}f(a)(x-a)^k\]
\end{Def}
\begin{Def}{reell-analytisch}
  Besitzt jeder Punkt von $U$ eine Umgebung, wo die Taylorreihe von $f$ gegen $f$ konvergiert, so heisst $f$ reell-analytisch.  
\end{Def}
\begin{Lem}
  Sei $f\in\mathcal{C}^\infty(U)$, $a\in U$, $r>0$ $K_r(a)\subset U$ $\forall k$ sei $P_k$ homogenes Polynom von Grad $k$ s.d.
  \begin{align*}
    f(x)=\sum\infty_{k=0}P_k(x-a)&&\forall x\in K_r(a)
  \end{align*}
  dann ist
  \[Tf(x;a)=\sum P_k(x-a)\]
\end{Lem}
\subsubsection{Geometrische Auffassung}
\begin{Def}{Tangentialhyperebene}
  Sei $f:\underbrace{U}_{\subset\mb{R}}\to \mb{C}$ und $f\in\mathcal{C}^1$. Der Graph des Taylorpolynomes 1. Ordnung
  \[\left\{ (x,z)\in\mb{R}^{n+1}:z=T_1f(x,a):=f(a)+\md f_a(x) \right\}\]
  heisst Tangentialhyperebene von $f$ im Punkt $a$.
\end{Def}
\begin{Def}{Schmiegquadrik}
  Sei $f\in\mathcal{C}^2(U)$. Der Graph des Taylorpolynomes 2. Ordnung
  \[\left\{ (x,z(\in\mb{R}^{n+1}:z=T_2f(x;a) \right\}\]
  heisst Schmiegquadrik an den Graphen von $f$ in $\left( a,f(a) \right)$
  \[z=f(a)+f'(a)(x-a)+\frac{1}{2}(x-a)^tf''(a)(x-a)\]
  \begin{gather*}
    \tilde x:=x-a\\
    \tilde z:=z-f(a)-f'(a)(x-a)\\
    \tilde z=frac{1}{2}\tilde x^tf''(a)\tilde x
  \end{gather*}
  Graph eine quadratische Funktion =: Quadrik
\end{Def}
\begin{Bsp}
  \begin{table}
    \centering
    \begin{tabular}{c|c|c}
      Funktion&Name&(0,0)\\
      \hline
      $z=x^2+y^2$& elliptisches Paraboloid&Minimum\\
      $z=-(x^2+y^2)$&&Maximum\\
      \hline
      $z=x^2-y^2$&hyperbolisches Paraboloid&Sattelpunkt\\
      \hline
      $z=x^2$&parabolischer Zylinder&Minimum
    \end{tabular}
    \caption{$n=2$}
  \end{table}
\end{Bsp}

\subsection{Minima und Maxima}
\begin{Bem}
  Sei $A$ $n\times n$ symmetrische Matrix. Sei 
  \begin{align*}
    Q(x):=\frac{1}{2} x^tAx,x\in\mb{R}^n
  \end{align*}
  \begin{table}[htb]
    \centering
    \begin{tabular}{l|c|c|c|l}
      Name&Bedingung&Bezeichnung&Eigenwerte&(0,0)\\
      \hline
      positiv definit&$Q(x)>0$ $\forall x\neq 0$&$Q>0$& alle $>0$&isoliertes Minimum\\
      negativ definit&$Q(x)<0$ $\forall x\neq 0$&$Q<0$& alle $<0$&isoliertes Maximum\\
      positiv semidefinit&$Q(x)\geq0$ $\forall x$&$Q\geq0$& alle $\geq0$&Minimum\\
      negativ semidefinit&$Q(x)\leq0$ $\forall x$&$Q\leq0$& alle $\leq0$&Maximum\\
      indefinit&$\exists x:Q(x)>0$ $\exists <:Q(x)<0$& $Q\gtrless 0$&$\exists \lambda>0$ $\exists \mu<0$&kein Extremum
    \end{tabular}
  \end{table}
  $Q$ definit, wenn $Q>0$ oder $Q<0$\\
  $n=2$
  \begin{align*}
    Q>0 &\iff \det A>0\ \text{und}\ a>0\\
    Q<0 &\iff \det A<0\ \text{und}\ a>0\\
    Q\geq0 \text{oder}\ Q\leq 0&\iff \det A\geq0\\
    Q\gtrless 0&\iff \det A<0\\
  \end{align*}
\end{Bem}
\begin{Bew}
  \begin{gather*}
    A\sim\Mx{\lambda_1 0&\\0&\lambda_2}\\
    \det A=\lambda_1\lambda_2\\
    \Spur A:= a+c = \lambda_1\lambda_2\\
    \det A=ac-b^2\\
    \det A >0\implies ac > 0
  \end{gather*}
\end{Bew}
\begin{Def}
  Sei $f\in\mathcal{C}^2(U,\mb{R})$, sei $A=f''(a)$ und $Q(x)=\frac{1}{2}\md^{(2)}f(a)x^2$
  \begin{table}[htb]
    \centering
    \begin{tabular}{l|l|c}
      $f$ heisst& wenn $\md^2 f_a$\\
      \hline
      elliptisch&definit&$\md^2f>0$ oder $\md^2<0$\\
      hyperbolisch&nicht indefinit&$\md^2f\gtrless 0$\\
      flach&$\md^2f=0$
    \end{tabular}
  \end{table}
\end{Def}
\begin{Def}
  Sei $f:\underbrace{U}_{\subset\mb{R}}\to \mb{R}$. $f$ hat in $a\in X$ ein
  \begin{description}
    \item[lokales Maximum] wenn es eine Umgebung $V$ von $a$ gibt, so dass $f(x)\leq f(a)$ $\forall x\in V$
    \item[lokales Minimum] wenn es eine Umgebung $V$ von $a$ gibt, so dass $f(x)\geq f(a)$ $\forall x\in V$
    \item[isoliertes lokales Maximum] wenn es eine Umgebung $V$ von $a$ gibt, so dass $f(x)<f(a)$ $\forall x\in V\setminus \left\{ a \right\}$
    \item[isoliertes lokales Minimum] wenn es eine Umgebung $V$ von $a$ gibt, so dass $f(x)>f(a)$ $\forall x\in V\setminus\left\{ a \right\}$
  \end{description}
\end{Def}
\begin{Sat}{Notwendiges Kriterium}
  Sei $f:\underbrace{U}_{\subset\mb{R}}\to \mb{R}$. Ist $a\in U$ ein lokales Extremum von $f$ und ist $f$ partiell differenzierbar in $a$, so gilt
  \[\partial_1f(a)=\partial_2f(a)=\cdots=\partial_nf(a)=0\]
\end{Sat}
\begin{Bem}
  Ist $f$ differenzierbar in $a$, dann gilt $\md f_a=0$
\end{Bem}
\begin{Bew}
  $k\in \left\{ 1,\cdots,n \right\}$. Sei $F(t)=f(a+te_k)$. $a$ lokales Extremum von $f$ $\implies$ $t=0$ lokales Extremum von $F$ $f$ partiell differenzierbar in $a$ $\implies$ $F$ differenzierbnar in 0 $\implies$ $F'(0)=0$\\
  Kettenregel $F'(0)=\partial_kf(a)$
\end{Bew}
\begin{Def}{stationäre Stelle}
  Sei $f$ differenzierbar in $a$ und es gelte $\md f_a=$. Dann heisst $f$ stationär in $a$ und $a$ heisst stationäre Stelle in $f$ oder kritischer Punkt.
\end{Def}
\begin{Sat}{hinreichendes Kriterium}
  Sei $f\in\mathcal{C}^2(U,\mb{R})$, sei $\md f_a=0$ Dann
  \begin{enumerate}
    \item $\md^2f_a>0$ $\implies$ $f$ hat in $a$ ein isoliertes lokales Maximum
    \item $\md^2f_a<0$ $\implies$ $f$ hat in $a$ ein isoliertes lokales Minimum
    \item $\md^2f_a\gtrless0$ $\implies$ $f$ hat in $a$ kein Extremum
  \end{enumerate}
\end{Sat}
\begin{Bsp}
  $f(x,y)=x^4+y^4$ $(0,0)$ ist ein isoliertes lokales Minimum
  \begin{gather*}
    \md f_{(0,0)}=0\\
    \md^2 f_{(0,0)}=0
  \end{gather*}
\end{Bsp}
\begin{Bew}
  \begin{gather*}
    \md f_a=0\implies T_2f(x;a)=f(a)+\frac{1}{2}\md^2f(a)(x-a)^2\\
    f(a+h)=f(a)+\frac{1}{2}\md^2f(a)h^2+R_2(h)\\
    \frac{R_2(h)}{\Norm{h}^2}\to 0
  \end{gather*}
  \begin{enumerate}
    \item $\md ^2f_a>0$ Sei
      \begin{align*}
        \phi:S^{n-1}&\to\mb{R}\\
        h&\mapsto\md^2f(a)h^2
      \end{align*}
      $S^{n-1}$ kompakt $\implies$ $\phi$ nimmt ein Minimum $m$. $\phi>0\implies m>0$
      \begin{align*}
        \md^2f_a(h)^2=\md^2f_a\left( \Norm{h}\frac{h}{\Norm{h}} \right)^2=\Norm{h}^2\phi\left( \frac{h}{\Norm{h}} \right)\geq \Norm{h}^2m &&\forall h\in\mb{R}^h\setminus\left\{ 0 \right\}
      \end{align*}
      Sei $\delta>0$:
      \begin{enumerate}
        \item $K_\delta(a)\subset U$
        \item $\Abs{R_2(h)}\leq \frac{1}{4}m\Norm{h}^2$ $\forall h:\Norm{h}<\delta$
      \end{enumerate}
      Für $h$: $\Norm{h}<\delta$
      \begin{gather*}
        f(a+h)=f(a)+\underbrace{\frac{1}{2}\md^2f_ah^2}_{\geq \frac{m}{2}\Norm{h}}+R_2(h)\geq f(a)+\frac{m}{4}\Norm{h}^2\\
        \xRightarrow{m>0} f(a+h)> f(a)\ \forall h\in K_\delta(a)\setminus\left\{ 0 \right\}
      \end{gather*}
      $a$ ist ein isoliertes lokales Minimum
    \item $\md^2f_a<0$ Sei $\phi:=-f$. $\md^2g>0$ $\implies$ $g$ heisst in $a$ ein isoliertes lokales Minimum $\implies$ $f$ ist in $a$ in isoliertes lokales Maximum
    \item Sei $\md^2f_a \gtrless 0$. D.h. $\exists \md^2 f_av^2>0$ und $\exists \md^2 f_aw^2<0$ Sei $F_v(t):=f(a+tv)$ und $F_w(t):=f(a+tw)$
      \begin{gather*}
        \dot F_v(0)=\dot F_w(0)=0\\
        F_v''(0)=\md^2f(a)v^2>0
      \end{gather*}
      $\implies$ $F_v$ nimmt in $t=0$ ein Minimum an
      \begin{gather*}
        F_w''(0)=\md^2f(a)w^2<0
      \end{gather*}
      $\implies$ $F_v$ nimmt in $t=0$ ein Maximum an. Deshalb nimmt $f$ in $a$ kein Extremum an.
  \end{enumerate}
\end{Bew}
\begin{Sat}
  Sei $f\in\mathcal{C}^2(U,\mb{R})$, $a\in U$.
  \[a\ \text{lokales Maximum} \implies \md^2f\leq 0\]
  \[a\ \text{lokales Minimum} \implies \md^2f\geq 0\]
\end{Sat}
\begin{Bem}{Methode}
  Sei $f:U\to\mb{R}\left( \mathcal{C}^2 \right)$ und $U$ offen. Wir wollen alle Extrema finden
  \begin{enumerate}
    \item Man findet alle stationären Stellen d.h.
      \[a\in U:\md f_a=0\]
    \item Man studiert $\md^2f_a$ für $a$ stationär.
    \item beten (z.B. Taylor weiterentwickeln)
  \end{enumerate}
  Ist $f$ auf $\bar U$ definiert, so muss man Extrema auf $\bar U$ finden
  \begin{enumerate}
    \item Man sucht nach Extrema auf $f|_{\partial U}$
    \item Man verifiziert, ob solche Extrema eigentliche Extrema von $f$ auf $\bar U$ sind.
  \end{enumerate}
\end{Bem}
\subsection{harmonische Funktionen}
\begin{Def}{harmonische Funktionen}
  Sei $f\in\mathcal{C}^2(U)$ $f$ heisst harmonisch, falls
  \begin{align*}
    \Delta f(x)=0&&\forall x\in U
  \end{align*}
\end{Def}
\begin{Sat}{Schwaches Maximums-Prinzip für harmonische Funktionen} Sei $U\subset\mb{R}^n$ offen und beschränkt und sei $f:\bar U\to\mb{R}^n$ stetig s.d. $f|_U$ harmonisch ist. Dann nimmt $f$ ihr Maximum und ihr Minimum auf $\partial U$ an.
\end{Sat}
\begin{Bew}{Widerspruchsbeweis} Wir nehmen an:
  \begin{itemize}
    \item $f|_U$ harmonisch
    \item $f$ nimmt ihr Maximum nicht auf $\partial U$ an.
  \end{itemize}
  Seien $M:=\max_{x \in U} f(x)<\infty$ und $\mu:=\max_{x\in \partial x} f(x)<\infty$ denn $f$ ist stetig und $U$ beschränkt. Aus der 2. Annahme folgt, dass $\mu <M$
  \begin{gather*}
    \lambda:=\max_{x\in \partial U}\left( x_1^2+x_2^2+\cdots+x_n^2 \right)<\infty
  \end{gather*}
  Sei $\varepsilon>0:\mu+\varepsilon\lambda<M$. Sei $f_\varepsilon(x):=f(x)+\varepsilon\left( x_1^2+\cdots+x_n^2 \right)$.
  \begin{gather*}
    M_\varepsilon:=\max_{x\in \bar U}f_\varepsilon(x),\mu_\varepsilon:=\max_{x\in \partial U}f_\varepsilon (x)\\
    f_\varepsilon(x)>f(x)\ \forall x\\
    M_\varepsilon\geq M\\
    \mu_\varepsilon\leq \mu +\varepsilon\lambda<M\\
    \implies \mu_\varepsilon<M_\varepsilon
  \end{gather*}
  $\implies$ $f_\varepsilon$ hat ein Maximum $M_\varepsilon$ an einem Punkt $a\in U$
  \[\implies \md^2f_\varepsilon\leq 0\]
  \begin{gather*}
    \Delta f_\varepsilon(a)=\Spur f_\varepsilon''(a)\leq 0
  \end{gather*}
  Aber
  \begin{gather*}
    \Delta f_\varepsilon(a)=\underbrace{\Delta f(a)}_{=0}+\underbrace{\varepsilon\Delta \left( x_1^2+\cdots+x_n^2 \right)}_{2n\varepsilon>0}
  \end{gather*}
  $\implies$ $\Delta f_\varepsilon(a)$. Dies ist ein Widerspruch zu $\Delta f_\varepsilon(a)\leq 0$.
\end{Bew}
\subsection{Konvexität von Funktionen}
\begin{Def}
  Sei $F:U\to \mb{R}$, $U\subset\mb{R}^n$ $f$ heisst konvex, wenn $\forall a,b\in U$ $\forall t\in (0;1)$
  \[f\left( \left( 1-t \right)a+tb \right)\leq \left( 1-t \right)f(a)+t f(b)\]
  \begin{itemize}
    \item konvex: $\leq$
    \item konkav: $\geq$
    \item streng konvex $<$
    \item streng konkav $>$
  \end{itemize}
\end{Def}
\begin{Sat}
  Sei $f\in\mathcal{C}^2(U,\mb{R})$, $U$ konvex und offen
  \begin{align*}
    f\ \text{konvex}\iff \md^2f_a\geq 0&&\forall a\in U\\
    \md^2f_a>0\implies f\ \text{streng konvex}&&\forall a\in U
  \end{align*}
\end{Sat}
\begin{Bew}
  \[F(t)=f\left( \left( 1-t \right)a+tb \right)\]
  1-dimensional
\end{Bew}
\subsection{Parameterabhängige Integrale}
\begin{Def}
  Sei $U\subset\mb{R}^n$ und sei $f:U\times [a;b]\to\mb{C}$ s.d. $\forall x\in U:t\mapsto f(x,t)$ stetig auf $[a;b]$ $\implies$ Regelfunktion
  \[F(x):=\int_a^bf(x,t)\md t\]
  $f$ stetig $\implies$ $F$ stetig
\end{Def}
\subsubsection{Differentiationssatz}
\begin{Sat}
  Es gelte zusätzlich
  \begin{itemize}
    \item $\forall t\in [a;b]$ ist $f$ nach $x_i$ differenzierbar
    \item $\partial_if:U\times [a;b]\to\mb{C}$ stetig
  \end{itemize}
  Dann ist $F$ nach $x_i$ differenzierbar und 
  \[\partial_i F(x)=\int_a^b\partial_if(x,t)\md t\]
  d.h.
  \[\partial\int=\int\partial\]
\end{Sat}
\begin{Bem}
  Mit den Lebesgne-Integralen gilt das unter viel schwächeren Bedingungen.
\end{Bem}

\begin{Bew}
  Für $n=1$. Sei $x_0\in U$. Wir wollen
  \begin{gather*}
    \Delta := \frac{F(x)-F(x_0)}{x-x_0}-\int^b_a\partial_xf(x_0,t)\md t
  \end{gather*}
  abschätzen.
  \begin{gather*}
    \Delta = \int^b_a\left(  \frac{f(x,t)-f(x_0,t)}{x-x_0}-\partial_xf(x_0,t) \right)\md t
  \end{gather*}
  aus dem Satz folgt, dass $f$ nach $x$ differenzierbar ist $\forall t$. Daraus folgt aus dem Mittelwertsatz, dass $\forall t:\exists \xi(t)$ zwischen $x$ und $x_0$ s.d.
  \[\frac{(x,t)-f(x_0,t)}{x-x_0}=\partial_xf\left( \xi(t),t \right)\]
  \begin{gather*}
    \Delta = \int^b_a\left( \partial_xf\left( \xi(t),t \right)-\partial_xf(x_0,t) \right)\md t\\
    \psi(x,t):=\partial_xf(x,t)-\partial_xf(x_0,t)\\
    \Delta = \int_a^b\psi(\xi(t),t)\md t
  \end{gather*}
  Sei $\varepsilon>0$
  \begin{gather*}
    W:=\left\{ (x,t)\in U\times \left[ a;b \right]:\Abs{\psi(x,t)}<\frac{\varepsilon}{b-a} \right\}
  \end{gather*}
  Aus der zweiten Bedingung des Satzes folgt, dass $\psi$ stetig ist, $\implies$ $W$ offen
  \begin{gather*}
    \psi(x_0,t)=0\implies \left\{ x_0 \right\}\times \left[ a;b \right]\subset W\  \forall t\\
    \xRightarrow{\text{Tubenlemma}} \exists I\subset_{\text{offen}} U\ \text{mit}\  x_0\in I\\
    \text{s.d.}\ I\times \left[ a;b \right]\subset W\\
    \forall x\in I:\Abs{\psi(x,t)}<\frac{\varepsilon}{b-a}\\
    \Abs{\Delta}\leq\int^b_a\Abs{\psi\left( x(t),t \right)}\md t\leq (b-a)\frac{\varepsilon}{b-a}\\
    \implies\lim_{x\to x_0}\Delta =0
  \end{gather*}
\end{Bew}
\begin{Kor}{Vertauschbarkeitssatz für iterierte Integrale}
  Sei $f:\left[ a;b \right]\times \left[ a;b \right]\to\mb{C}$ stetig. Dann
  \[\int_c^d\left( \int_a^bf(x,t)\md t \right)\md x = \int_a^b\left( \int_c^df(x,t)\md x \right)\md t\]
\end{Kor}
\begin{Bew}
  Sei
  \[F(x):=\int^b_af(x,t)\md t\]
  \[\Phi(\xi):=\int_C^\xi F(x)\md x=\int_C^\xi \left( \int_a^bf(x,t)\md t \right)\md x\]
  Hauptsatz $\implies$
  \[\Phi'_1(\xi)=F(\xi)=\int_a^bf(\xi,t)\]
  Sei
  \[\Phi_2(\xi):=\int_a^bG(\xi,t)\md t=\int^b_a\left( \int_C^\xi f(x,t)\md x \right)\md t\]
\end{Bew}
\begin{Bem}
  $\forall \xi$ ist $G$ stetig bezüglich $t$\\
  Hauptsatz $\partial_xG(\xi,t)=f(\xi,t)$ stetig $\xRightarrow{\text{Diffsatz}}$
  \begin{align*}
    \Phi_2'(\xi)=\int^b_a\partial_xG(\xi,t)\md t=\int_a^bf(\xi,t)\md t\\
    \implies \Phi_2'(\xi)=\Phi_1'(\xi)&&\forall \xi\\
    \Phi_2(C)=0=\Phi(C)\implies \Phi_1(\xi)=\Phi_2(\xi)&&\forall \xi
  \end{align*}
  Insbesondere $\xi=d$
\end{Bem}
\begin{Sat}
  Sei $f:\left[ a_1;b_1 \right]\times \left[ a_2;b_2 \right]\times\cdots\times\left[ a_n;b_n \right]\to\mb{C}$ stetig. Dann
  \[\int_{a_1}^{b_1}\left( \cdots\left( \int_{a_n}^{b_n}f\left( x_1,\cdots,x_n\md x_n \right)\right)\cdots  \right)\md x_1=\int^{b_{\sigma(1)}}_{a_{\sigma(1)}}\left( \cdots \right)\md x_{\sigma(1)}\]
  $\forall$ Permutationen $\sigma$
\end{Sat}
\begin{Bew}
  Durch Induktion
\end{Bew}
\begin{Def}{Mehrfachintegral}
  Das ist das Mehrfachintegral einer stetigen Funktion auf einem kompakten Quader.
  \[\int_Qf\left( x_1,\cdots,x_n \right)\md^n x\]
\end{Def}
\begin{Bem}
  Man kann weitere Klassen von Funktionen auf weiteren Klassen von Bereichen integrieren: Analysis III
\end{Bem}
\begin{Sat}{Satz von Stokes auf einem Rechteck}
  Sei $\omega$ eine stetig differenzierbare 1-Form auf $U\subset\mb{R}^2$. (d.h. $\omega(x_1,x_2)=w_1(x_1,x_2)\md x_1+\omega_2(x_1,x_2)\md x_2$, $\omega_1$, $\omega_2$ stetig differenzierbar auf $U$). Sei $Q:=\left[ a;b \right]\times \left[ c;d \right]\subset U$. Sei $\partial Q$ der Rand von $Q$ als parametrisierte Kurve mit Orientierung im Gegenurzeigersinn. z.B.
  \begin{align*}
    \gamma_1:\left[ a;b \right]&\to\mb{R}\\
    t&\mapsto (t,c)
  \end{align*}
  Dann
  \begin{gather*}
    \int_{\partial Q}\omega\int_Q\left( \partial_1\omega_2-\partial_2\omega_1 \right)\md^2 x
  \end{gather*}
  \[\md \omega := \left( \partial_1\omega_2-\partial_2\omega_1 \right)\md^2 x\]
  \[\int_{\partial Q}\omega = \int_Q\md \omega\]
\end{Sat}
\begin{Bew}
  \begin{gather*}
    \int_Q\partial_1\omega_2\md^2x=\int_c^d\left( \int_a^b \partial_1\omega_2\md x_1 \right)\md x_2\\
    \int_a^b\partial_1\omega_2\left( x_1,x_2 \right)\md x_1=\omega_2\left( x_1\big|^b_a,x_2 \right) = \omega_2(b,x_2)-\omega_2(a,x_2)\\
    \int_Q\partial_1\omega_2=\underbrace{\int_c^d\omega_2(b,x_2)\md x_2}_{=\int_{\gamma_2}\omega}-\underbrace{\int^d_c\omega_2(a,x_2)\md x_2}_{=\int_{\gamma_4}\omega}\\
    =\int_{\gamma_2}\omega+\int_{\gamma_4}\omega
  \end{gather*}
  \begin{gather*}
    \int_Q\partial_2\omega_1\md^2x=\int_a^b\left( \int_c^d \partial_2\omega_1\md x_2 \right)\md x_1\\
    \int_a^b\left( \omega_1(x_1,d)-\omega_1(x_1,c) \right)\md x_1=+\int_{\gamma_3}\omega+\int_{\gamma_1}\omega\\
    \implies\int_Q\md\omega=\int_{\gamma_1}\omega+\int_{\gamma_2}\omega+\int_{\gamma_3}\omega+\int_{\gamma_4}\omega=\int_{\partial Q}\omega
  \end{gather*}
\end{Bew}
\begin{Bem}
  Ist $\omega=\md f$ ($\omega_1=\partial_1f$, $\omega_2=\partial_2f$)
  \[\int_{\partial Q}\omega=0\]
  und $\md\omega = 0$
  \[\md\omega = \left( \partial_1\omega_2-\partial_2\omega_1 \right)\md^2x=\left( \partial_1\partial_2f-\partial_2\partial_1f \right)\md^2 x \stackrel{\text{Schwarz}}{=}0\]
  Umkehrung gilt im Allgemeinen nicht!
\end{Bem}
\begin{Bsp}
  \begin{gather*}
    \omega=\frac{x\md y-y\md x}{2}\\
    \omega_1=\frac{-x_2}{2},\ \omega_2=\frac{x_1}{2}\\
    \md\omega = \left( \frac{1}{2}+\frac{1}{2} \right)\md^2 x=\md^2x\\
    \int_Q\md \omega=\left( b-a \right)\left( c-d \right)\\
    \int_{\partial Q}\omega = \ \text{Sektorfläche}
  \end{gather*}
\end{Bsp}
\begin{Sat}{Satz von Stokes} Sei $X$ ein stetig differenzierbares Vektorfeld auf $U\subset\mb{R}^2$. Sei $Q=\left[ a;b \right]\times\left[ c;d \right]\subset U$
  \[\int_{\partial Q}X\md x=\int_Q\rot X\md^2x\]
\end{Sat}
\begin{Def}{Rotation von $X$}
  \[\rot X:=\partial_1X_2-\partial_2X_1\]
\end{Def}
\begin{Bem}
  \[X=\nabla f\implies \rot X=0\]
  Im Allgemeinen gilt die Umkehrung nicht.
\end{Bem}
\begin{Bem}
  Verallgemeinerungen:
  \begin{itemize}
    \item andere Integrationsbereiche
    \item höhere Dimensionen
  \end{itemize}
  z.B. Gausscher Integralsatz in 3 Dimensionen (Differenzierbare Mannigfaltigkeiten)
\end{Bem}
\section{Differenzierbare Abbildungen}
\begin{Def}{Differenzierbare Abbildungen}
  Sei $\mb{K}=\mb{R}$ oder $\mb{C}$. Seien $X$ und $Y$ normierte Vektorräume über $\mb{K}$. Sei $U\subset X$ offen, $a\in U$. Eine Abbildung $f:U\to Y$ heisst differenzierbar im Punkt $a$, wenn es eine stetige lineare Abbildung $L:X\to Y$ gibt, s.d.
  \[\Limo{h}\frac{f(a+h)-f(a)-Lh}{\Norm{h}_x}\]
\end{Def}
\begin{Bem}
  $R(h):=f(a+h)-f(a)-Lh$ Rest:
  \[\lim\frac{R(h)}{\Norm{h}_x}=0\]
\end{Bem}
\begin{Bem}
  $L$ ist eindeutig bestimmt, wenn es existiert.
\end{Bem}
\begin{Bem}
  Ist $X$ endlichdimensional, dann
  \begin{itemize}
    \item $L$ linear ist automatisch stetig
    \item Die Wahl der Norm auf $X$ spielt keine Rolle.
  \end{itemize}
\end{Bem}
\begin{Bem}
  Von jetzt an $X$ und $Y$ endlichdimensional.
\end{Bem}
\begin{Bem}
  Jeder $\mb{C}$-Vektorraum ist ein $\mb{R}$-Vektorraum. Jede $\mb{C}$-lineare Abbildung ist $\mb{R}$-linear. 
  \[\mb{C}-\text{Differenzierbarkeit}\implies\mb{R}-\text{Differenzierbarkeit}\]
  Die Umkehrung gilt im Allgemeinen nicht.
\end{Bem}
\begin{Bsp}
  \begin{align*}
    f:\mb{C}&\to\mb{C}\\
    z&\mapsto \bar z
  \end{align*}
  Ist $\mb{R}$-differenzierbar aber nicht $\mb{C}$-differenzierbar.
\end{Bsp}
\begin{Bem}
  $\mb{C}$-differenzierbare Abbildungen: Funktionentheorie
\end{Bem}
\begin{Not}{Differential und Linearisierung}
  \[\md f_a=L\]
  Differential von $f$ im Punkt $a$, Linearisierung von $f$ im Punkt $a$
\end{Not}
\begin{Not}{Funktionalmatrix}
  $\dim X=n$, $\dim Y=n$
  \begin{align*}
    \md f_a\in \Hom_\mb{K}\left( x,y \right)&\cong M(m\times n,\mb{K})\\
    \md f_a&\mapsto f'(a)
  \end{align*}
  Die darstellende Matrix ist die Funktionalmatrix oder Ableitung
\end{Not}
\begin{Bem}{Funktionaldeterminante}
  Ist $m=n$, dann $\det d f_a$ Funktionaldeterminante
\end{Bem}
\begin{Bsp}
  Sei $A\in M(m\times n, \mb{K})$, sei $b\in\mb{K}^n$
  \begin{align*}
    f:\mb{k}^n&\to\mb{K}^m\\
    f(x)&=Ax+b
  \end{align*}
  ist auf ganz $\mb{K}^n$ differenzierbar.
  \begin{gather*}
    f(a+h)=A(a+h)-b=Ah+\overbrace{(Aa+b)}^{=f(a)}\\
    f(a+h)-f(a)=AH,\ R(h)=0\\
    f'(a)=A
  \end{gather*}
\end{Bsp}

\begin{Sat}
  \begin{align*}
    M(n\times n, \mb{K})&\to M(n\times n,\mb{K})\\
    A&\mapsto A^2
  \end{align*}
  ist überall differenzierbar
\end{Sat}
\begin{Bew}
  \begin{gather*}
    f(A+h)-f(A)=(A+h)^2-A^2=A^2+hA+Ah+h^2-A^2\\
    \md f_Ah=hA+Ah\\
    R(h)=h^2
  \end{gather*}
  zu zeigen:
  \[\Limo{h}\frac{h^2}{\Norm{h}}=0\]
  für irgendeine Norm\\
  Operatornorm:
  \begin{gather*}
    \Norm{R(h)}=\Norm{h^2}\leq\Norm{h}^2\\
    \Norm{\frac{R(h)}{\Norm{h}}}\leq\Norm{h}\xrightarrow{h\to 0}0
  \end{gather*}
\end{Bew}
\subsection{Operatornorm}
\begin{Def}{Operatornorm}
  Seinen $V,W$ normierte Vektorräume
  \[L(V,W):=\left\{ \text{stetige lineare Abbildungen $V\to W$} \right\}\]
  \[\left( \dim < \infty:L(V,W)=\Hom(V,W) \right)\]
  Sei $A\in L(V,W)$.
  \[\Norm{A}:=\sup\left\{ \Norm{Ax}_W \ \text{mit}\ x\in V\ \Norm{x}_V\leq 1 \right\}\]
\end{Def}
\begin{Bem}
  $\Norm{\ }$ ist wohldefiniert.
  \[A\ \text{stetig linear}\ \implies\exists C:\Norm{Ax}_W\leq C\Norm{x}_V\]
\end{Bem}
\begin{Lem}
  $\Norm{\ }$ ist eine Norm auf $L(V,W)$.
\end{Lem}
\begin{Bem}
  $x=\mb{K}^n$, $Y=\mb{K}^m$, $A\in M(m\times n, \mb{K})$
  \[\Norm{A}=\max_i\sum_{j=1}^n\Abs{a_{ij}}\]
\end{Bem}
\begin{Sat}
  $\forall x\in V$, $\forall A\in L(V,W)$
  \[\Norm{Ax}_W\leq\Norm{A}\Norm{x}_V\]
  $\forall A\in L(V,W)$, $B\in L(U,V)$
  \[\Norm{AB}\leq \Norm{A}\Norm{B}\]
\end{Sat}
\begin{Not}
  $Y_1$, $Y_2$ Vektorräume
  \begin{align*}
    Y_1\otimes Y_2=Y_1\times Y_2&&\text{als Menge}
  \end{align*}
  \begin{gather*}
    \Mx{a\\b}+\Mx{c\\d}:=\Mx{a+c\\b+d}\\
    \lambda\Mx{a\\b}:=\Mx{\lambda a\\ \lambda b}
  \end{gather*}
\end{Not}
\begin{Lem}{Reduktionslemma}
  Sei $U$ oft $\subset X$
  \begin{align*}
    f:U&\to Y_1\otimes Y_2\\
    x&\mapsto \Mx{f_1(x)\\f_2(x)}
  \end{align*}
  \[\text{$f$ differenzierbar in $a\in U$}\iff\text{$f_1$ und $f_2$ differenzierbar in $a\in U$}\]
  In diesem
  \begin{gather*}
    \underbrace{\md f(a)}_{\Hom\left( x,Y_1\otimes Y_2 \right)}=\underbrace{\Mx{\md f_1(a)\\\md f_2(a(}}_{\in \Hom\left( x,Y_1 \right)\otimes \Hom (x,Y_2)} =\md f_1(a)\otimes \md f_2(a)\\
    \md f_1\in\Hom(x,Y_1)\\
    \md f_2\in \Hom(x,Y_2)
  \end{gather*}
\end{Lem}
\begin{Bew}
  $\La$
  \begin{gather*}
    f_1(a+h)=f(a)+\md f_1(a)h+R_1(h)\\
    f_2(a+h)=f(a)+\md f_2(a)h+R_2(h)\\
    f(a+h)=\Mx{f_1(a+h)\\f_2(a+h)}=\underbrace{\Mx{f_1(a)\\f_2(a)}}_{=f(a)}+\underbrace{\Mx{\md f_1(a)h\\\md f_2(a)h}}_{\md f(a)h}+\underbrace{\Mx{R_1(h)\\R_2(h)}}_{=R (h)}\\
    \frac{R_1(h)}{\Norm{h}_x}\to 0\\
    \frac{R_2(h)}{\Norm{h}_x}\to 0\\
    \frac{R(h)}{\Norm{h}_x}=\Mx{\frac{R_1(h)}{\Norm{h}_x}\\\frac{R_2(h)}{\Norm{h}_x}}\to 0
  \end{gather*}
\end{Bew}
\begin{Kor}
  Sei $U\subset X$
  \begin{align*}
    f:U&\to\mb{K}^m=\overbrace{\mb{K}\otimes\mb{K}\otimes\cdots\otimes\mb{K}}^{\text{$m$ Körper}}\\
    x&\mapsto \Mx{f_1(x)\\\vdots\\f_n(x)}
  \end{align*}
  \[\text{$f$ diffbar in $a\in U$}\iff\text{$f_i$ diffbar in $a$ $\forall i$}\]
  In diesem Falle
  \[\md f(a)=\Mx{\md f_1(a)\\\vdots\\\md f_n(a)}\]
  Vektor von Linearformen
\end{Kor}
\begin{Bem}
  Ist $x=\mb{K}^n$
  \[\md f_i(a)=\left( \partial_1f_i(a),\cdots,\partial_nf_i(a) \right)\]
\end{Bem}
\begin{Kor}
  Sei $U\subset\mb{K}^n$
  \[f:U\to\mb{K}^n\]
  \[\text{$f$ diffbar in $a$}\iff\text{$f_i$ diffbar in $a$ $\forall i$}\]
  und in diesem Falle
  \begin{gather*}
    f'(a)=\Mx{f_1'(a)\\\vdots\\f_n'(a)}=\Mx{\partial_1f_1(a)&\partial_2f_1(a)&\cdots&\partial_nf_1(a)\\\vdots&\vdots&&\vdots\\\partial_1f_n(a)&\partial_2f_n(a)&\cdots&\partial_{nf_n(a)}}\\
  \end{gather*}
  \[f'(a)_{ij}=\partial_jf_i(a)=\Part{f_i(a)}{x_j}\]
  \begin{gather*}
    \md f_ah=\sum_{i=1}^m\sum_{j=1}^n\Part{f_i}{x_j}(a)h_je_i
  \end{gather*}
  $e_i$  Standardbasis von $\mb{K}^m$
\end{Kor}
\begin{Kor}
  Sei $U\subset\mb{R}^n$
  \[f:U\to\mb{R}^n\]
  $f$ ist in $a$ $\mb{R}$-differenzierbar, wenn alle partiellen Ableitungen $\partial_if_i$
  \begin{enumerate}
    \item in einer Umgebung von $a$ existieren
    \item in $a$ stetig sind
  \end{enumerate}
\end{Kor}
\begin{Def}{Richtungsableitung}
  Sei $U\subset X$, $a\in U$, $a\in U$
  \[f:U\to Y^n\]
  \[\partial_nf(a):=\Limo{t}\frac{f(a+th)-f(a)}{t}\]
  wenn der Grenzwert existiert
\end{Def}
\begin{Kor}
  Ist $f$ diffbar in $a$
  \[\partial_nf(a)=\md f_a h\]
\end{Kor}
\begin{Bem}{Spezialfall}
  $X=\mb{K}^n$, $\left\{ e_1,\cdots,e_n \right\}$ Standardbasis
  \[\partial_if(a):=\partial_{e_i}f(a)\in Y\]
\end{Bem}
\begin{Def}{stetig differenzierbar}
  Sei $f:U\to Y$ diffbar auf ganz $U$\\
  $f$ heisst stetig differenzierbar, wenn
  \begin{align*}
    \md f:U&\to L(x,y)\\
    x&\mapsto \md f_x
  \end{align*}
  stetig ist.
\end{Def}
\begin{Bem}
  \[\mathcal{C}^1(U,Y):= \left\{ f:U\to Y\ \text{stetig differenzierbar} \right\}\]
  ist ein Vektorraum
\end{Bem}
\begin{Lem}
  Sei $U\subset X$
  \begin{align*}
    f:U&\to\mb{K}^m\\
    x&\mapsto \Mx{f_1(x)\\\vdots\\f_m(x)}
  \end{align*}
  \[\text{$f$ stetig diffbar}\iff\text{$f_i$ stetig diffbar $\forall i$}\]
\end{Lem}
\begin{Bew}
  \[\md f=\Mx{\md f_1\\\vdots\\\md f_m}\]
  \[\text{Stetigkeit}\iff\text{komponentenweise Stetigkeit}\]
\end{Bew}
\begin{Kor}
  Sei $U\subset\mb{R}^n$
  \[f:U\to\mb{R}^m\]
  \[\text{$f$ stetig diffbar im reellen Sinne}\iff\text{$\partial_if_i$ stetig $\forall i$ $\forall j$}\]
\end{Kor}
\begin{Def}
  Sei $U\in\mb{K}^n$
  \[f:U\to\mb{K}^m\]
  heisst $k$-mal stetig differenzierbar, wenn alle Komponentenfunktionen $f_1,\cdots,f_n$ $k$-mal stetig diffbar sind.
  \[\mathcal{C}^k\left( U,\mb{K}^m \right):=\left\{ \text{$k$-mal stetig diffbare Abbildungen $U\to\mb{K}^m$} \right\}\]
  \[\mathcal{C}^\infty\left( U,\mb{K}^m \right)=\cap_k\mathcal{C}^k\left( U,\mb{K}^m \right)\]
  sind Vektorräume und
  \[\mathcal{C}^{k+l}\stackrel{l\geq 0}{\subset}\mathcal{C}^k\]
\end{Def}
\begin{Bem}{Rechenregeln}
  \begin{align*}
    \md(f+g)&=\md f+\md g
    \md(\lambda f)&=\lambda \md f &\lambda\in \mb{K}
  \end{align*}
\end{Bem}
\begin{Bem}{Kettenregel}
  Seien $U\subset Y$, $V\subset X$
  \begin{align*}
    f:U&\to Z\\
    g:V\to U\subset Y
  \end{align*}
  Sei $g$ diffbar in $a\in V$ und $f$ diffbar in $b=g(a)\subset$. Dann $f\circ g$ diffbar in $a$ und
  \begin{align*}
    \md\left( f\circ g \right)_a=\md f_b\circ \md g_a&&\text{Abbildungskomposition}
  \end{align*}
  Funktionalmatrizen
  \begin{align*}
    \left( f\circ g \right)'(a)=f'(b)g'(a)&&\text{Matrixmultiplikation}
  \end{align*}
  \[\Part{\left( f\circ g \right)_i}{\partial x_j}=\sum_k\Part{f_i}{y_k}\Part{g_k}{x_j}\]
\end{Bem}
\begin{Lem}
  Seien $Y_1,Y_2,Z$ endliche normierte Vektorräume. Sei
  \[\beta:Y_1\times Y_2\to Z\]
  bilinear. Dann ist $\beta$ stetig differenzierbar. und
  \[\md \beta_{\left( a_1,a_2 \right)}\left( h_1,h_2 \right)=\beta\left( h_1,a_2 \right)+\beta\left( a_1,h_2 \right)\]
\end{Lem}
\begin{Bew}
  \begin{gather*}
    \beta\left( a_1+h_1, a_2+h_2\right)=\beta\left( a_1,a_2 \right)+\beta\left( h_1,a_2 \right)+\beta\left( a_1,h_2 \right)+\beta\left( h_1,h_2 \right)\\
    R=\beta\left( h_1,h_2 \right)
  \end{gather*}
  zu zeigen
  \[\Limo{\left( h_1,h_2 \right)}\frac{\beta\left( h_1,h_2 \right)}{\Norm{\left( h_1,h_2 \right)}}=0\]
  Seien
  \begin{align*}
    f_1:U&\to Y_1\\
    f_2:U&\to Y_2
  \end{align*}
  differenzierbar. Dann ist
  \begin{align*}
    \beta\circ f_1\times f_2:U&\to Z\\
    x&\mapsto \beta\left( f_1(x),f_2(x) \right)
  \end{align*}
  differenzierbar.
  \begin{gather*}
    \md\left( \beta\circ f_1\times f_2 \right)_ah=\left( \md f_1(a)h,f_2(a) \right)+\left( f_1(a)\md f_2(a)h \right)
  \end{gather*}
\end{Bew}
\begin{Bsp}
  Sei $I\subset\mb{R}$
  \[f,g:I\to\mb{R}^n\]
  $\beta$ Skalarprodukt
  \[\left( fg \right)'=f'g+fg'\]
\end{Bsp}
\begin{Bsp}
  Sei
  \[f,g:I\to\mb{R}^3\]
  $\beta$ Vektorprodukt
  \[\left( f\times g \right)'=f'\times g+f\times g'\]
\end{Bsp}
\begin{Bsp}
  Sei
  \begin{align*}
    M(m\times n)\times M(n\times l)&\to M(m\times l)\\
    \left( A,B \right)&\mapsto AB
  \end{align*}
  ist stetig differenzierbar
  \begin{gather*}
    \left( AB \right)_{ij}=\underbrace{\sum_ka_{ik}b_kl}_{\text{bilinear}}\\
    \left( AB \right)_{ij}\implies\left( AB \right)_{ij}\ \text{stetig diffbar}\ \forall ij
  \end{gather*}
  \begin{gather*}
    \md m_{A,B}(h,k)=hB+Ak\\
    \left( A+h \right)\left( B+k \right)-AB
  \end{gather*}
\end{Bsp}
\begin{Def}
  Sei 
  \begin{align*}
    \det:m(n\times n,\mb{K})&\to\mb{K}\\
    A&\mapsto \det A
  \end{align*}
  ist stetig differenzierbar ($\det$ ist eine Summe von Produkten von Einträgen)
\end{Def}
\begin{Kor}
  \[GL_n:=\left\{ A\in m(n\times n),\det A\neq 0 \right\}\]
  \begin{align*}
    \phi:GL_n&\to GL_n\\
    A&\mapsto A^{-1}
  \end{align*}
  ist stetig differenzierbar $\forall i$ und
  \[\md \phi_A h=-A^{-1}hA^{-1}\] % rot hier
  Verallgemeinerung: $f:I\to\mb{R}$
  \[\left( \frac{1}{f} \right)'=-\frac{f'}{f^2}\]
\end{Kor}
\begin{Bew}
  Cramersche Regel
  \[\left( A^{-1} \right)_{ij}=\frac{\det\left( \cdots \right)}{\det A}\]
  $\left( A^{-1} \right)_{ij}$ ist eine ratinale Funktoin der Einträge, deshalb stetig differenziarbar.
  \begin{gather*}
    m(A,B)=AB\\
    \psi(A):=m\left( A,\phi(A) \right)=AA^{-1}=Id
  \end{gather*}
  $\psi$ konstant
  \begin{align*}
    \md \psi_Ah=0&&\forall A\in GL_n\ \forall h\in A+h\in GL_n
  \end{align*}
  Kettenregel
  \begin{gather*}
    =m\left( h,\phi(A) \right)+m\left( A,\md \phi_Ah \right)=hA^{-1}+A\md \phi_Ah\\
    A\md \phi_Ah=-h^{-1}\\
    \md \phi_Ah=-A^{-1}hA^{-1}
  \end{gather*}
\end{Bew}
\begin{Bem}{Lie-Gruppe}
  $GL_n$ ist eine Gruppe, die Multiplikation und die Inversion stetig differenzierbare Abbildungen sind. Das ist ein Beispiel einer Lie-Gruppe.
\end{Bem}
\subsection{Schrankensatz}
\begin{Def}{Supremumsnorm}
  Sei $K$ kompakter Raum. Sei $V$ normierter Raum. Sei $\phi:K\to V$ stetig.
  \[\Norm{\phi}_K:=\sup_{x\in K}\Norm{\phi(x)}_V\]
\end{Def}
\begin{Lem}
  $\Norm{\ }_K$ ist eine Norm auf $\mathcal{C}\left( K,V \right)$
\end{Lem}
\begin{Sat}{Schrankensatz}
  Sei $f\in\mathcal{C}^1\left( U,Y \right)$. Sei $K\subset U$ kompakt. Dann ist $f|_K$ Lipschitz-stetig $\forall x,y\in K$
  \[\Norm{f(x)-f(y)}_Y\leq \Norm{\md f}_K\Norm{x-y}_X\]
  \[\md f|_K\in \mathcal{C}\left( K,\underbrace{L(x,y)}_{\text{Op-Norm}} \right)\]
\end{Sat}
\begin{Bem}
  Falls $X=\mb{K}^n$, $Y=\mb{K}^m$ $A\in M(m\times x)$
  \[\Norm{A}=\max_i\sum_j\Abs{a_{ij}}\]
  \[\Norm{\md f}_K=\sup_{\xi\in K}\max_i\sum_j\Abs{\partial_jf_i(\xi)}\]
\end{Bem}


\newpage

%= Stichwortverzeichnis ======================================================================
\rhead{}
\addcontentsline{toc}{section}{Stichwortverzeichnis}
\printindex

\end{document}
