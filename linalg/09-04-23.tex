\subsection{Das Tensorprodukt}
\[V\ \rsa\ V^*\ \text{Linearformen}\]
Wir möchten eine Redeweise haben, genug flexibel für solche Situationen. Eine nützliche Konstruktion dafür ist das Tensorprodukt.
\begin{Def}{Tensorprodukt}
  Sei $\mb{K}$ ein Körper und $V$ und $W$ Vektorräume über $\mb{K}$. Ein $\mb{K}$-Vektorraum heisst Tensorprodukt von $V$ und $W$, geschrieben $V\otimes W$, falls, eine bilineare Abbildung
  \[\eta:V\times W\to V\otimes W\]
  gegeben ist, die folgende universelle Eigenschaften erfüllt:\\
  \item Zu jedem $\mb{K}$-Vektorraum $U$ mit bilinearer Abbildung
    \[\xi:V\times W\to U\]
    gibt es eine eindeutige lineare Abbildung 
    \[\xi':V\otimes W\to U\]
    so dass das Diagramm
    \[\begindc{\commdiag}[60]
      \obj(0,0)[ot]{$V\otimes W$}
      \obj(0,1)[t]{$V\times W$}
      \obj(1,0)[U]{$U$}
      \mor{t}{U}{$\xi$}
      \mor{ot}{U}{$\xi'$}
      \mor{t}{ot}{$\eta$}
    \enddc\]
    kommutiert.
\end{Def}
\begin{Bem}
  Es ist noch unklar, wie die Elemente von $V\otimes W$ aussehen, oder ob $V\otimes W$ gar existiert. Auch unklar: warum.
\end{Bem}
\begin{Kor}
  Aus der Definition folgt: $V\otimes W$, falls es existiert, ist eindeutig bis auf Isomorphismus.
\end{Kor}
\begin{Bem}
  Seien
  \[\eta: V\times W\to V\otimes W\]
  und
  \[\tilde\eta: V\times W\to \widetilde{V\otimes W}\]
  gegeben, beide erfüllen die universellen Eigenschaften.
  Wir wenden die universelle Eigenschaft an, mit $U:=\widetilde{V\otimes W}$, so dass das folgende Diagramm kommutiert:
  \[\begindc{\commdiag}[60]
    \obj(0,0)[ot]{$V\otimes W$}
    \obj(1,0)[ti]{$\widetilde{V\otimes W}$}
    \obj(0,1)[t]{$V\times W$}
    \mor{t}{ti}{$\tilde\eta$}
    \mor{ot}{ti}{$\zeta$}
    \mor{t}{ot}{$\eta$}
  \enddc\]
  Wir wenden die universelle Eigenschaft an mit $U:=V\otimes W$, so dass das folgende Diagramm kommutiert:
  \[\begindc{\commdiag}[60]
    \obj(1,0)[ot]{$V\otimes W$}
    \obj(0,0)[ti]{$\widetilde{V\otimes W}$}
    \obj(0,1)[t]{$V\times W$}
    \mor{ti}{ot}{$\tilde\zeta$}
    \mor{t}{ti}{$\tilde\eta$}
    \mor{t}{ot}{$\eta$}
  \enddc\]
  Wir wenden die universelle Eigenschaft an mit $U:=V\otimes W$
  \[\begindc{\commdiag}[60]
    \obj(1,0)[ot]{$V\otimes W$}
    \obj(0,0)[ot2]{$V\otimes W$}
    \obj(0,1)[t]{$V\times W$}
    \mor{t}{ot}{$\eta$}
    \mor{t}{ot2}{$\eta$}
    \mor{ot2}{ot}{$\tilde\zeta\circ\zeta$}
  \enddc
  \ \ \ 
  \begindc{\commdiag}[60]
    \obj(1,0)[ot]{$V\otimes W$}
    \obj(0,0)[ot2]{$V\otimes W$}
    \obj(0,1)[t]{$V\times W$}
    \mor{t}{ot}{$\eta$}
    \mor{t}{ot2}{$\eta$}
    \mor{ot2}{ot}{$1_{V\otimes W}$}
  \enddc\]
  Beide Diagramme kommutieren
  \[\tilde\zeta\circ\zeta\circ\eta=\tilde\zeta\circ\tilde\eta=\eta\]
  \[1_{V\otimes W}\circ\eta=\eta\]
  Es folgt aus der universellen Eigenschaft, dass
  \[\tilde\zeta\circ\zeta=1_{V\otimes W}\]
  Wir wenden die universelle Eigenschaft an, mit $U:=\widetilde{V\otimes W}$
  \[\begindc{\commdiag}[60]
    \obj(1,0)[ot]{$V\otimes W$}
    \obj(0,0)[ot2]{$\widetilde{V\otimes W}$}
    \obj(0,1)[t]{$\tilde{V\times W}$}
    \mor{t}{ot}{$\tilde\eta$}
    \mor{t}{ot2}{$\tilde\eta$}
    \mor{ot2}{ot}{$\tilde\zeta\circ\zeta$}
  \enddc
  \ \ \ 
  \begindc{\commdiag}[60]
    \obj(1,0)[ot]{$\widetilde{V\otimes W}$}
    \obj(0,0)[ot2]{$\widetilde{V\otimes W}$}
    \obj(0,1)[t]{$V\times W$}
    \mor{t}{ot}{$\tilde\eta$}
    \mor{t}{ot2}{$\tilde\eta$}
    \mor{ot2}{ot}{$1_{\widetilde{V\otimes W}}$}
  \enddc\]
  Beide Diagramme kommutieren
  \[\zeta\circ\tilde\zeta\circ\tilde\eta=\zeta\circ\eta=\tilde\eta\]
  \[1_{\widetilde{V\otimes W}}\circ\tilde\eta=\tilde\eta\]
  Es folgt aus der universellen Eigenschaft, dass
  \[\zeta\circ\tilde\zeta=1_{\widetilde{V\otimes W}}\]
  Ergebnis:
  \[V\otimes W\xrightarrow{\zeta}\widetilde{V\otimes W}\]
  ist Isomorphismus, inverse zu
  \[\tilde\zeta:\widetilde{V\otimes W}\to V\otimes W\]
\end{Bem}
\begin{Faz}
  universelle Eigenschaft $\rsa$ Eindeutigkeit bis auf Isomorphismus
\end{Faz}
\begin{Not}{Tensorprodukt}
  $V\otimes W$ oder $V\otimes_\mb{K} W$
\end{Not}
\subsubsection{Existenz vom Tensorprodukt}
\begin{Bem}
  Es gibt zwei Methoden
  \begin{itemize}
    \item Durch Auswahl von Basen
    \item Beschreibung als Quotientenvektorraum
  \end{itemize}
  Heute: Methode 1
  \begin{itemize}
    \item braucht die Existenz von Basen
    \item klar fall $\dim_\mb{K} V< \infty$
  \end{itemize}
  oder im Allgemeinen für die, die das Auswahlaxiom gesehen haben.
\end{Bem}
\begin{Prop}
  Sei $(v_i)_{i\in I}$ eine Basis von $V$ und $(w_j)_{j\in J}$ eine Basis von $W$. Dann existiert das Tensorprodukt $V\otimes W$, mit Basis $(v_i\otimes w_j)_{(i,j)\in I\times J}$ und 
  \[\eta:V\times W\to V \otimes W\]
  \[\left( \sum_{i\in I}a_iv_i \sum_{j\in J}b_jw_j \right)\mapsto \sum_{(i,i)\in I\times J}a_ib_j\left( v_i\otimes w_j \right)\]
  mit $a_i\neq 0$ für endlich viele $i$ und $b_j\neq 0$ für endlich viele $j$\\
  Das bedeutet, dass die Elemente von $V\otimes W$
  \[\sum_{(i,j)\in I\times J}c_{ij}\left( v_i\otimes w_j \right)\ c_{ij}\in\mb{K}\]
  nur endlich viele $c_{ij}\neq 0$.
\end{Prop}
\begin{Bew}
  Zu verifizieren:
  \begin{itemize}
    \item dass $\eta$ bilinear ist
    \item und erfüllt die universelle Eigenschaft
  \end{itemize}
  $\eta$ ist bilinear:
  \begin{gather*}
    \eta\left( \sum_{i\in I} a_iv_i,\ \sum_{j\in J} b_jw_j \right)+\eta\left( \sum_{i\in I}a_i'v_i,\ \sum_{j\in J} b_jw_j \right)=\\
    =\sum_{i\in I} a_ib_j\left( v_i\otimes w_j \right)+\sum_{\left( i,\ j \right)\in I\times J}a_i'b_j\left( v_i\otimes w_j \right)=\\
    =\sum_{\left( i,\ j \right)\in I\times J}\left( a_ib_j+a_i'b_j \right)\left( v_i\otimes w_j \right)=\\
    =\sum_{\left( i,\ j \right)\in I\times J}\left( a_i+a_i' \right)b_j\left( v_i\otimes w_j \right)=\\
    =\eta\left( \sum_{i\in I}(a_i1a_i')v_i,\ \sum_{j\in J}b_jw_j \right)\\
    =\eta\left( \sum_{i\in I}a_iv_i+\sum_{i\in I}a_i'v_j,\ \sum_{j\in J}b_jw_j \right)
  \end{gather*}
\end{Bew}
\begin{Bsp}
  $V=\mb{K}^2$, $W=\mb{K}[t]$\\
  $V$ hat die Standardbasis $(e_1,e_2)$\\
  $W$ hat die Basis $(1,t,t^2,\cdots)$\\
  $\implies$ $V\otimes W$ hat die Basis $e_1\otimes 1,e_1\otimes t,\cdots$, $w_2\otimes1, e_2\otimes t,\cdots$
  \begin{align*}
    \eta:\mb{K}^2\times K[t]&\to K^2\otimes K[t]\\
    \left( (1,0),t^2 \right)&\mapsto e_1\otimes t^2\\
    \left( (0,1),t^3 \right)&\mapsto e_2\otimes t^3\\
    \left( (2,3),t^4 \right)&\mapsto 2e_1\otimes t^4+3e_2\otimes t^4
  \end{align*}
  Typisches Element von $\mb{K}^2\otimes \mb{K}[t]$
  \[e_1\otimes t^2+e_2\otimes t^3+2e_1\otimes t^4+3e_2\otimes t^4\]
  Mit anderer Schreibweise:
  \[(t^2,0)+(0,t^3)+(2t^4,0)+(0,3t^4)\]
  oder:
  \[(t^2+2t^4,t^3+3t^4)\]
\end{Bsp}
\begin{Def}
  Sei $U$ ein $\mb{K}$-Vektorraum und 
  \[\xi:V\times W\to U\]
  eine bilineare Abbildung. Wir definieren
  \[\begindc{\commdiag}[60]
    \obj(0,0)[ot]{$V\otimes W$}
    \obj(0,1)[t]{$V\times W$}
    \obj(1,0)[U]{$U$}
    \mor{t}{U}{$\xi$}
    \mor{ot}{U}{$\xi'$}
    \mor{t}{ot}{$\eta$}
  \enddc\]
  \[\xi':V\otimes W\to U\]
  durch
  \[\xi'(v_i \otimes w_j):=\xi(v_i,w_j)\]
  für $i\in I$, $j\in J$, und deshalb:
  \[\xi'\left( \sum_{(i,j)\in I\times J} c_{ij}v_i\otimes w_j \right)= \sum_{(i,j)\in I\times J} c_{ij}(v_i,w_j)\]
  Das Diagram kommutiert (aus der Bilinearität von $\xi$)\\
  Die Eindeutigkeit ist durch die identischen Basenvektoren gegeben.
\end{Def}
