\subsubsection{Spezialfälle}
\begin{Bem}{Spezialfall 1}
  $V,V',W,W'$ endlichdimensional mit gegebenen Basen $\dim V=m$ Basis $v_1,\cdots,v_m$, $\dim W=n$ Basis $w_1,\cdots,w_m$, $\dim V'=m'$ Basis $v'_1,\cdots,v'_m$, $\dim W'=n'$  Basis $w'_1,\cdots,w'_m$ dargestellt durch $\phi$ dargestellt als $A\in M(m'\times m, \mb{K})$, $\psi$ durch $B\in M(n'\times n),\mb{K}$. Wie wir schon gesehen haben, hat $V\otimes W$ die Basis 
  \[\Mx{v_1\otimes w_1&\cdots&v_1\otimes w_n\\\cdots&\cdots&\cdots\\v_m\otimes w_1&\cdots&v_m\otimes w_n}\]
  und $V'\otimes W'$ hat die Basis
  \[\Mx{v'_1\otimes w'_1&\cdots&v'_1\otimes w'_n\\\cdots&\cdots&\cdots\\v'_m\otimes w'_1&\cdots&v'_m\otimes w'_n}\]
  Wir betrachten die darstellende Matrix von $\phi\otimes\psi$, Basiselement $v_i\otimes w_j$ von $V\otimes W$ geht auf was?
  \[\dcp{40}{
  \obj(0,0)[vow]{$v_i\otimes w_j$}
  \obj(0,2)[vw]{$\left( v_i,w_j \right)$}
  \obj(2,2)[phipsi]{$\left( \phi(v_i),\psi(w_j) \right)$}
  \obj(6,2)[bas]{$\left( a_{1i}v_i'+\cdots+a_{m'i}v_{m'}', b_{1j}w_j'+\cdots+b_{n'j}w_{n'}' \right)$}
  \obj(6,1)[otimes]{$\left( a_{1i}v_i'+\cdots+a_{m'i}v_{m'}' \right)\otimes \left( b_{1j}w_1'+\cdots+b_{n'j}w_{n'}' \right) =$}
  \mor{vw}{vow}{$\eta$}[1,6]
  \mor{vw}{phipsi}{\ }[1,6]
  \mor{phipsi}{otimes}{$\eta'$}[-1,6]
  \mor{vow}{otimes}{$\phi\otimes\psi$}[1,6]
  \mor{phipsi}{bas}{$=$}[1,2]
  }\]

  \begin{gather*}
  =a_{1i}b_{1j}\left( v_1'\otimes w_1' \right)+a_{1i}b_{2j}\left( v_1'\otimes w_2' \right)+\cdots+a_{1i}b_{n'j}\left( v_1'\otimes w_{n'}' \right)+ \\
  a_{2i}b_{1j}\left( v_2'\otimes w_1' \right)+\cdots+a_{2i}b_{n'j}\left( v_2'\otimes w_{n'}'\right)+\\
  \cdots +a_{m'i}b_{1j}\left( v_{m'}\otimes w_1' \right)+\cdots +a_{m'i}b_{n'j}\left( v_{m'}'\otimes w_{n'}' \right)
  \end{gather*}
  $A\otimes B$ heisst die darstellende Matrix von $\phi\otimes\psi$ und sieht so aus:
  \[\Mx{
  a_{11}b_{11}&a_{11}b_{12}&\cdots&a_{11}b_{1n}&\cdots& a_{1m}b_{11}&\cdots&a_{1m}b_{1n} \\
  a_{11}b_{21}&a_{11}b_{22}&\cdots&a_{11}b_{2n}&\cdots& a_{1m}b_{21}&\cdots&a_{1m}b_{2n}\\
  \vdots&\vdots& &\vdots& &\vdots& &\vdots \\
  a_{11}b_{n'1}&a_{11}b_{n'2}&\cdots&a_{11}b_{n'n}&\cdots&a_{1m}b_{n'1}&\cdots&a_{1n}b_{n'n} \\
  \vdots&\vdots& &\vdots& &\vdots& &\vdots \\
  a_{m'1}b_{11}&a_{m'1}b_{12}&\cdots&a_{m'1}b_{1n}&\cdots&a_{m'm}b_{11}&\cdots&a_{m'm}b_{1n} \\
  \vdots&\vdots& &\vdots& &\vdots& &\vdots \\
  a_{m'1}b_{n'1}&a_{m'1}b_{n'2}&\cdots&a_{m'1}b_{n'n}&\cdots&a_{m'm}b_{n'1}&\cdots&a_{m'm}b_{n'n} } \in M(m'n'\times mn,\mb{K}\]
  Diese Matrix lässt sich in Blockmatrixen unterteilen:
  \[\Mx{
  a_{11}B&\cdots&a_{1m}B\\
  a_{21}B&\cdots&a_{2m}B\\
  \vdots& &\vdots\\
  a_{m1}B&\cdots&a_{m'm}B
  } = A\otimes B\]
\end{Bem}
\begin{Bsp}
  \[\Mx{2&3\\1&-1}\otimes\Mx{1&2\\3&4}=\Mx{2&4&3&6\\6&8&9&12\\1&2&-1&-2\\3&4&-3&-4}\]
\end{Bsp}
\begin{Bem}{Spezialfall 2}
  $V'=V$ und $\phi=1_V$. Dann folgt aus $\psi:W\to W'$
  \[V\otimes_\mb{K}W\xrightarrow{1_v\otimes \psi}V\otimes_\mb{K}W'\]
  $1_V\otimes \psi$ kann auch mit $V\otimes \psi$ bezeichnet werden.
\end{Bem}
\begin{Bsp}
  $\mb{K}=\mb{R}$, $V=\mb{C}$ $\rsa$ Komplexifizierung einer linearen Abbildung
  \[\psi:W\to W'\rsa\mb{C}\otimes\psi:\mb{C}\otimes_\mb{R}W\to \mb{C}\otimes_\mb{R}W'\]
  wenn $\dim W, \dim W'<\infty$, darstellende Matrix $B=(b_{ij}), b_{ij}\in\mb{R}$, so bekommen wir $\mb{C}\otimes B$, mit derselben Grösse, denselben Einträgen, aber als komplexe Zahlen betrachtet. Ähnlich für eine beliebige Körpererweiterung $\mb{K}\to\mb{L}$
  \[\psi:W\to W'\ \rsa\ \mb{L}\otimes \psi:\mb{L}W\to\mb{L}\otimes W'\]
\end{Bsp}
\begin{Bem}{Spezialfall 3}
  $V'=W'=\mb{K}$, und so $\phi\in V^*$, $\psi\in W^*$
  \[\rsa\ \phi\otimes \psi:V\otimes_\mb{K}W\to \underbrace{\mb{K}\otimes_\mb{K}\mb{K}}_{\text{Basis}\ \left( 1\otimes 1 \right)\mapsto 1}\approx \mb{K}\]
  So können wir schreiben
  \[\phi\otimes \psi:V\otimes_\mb{K}W\to\mb{K}\]
  d.h.
  \[\phi\otimes\psi\in\left( V\otimes_\mb{K}W \right)^*\]
  Anscheinend hat $\phi\otimes \psi$ auch eine Bedeutung als Element von $V^*\otimes_\mb{K}W^*$ Es gibt einen Zusammenhang:
  \[\dcp{50}{
  \obj(0,0)[vow]{$V^*\otimes W^*$}
  \obj(0,1)[vtw]{$V^*\times W^*$}
  \obj(1,0)[vow*]{$\left( V\otimes W \right)^*$}
  \mor{vtw}{vow}{\ }
  \mor{vtw}{vow*}{$\left( \phi,\psi\ \right)\mapsto\phi\otimes\psi$}
  }\]
\end{Bem}
\begin{Sat}
  \begin{align*}
    V^*\times W^*&\to\left( V\otimes W \right)^*\\
    \left( \psi,\phi \right)&\mapsto\phi\otimes\psi    
  \end{align*}
  ist eine bilineare Abbildung.
\end{Sat}
\begin{Bew}
  \[\dcp{40}{
  \obj(0,1)[vtw]{$V\times W$}
  \obj(0,0)[vow]{$V\otimes W$}
  \obj(2,0)[k]{$\mb{K}$}
  \obj(2,1)[kk]{$\mb{K}\times\mb{K}$}
  \mor{vtw}{vow}{\ }
  \mor{vtw}{kk}{$\left( \phi+\tilde\phi \right)\times\psi$}
  \mor{kk}{k}{\ }
  \mor{vow}{k}{$\left( \phi+\tilde\phi\otimes\psi \right)$}
  }\]
  \begin{itemize}
    \item \[\left( \psi+\tilde\psi,\phi \right)\mapsto\left( \phi+\tilde\phi \right)\otimes \stackrel{?}{=} \phi\otimes\phi+\tilde\phi\otimes\psi\]
      ok
    \item \ldots
  \end{itemize}
\end{Bew}
\begin{Prop}
  Sei $\mb{K}$ ein Körper und $V,W$ endlichdimensionale Vektorräume über $\mb{K}$. Die oben beschriebene lineare Abbildung
  \[V^*\otimes W^*\to\left( V\otimes W \right)^*\]
  ist ein Isomorphismus
\end{Prop}
\begin{Bew}
  Wir rechnen mit Basen. Seien $\left( v_1,\cdots,v_n \right)$ Basis in $V$ mit dualen Basen $\left( v_1^*,\cdots,v_m^* \right)$ in $V^*$. Seien $\left( w_1,\cdots,w_m \right)$ Basis in $W$ mit dualen Basen $\left( w_1^*,\cdots,w_n^* \right)$ in $W^*$. Dann ist $\left( v_i^*\otimes w_j^* \right)_{1\leq i\leq m, 1\leq j\leq n}$ eine Vasis von $V^*\otimes W^*$
  \begin{gather*}
    V\xrightarrow{v_i^*}\mb{K}\\
    W\xrightarrow{w_j^*}\mb{K}
  \end{gather*}
  \begin{align*}
    V\otimes W&\xrightarrow{v_i^*\otimes w_j^*}\mb{K}\\
    v_i\otimes w_j\otimes&\mapsto\delta_{ik}\cdot\delta_{jl}
  \end{align*}
  Wir erkennen das als Basiselement von $\left( V\otimes W \right)^*$. Wir können auch $\Hom_\mb{K}\left( V,W \right)$ interpretieren als Tensorprodukt. $V,W$ Vektorräume über $\mb{K}$
  \[\dcp{30}{
  \obj(0,0)[vow]{$V^*\otimes W$}
  \obj(0,2)[vtw]{$V^*\times W$}
  \obj(0,3)[pw]{$\left( \phi,w \right)$}
  \obj(3,3)[map]{$v\mapsto\psi(v)w$}
  \obj(3,2)[hom]{$\Hom(V,W)$}
  \mor{vtw}{vow}{}
  \mor{pw}{map}{}[1,6]
  \mor{vow}{hom}{}[1,1]
  \mor{vtw}{hom}{}
  }\]
  Beh: Das ist eine bilineare Abbildung \ldots
\end{Bew}
\begin{Prop}
  Sei $\mb{K}$ ein Körper, $V,W$ endlichdimensionale $\mb{K}$-Vektorräume. Dann ist die lineare Abbildung
  \[V^*\otimes W\to\Hom(V,W)\]
  ist ein Isomorphismus.
\end{Prop}
\begin{Bew}
  Eine ähnliche Berechnung mit Basen.
\end{Bew}
\begin{Bsp}
  $V=W=\mb{R}^3$
  A = \[e_2^*\otimes e_3\mapsto \Mx{e_1\mapsto 0 \\ e_2\mapsto e_3\\e_3\mapsto 0}\]
  oder durch eine Matrix
  \[\Mx{0&0&0\\0&0&0\\0&1&0}\]
  $\implies$
  \begin{gather*}
    \Mx{1&0&2\\0&-1&1\\1&2&0}\ \lra\ e_1^*\otimes e_1+2e_3^*\otimes e_1-e_2^*\otimes e_2+e_3^*\otimes e_2+e_1^*\otimes e_3+e_2^*\otimes e_3\\
    =\left( e_1^*+2e_3^* \right)\otimes e_1+\left( -e_2^*+e_3^* \right)\otimes e_2+\left( e_1^*+2e_2^* \right)\otimes e_3 \stackrel{?}{=} \left( \  \right)\otimes \left( \  \right)+\left( \  \right)\otimes \left( \ \right)
  \end{gather*}
  Standardbasis $(e_1,e_2,e_3)$, Basiswechsel von $V$ und $W$
  \begin{gather*}
    \Mx{1&0&2\\0&-1&1\\1&2&0} = \Mx{1&0&0\\0&-1&0\\1&2&1}\Mx{1&0&0\\0&1&0\\0&0&0}\Mx{1&0&2\\0&1&-1\\0&0&1}
  \end{gather*}
  oder
  \begin{gather*}
    \Mx{1&0&2\\0&-1&1\\1&2&0}\Mx{1&0&-2\\0&1&1\\0&0&1}=\Mx{1&0&0\\0&-1&0\\1&2&1}\Mx{1&0&0\\0&1&0\\0&0&0}
  \end{gather*}
  Neue Basen von $V$
  \begin{align*}
    f_1&=\left( 1,0,0 \right)\\
    f_2&=\left( 0,1,0 \right)\\
    f_3&=\left( -2,1,1 \right)
  \end{align*}
  Neue Basen von $W$
  \begin{align*}
    f_1&=\left( 1,0,1 \right)\\
    f_2&=\left( 0,-1,2 \right)\\
    f_3&=\left( 0,0,1 \right)
  \end{align*}
  so:
  \begin{align*}
    f_1&\mapsto g_1\\
    f_2&\mapsto g_2\\
    f_3&\mapsto 0
  \end{align*}
  Das bedeutet, wir können $\phi$ als
  \[f_1^*\otimes g_1+f_2^*\otimes g_2\]
  schreiben.\\
  Zu finden: $f_i^*$ bezüglich $\left( e_1^*,e_2^*,e_3^* \right)$. Zeilen aus der Transformationsmatrix:
  \begin{align*}
    f_1^*&=e_1^*+2e_3^*\\
    f_2^*&=e_2^*-e_3^*\\
    f_3^*&=e_3^*
  \end{align*}
  So haben wir
  \[\phi\ \lra\ \left( e_1^*+2e_3^* \right)\otimes \left( e_1+e_3 \right)+\left( e_2^*-e_3^* \right)\otimes \left( -e_2+2e_3 \right)\]
  checken
  \[e_1^*\otimes e_1+2e_3^*\otimes e_1+e_1^*\otimes e_3+2e_3^*\otimes e_3 -e_2^*\otimes e_2+e_3^*\otimes e_2+2e_2^*\otimes e_3-2e_3^*\otimes e_3\]
\end{Bsp}
