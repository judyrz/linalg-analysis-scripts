\begin{Bem}
  \[\dim V\otimes W=\left( \dim V \right)\left( \dim W \right)\]
\end{Bem}
\begin{Not}
  Für $v\in V$, $w\in W$ schreibt man oft $v\otimes w$ für $\eta\left( v,w \right)$
\end{Not}
\begin{Bem}
  Weil
  \[\left( v,w \right)\mapsto v\otimes w:=\eta\left( v,w \right)\]
  eine bilineare Abbildung ist, haben wir
  \begin{align*}
    v\otimes w+v'\otimes w&=\left( v\times v'\right)\otimes w\\
    v\otimes w+v\otimes w'&=v\otimes \left(w'\times w\right)\\
    \left( \lambda v \right)\otimes w=\lambda\left( v\otimes w \right)=v\otimes\left( \lambda w) \right)
  \end{align*}
  Rechenregeln für Tensoren
\end{Bem}
\begin{Not}
  Letztes Mal: für Basiselemente $v_i$, $w_j$ bezeichnet $v_i\otimes w_j$ ein Basiselement von $V\otimes W$
\end{Not}
\begin{Bem}
  Die Abbildung lässt sich schreiben als
  \begin{align*}
    V\times W&\xrightarrow{\eta}V\otimes W\\
    \left( v_i,w_j \right)&\mapsto v_i\otimes w_j
  \end{align*}
\end{Bem}
\begin{Bem}
  Jetzt für beliebige $v\in V$, $w\in W$ bezeichnet $v\otimes w$ das Element
  \[\eta\left( v,w \right)\in V\otimes W\]
  Weil in der Konstruktion wir $\eta$ durch
  \[\eta\left( v_i,w_j \right):=v_i\otimes w_j\]
  definiert haben, stimmen die beiden Bedeutungen von $v_i\otimes w_j$ überein.
\end{Bem}
\begin{Bsp}{Isomorphismus von Vektorräumen über $\mb{R}$}
  \begin{align*}
    \mb{R}^2\otimes_\mb{R}\mb{R}[t]&\xrightarrow{\sim}\mb{R}[t]\oplus\mb{R}[t]\\
    e_1\otimes t^j&\mapsto\left( t^j,0 \right)\\
    e_2\otimes t^j&\mapsto\left( 0,t^j \right)
  \end{align*}
  Ganz ähnlich
  \begin{align*}
    \mb{C}\otimes_\mb{R}\mb{R}[t]&\xrightarrow{\sim}\mb{C}[t]\\
    1\otimes t^j&\mapsto t^j\\
    i\otimes t^j&\mapsto it^j
  \end{align*}
  $\implies$ für $\gamma\in \mb{C}$ gilt:
  \[\gamma\otimes t^j\mapsto \gamma t^j\]
  weil: 
  \[\gamma=\alpha+\beta i\ \alpha,\beta\in \mb{R}\]
  haben wir
  \begin{align*}
    \gamma \otimes t'&=\left( \alpha+\beta i \right)\otimes t^j\\
    &=\alpha\otimes t^j+\beta i\otimes t^j\\
    &=\alpha\left( 1\otimes t^j \right)+\beta\left( i\otimes t^j \right)\\
    &\mapsto \alpha\left( t^j \right)+\beta(it^j)\\
    &=\left( \alpha+\beta i \right)t^j\\
    &=\gamma t^j
  \end{align*}
\end{Bsp}
\begin{Prop}
  Sei $\mb{K}\to\mb{L}$ eine Körpererweiterung und $V$ ein $\mb{K}$-Vektorraum. Dann hat $K\otimes_\mb{K}V$ die Struktur von $L$-Vektorraum, mit
  \[\alpha\left( \beta\otimes v \right)=\left( \alpha\beta \right)\otimes v\ \ \text{für $\alpha,\beta\in L$ und $v\in V$}\]
\end{Prop}
\begin{Bew}
  Wir müssen verifizieren, dass
  \[\left( \alpha,\beta\otimes v \right)\mapsto \left( \alpha\beta \right)\otimes v\]
  eine Abbildung von $L\times\left( L\otimes_\mb{K}V \right)$ nach $L\otimes_\mb{K}V$ beschreibt. D.h. für jedes $\alpha\in L$, $\exists$ eine Abbildung $L\otimes V\to L\otimes V$
  \[\begindc{\commdiag}[100]
  \obj(0,0)[LoV1]{$L\otimes V$}
  \obj(1,0)[LoV2]{$L\otimes V$}
  \obj(0,1)[LtV1]{$L\times V$}
  \obj(1,1)[LtV2]{$L\times V$}
  \mor{LtV1}{LtV2}{$\left( \beta,\alpha \right)\mapsto \left( \alpha\beta,v \right)$}
  \mor{LtV1}{LoV1}{$\eta$}
  \mor{LtV1}{LoV2}{$\left( \alpha,\beta \right)\mapsto \left( \alpha\beta \right)\otimes v$}[1,0]
  \mor{LtV2}{LoV2}{$\eta$}
  \mor{LoV1}{LoV2}{Aus der u.E.}[-1,1]
  \enddc\]
  \[\left( \alpha,\beta \right)\mapsto \left( \alpha\beta \right)\otimes v\]
  ist bilinear:
  \begin{align*}
    \left( \beta+\beta',v \right)\mapsto\left( \alpha\left( \beta+\beta' \right) \right)\otimes v=&\\
    &=\left( \alpha\beta+\alpha\beta' \right)\otimes v&\text{Körpereigenschaft}\\
    &=\alpha\beta\otimes v+\alpha\beta'\otimes v&\text{Rechenregeln für Tensoren}
  \end{align*}
  So bekommen wir
  \begin{align*}
    L\times \left( L\otimes V \right)&\to L\otimes V\\
    \left( \alpha,\beta\otimes v \right)&\mapsto \left( \alpha\beta \right)\otimes v
  \end{align*}
  Wir müssen auch die Axiome für den Vektorraum über $L$ verifizieren, d.h.:
  \begin{align*}
    \alpha\left( w+w' \right)&=\alpha w+\alpha w'&\text{für $\alpha\in L$, $w,w'\in L\otimes V$}\\
    \alpha\left( \alpha'w \right)&=\left( \alpha\alpha' \right)w&\text{für $\alpha,\alpha'\in L$, $w\in L\otimes V$}
  \end{align*}
  Die erste Gleichung gilt weil $L\otimes V\dashrightarrow L\otimes V$ über $\mb{K}$-linear ist. Die zweite folgt aus der ersten, falls wir nur den Fall verifizieren, wobei $w=\beta\otimes v$. Dafür benutzen wir
  \begin{itemize}
    \item $L\otimes_\mb{K} V$ ist von Elementen $\beta\otimes v$ ($\beta\in L$, $v\in V$) aufgespannt, als $\mb{K}$-Vektorraum (Klar von der Konstruktion)
    \item Dann können wir schreiben
      \begin{align*}
        w&=\beta_1\otimes v_1+\cdots+\beta_\gamma\otimes v_\gamma&\gamma\in\mb{N}
      \end{align*}
  \end{itemize}
  \begin{align*}
    \alpha\left( \alpha'\left( \beta_1\otimes v_1+\cdots+ \beta_\gamma\otimes v_\gamma\right) \right)&=\alpha\left( \alpha'\left( \beta_1 \otimes v_1\right)+\cdots+\alpha'\left( \beta_\gamma \otimes v_\gamma\right) \right)\\
    &=\alpha\left( \alpha'\left( \beta_1\otimes v_1 \right) \right)+\alpha\left( \alpha'\left( \beta_\gamma \right) \right)\\
    &=\left( \alpha\alpha' \right)\left( \beta_1\otimes v_1 \right)+\cdots+\left( \alpha\alpha' \right)\left( \beta_\gamma\otimes v_\gamma \right)\\
    &=\left( \alpha\alpha' \right)\left( \beta_1\otimes v_1 +\cdots+ \beta_\gamma\otimes v_\gamma\right)
  \end{align*}
  für $w:=\beta\otimes v$:
  \begin{align*}
    \alpha\left( \alpha'\left( \beta\otimes v \right) \right)&=\alpha\left( \left( \alpha'\beta \right)\otimes v\right)\\
    &=\left( \alpha\left( \alpha'\beta \right) \right)\otimes v\\
    &=\left( \left( \alpha\alpha' \right)\beta \right)\otimes v\\
    &=\left( \alpha\alpha' \right)\left( \beta\otimes v \right)
  \end{align*}
\end{Bew}
\begin{Sat}
  \[\underbrace{\mb{C}\otimes_\mb{R}\mb{R}[t]}_{\mb{C}-\text{Vektorraum}}\to\mb{C}[t]\]
  Beh: dies ist ein Isomorphismus von $\mb{C}$-Vektorräumen. Nur noch zu verifzieren: die $\mb{C}$-Linearität. Im allgemeinen haben wir
  \[L\otimes_\mb{K}\mb{K}[t]\xrightarrow{\sim}L[t]\]
  $L$-linearer Isomorphismus
\end{Sat}
\begin{Bew}
  Sei $\gamma\in \mb{C}$
  \[\begindc{\commdiag}[40]
  \obj(0,0)[CR2]{$\mb{C}\otimes_\mb{R}\mb{R}[t]$}
  \obj(1,1)[g(g]{}
  \obj(2,1)[(gg)]{$\gamma\left( \gamma'\otimes t^j \right)=\left( \gamma\gamma'\right)\otimes t^j $}
  \obj(0,4)[CR1]{$\mb{C}\otimes_\mb{R}\mb{R}[t]$}
  \obj(5,0)[C2]{$\mb{C}[t]$}
  \obj(5,4)[C1]{$\mb{C}[t]$}
  \obj(4,1)[gg]{$\gamma\gamma't^j$}
  \obj(4,3)[gt]{$\gamma t^j$}
  \obj(1,3)[g']{$\gamma'\oplus t^j$}
  \mor{g'}{(gg)}{}[1,6]
  \mor{(gg)}{gg}{}[1,6]
  \mor{gt}{gg}{}[1,6]
  \mor{g'}{gt}{}[1,6]
  \mor{CR1}{C1}{$\sim$}
  \mor{CR2}{C2}{$\sim$}
  \mor{CR1}{CR2}{$\gamma\cdot\left( \ \right)$}[-1,0]
  \mor{C1}{C2}{$\gamma\cdot\left( \  \right)$}
  \enddc\]
\end{Bew}
\begin{Bem}
  Ganz ähnlich
  \[L\otimes \mb{K}^n\xrightarrow{\sim}L^n\]
  $L$-linearer Isomorphimsus. ($n\in\mb{N}$) Insbesondere:
  \[\mb{C}\otimes_\mb{R}\mb{R}^n\xrightarrow{\sim}\mb{C}^n\]
\end{Bem}
\begin{Def}{Komplexifizierung}
  Ist $V$ ein $\mb{R}$-Vektorraum, so heisst der $\mb{C}$-Vektorraum $\mb{C}\otimes_\mb{R}V$ die Komplexifizierung von $V$
\end{Def}
\subsubsection{Tensorprodukt von linearen Abbildungen}
\begin{Def}{Tensorprodukt von linearen Abbildungen}
  Sei $V$, $W$, $V\otimes_\mb{K}W$, $V\times W\xrightarrow{\eta}V\otimes_\mb{K}W$ und $V'$, $W'$, $V'\otimes_\mb{K}W'$, $V'\times W'\xrightarrow{\eta'}V'\otimes_\mb{K}W'$ gegeben, mit linearen Abbildungen
  \begin{gather*}
    V\xrightarrow{\phi}V'\\
    W\xrightarrow{\psi}W'
  \end{gather*}
  Dann haben wir:
  \[\dcp{40}{
  \obj(0,1)[VW]{$V\times W$}
  \obj(0,0)[VoW]{$V\otimes W$}
  \obj(2,0)[V'o]{$V'\otimes W'$}
  \obj(2,1)[V't]{$V'\times W'$}
  \mor{VW}{VoW}{$\eta$}
  \mor{VW}{V't}{$\phi\times\psi$}
  \mor{V't}{V'o}{$\eta'$}
  \mor{VoW}{V'o}{neu}[1,1]
  }\]
  Beh: die Komposition ist bilinear
  \[\left( v,w \right)\mapsto \psi(v)\otimes \psi(w)\]
  (neu) aus der universellen Eigenschaft das Tensorprodukt von linearen Abbildungen
\end{Def}
\begin{Not}{Tensorprodukt von linearen Abbildungen}
  \[\phi\otimes \psi\]
\end{Not}
