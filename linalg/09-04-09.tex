\subsection{Zusammenhang zwischen Dualraum und bilinearen Abbildungen}
\begin{itemize}
  \item schon gesehen, z.B. bei der Definition ``nicht ausgeartet''
  \item jetzt explizit
\end{itemize}
\begin{Def}{bilineare Abbildung}
  Sei $\mb{K}$ ein Körper, $v$ und $W$ Vektorräume über $\mb{K}$. Eine Abbildung $b:V\times W\to \mb{K}$ heisst bilinear falls 
  \[w\mapsto b(v,w)\ \text{ist linear}\  \forall v\in V\]
  und
  \[v\mapsto b(v,w)\ \text{ist linear}\  \forall w\in W\]
  \[\left[ (w\mapsto b(v,w))\in W^* \right]\]
\end{Def}
\begin{Bem}
  Im Fall $W=V$ ist dies genau zu sagen, dass $b$ eine Bilinearform ist. Also haben wir Abbildungen
  \[b':V\to W^*\]
  und
  \[b'':W\to V^*\]
  Aus der Definition folgt, dass $b'$ und $b''$ sind linear.
\end{Bem}
\begin{Def}{nicht ausgeartet}
  Eine bilineare Abbildung $b:V\times W\to\mb{K}$ ist nicht ausgeartet, falls $b':V\to W^*$ und $b'':W\to V^*$ injektiv sind.
\end{Def}
\begin{Bem}
  Falls $V$ und $W$ endlich dimensional sind, ist es nur möglich, eine nicht ausgeartete bilineare Abbildung zu haben, wenn $\dim V=\dim W$. Falls $\dim V=\dim W:$ ``injektiv'' oben ist äquivalent zu ``bijektiv''.
\end{Bem}
\begin{Bsp}
  \begin{itemize}
    \item $V$ beliebig, dann ist
      \begin{align*}
        V\times V^*&\to\mb{K}\\
        (v,\phi)&\mapsto \phi(v)
      \end{align*}
      stets nicht ausgeartet.
      \begin{align*}
        b':V&\to V^{**}\\
        v&\mapsto e v_v
      \end{align*}
      ist die kanonische Abbildung, ist injektiv
      \begin{align*}
        b'':V^*&\to V^*\\
        \phi&\mapsto\phi
      \end{align*}
      ist $id_{V^*}$ ist ein Isomorpismus
    \item $\mb{K}=\mb{R}$, $\left\langle , \right\rangle $ Skalarprodukt auf $V$.
      \[b(v,w):=\left\langle v,w \right\rangle\]
      $b'$ und $b''$ sind gleich, definiert als $\Psi$
      \[\rsa \Psi:V\to V^*\]
      injektiv\\
  \end{itemize}
\end{Bsp}
\begin{Bem}
  Das zeigt, dass jedes Skalarprodukt nicht ausgeartet ist. Und: falls $\dim_\mb{R}V<\infty$ ist $\Psi$ ein Isomorphismus. $\Psi$ heisst kanonisch. (kanonische Abbildung bzw. kanonischer Isomorphismus)
\end{Bem}
\begin{Eig}{$V$, $\dim V = n$, mit Skalarprodukt, kanonischer Isomorphismus $\Psi$}
  \begin{itemize}
    \item Für $U\subset V$ Untervektorraum gilt 
      \[\Psi(U^\perp)=U^0\]
    \item Für $B=(v_1,\cdots,v_n)$ eine orthonormal Basis haben wir 
      \[\Psi(v_i)=V^*_i\]
      für $i=1,\cdots,n$, wobei $(v_1^*,\cdots,v_n^*)$ die duale Basis ist.\\
      zeigen:
      \[\underbrace{\Psi(U^\perp)}_{\dim = \dim V-\dim U}\subset \underbrace{U^0}_{\dim=\dim v-\dim U}\ \text{klar}\]
      \[\left\langle v_i,\sum^n_{j=1}a_jv_j \right\rangle =a_i\]
      \[v_i^*\left( \sum^n_{j=1}a_jv_j \right)=a_i\]
  \end{itemize}
\end{Eig}
\begin{Bsp}
  Graphiker gesucht ;)
\end{Bsp}
\begin{Bem}
  Wir haben zwei kanonische Abbildungen:
  \[V\to V^{**}\ \text{für beliebiges $V$ / $\mb{K}$}\]
  \[V\xrightarrow{\Psi} V^*\ \text{für $V/\mb{R}$ mit Skalarprodukt}\]
\end{Bem}
\begin{Def}{adjugierte Abbildung}
  $V$, $W$ euklidische Vektorräume
  \[F:V\to W\ \text{lineare Abbildung}\]
  adjugiert: $F^{ad}:W\to V$ ist adjugiert zu $F$ falls gilt
  \[\left\langle F(v),w \right\rangle =\left\langle v,F^{ad}(w) \right\rangle \ \forall v\in V, w\in W\]
\end{Def}
\begin{Bem}
  \[\begindc{\commdiag}[60]
  \obj(0,1)[V]{$V$}
  \obj(1,0)[W*]{$W^*$}
  \obj(0,0)[V*]{$V^*$}
  \obj(1,1)[W]{$W$}
  \mor{W}{V}{$F^{ad}$}
  \mor{V}{V*}{$\Phi$}
  \mor{W*}{V*}{$F^*$}
  \mor{W}{W*}{$\Psi$}
  \enddc\]
  \begin{gather*}
    F^*(\Psi(w))(v)=\Psi(w)\left( F(v) \right)=\Phi\left( F^{ad}(w) \right)(v)\\
    \implies F^*\left( \Phi(w) \right)=\Phi\left( F^{ad}(w) \right)\ \text{in}\ V^*    
  \end{gather*}
  Daraus folgt, dass das Diagramm kommutiert
\end{Bem}
\begin{Bem}
  Seien $v_1,\cdots,v_n$ orthonormale Basen von $V$, $w_1,\cdots,w_m$ für $W$ $\rsa$ duale Basen $v_1^*,\cdots,v_n^*$ und $w_1^*,\cdots,w_m^*$
  Bezüglich orthonormaler Basen ist $F^{ad}$ durch die transponierte Matrix gegeben:
  Sei
  \[F\ \lra\ A\in M(m\times n,\mb{R})\]
  dann, aus Prop 49 (13?):
  \[F^*\ \lra\ A^t\in M(m\times n,\mb{R})\implies F^{ad}\ \lra A^t\ \text{weil} \Phi(v_i)=v_i^*,\ \Psi(w_i)=w_i^*\ \forall i\]
\end{Bem}
\begin{Bsp}
  $V=\mb{R}^2$ mit Skalarprodukt, $W=\mb{R}[x]^{\leq 2}$ mit \[\left\langle f,g \right\rangle =\int_{-1}^1f(x)g(x)\md x\]
  \begin{align*}
    F:V&\to W\\
    (\alpha,\beta)&\mapsto \alpha+\beta x+\alpha x^2
  \end{align*}
  Basis von $W$ $1,x,x^2$
  \[V^*=(\mb{R}^2)^*\ \text{mit Basis}\ e_1^*,e_2^*\]
  %TODO ergänzen
  \[\begindc{\commdiag}[60]
  \obj(0,1)[V]{$V$}
  \obj(1,0)[W*]{$W^*$}
  \obj(0,0)[V*]{$V^*$}
  \obj(1,1)[W]{$W$}
  \mor{W}{V}{$F^{ad}$}
  \mor{V}{V*}{$\Phi$}
  \mor{W*}{V*}{$F^*$}
  \mor{W}{W*}{$\Psi$}
  \enddc\]
  \begin{align*}
    V&\xrightarrow{\Psi}V^*\\
    e_1&\mapsto e_1^*\\
    e_2&\mapsto e_2^*
  \end{align*}
  $W^*$ hat die duale Basis $1^*,x^*,x^{2^*}$.\\
  Wir berechnen $\Psi$ explizit:
  \begin{align*}
    \Psi(1)=\left( f\mapsto \int^1_{-1}f(x)\md x \right)
  \end{align*}
  \begin{align*}
    W&\xrightarrow{\Phi}W^*\\
    1&\mapsto 2(1^*)+\frac{2}{3}\left( x^{2^*} \right)\\
    x&\mapsto \cdots\\
    x^2&\mapsto \cdots
  \end{align*}
  Dann:
  \[F^*\left( \psi(1) \right)=\left(\left( \alpha,\beta \right)\int^1_{-1}\alpha+\beta x+\alpha x^2\md x=2\alpha+\frac{2}{3}\alpha=\frac{8}{3}\alpha\right)\]
  d.h.
  \[\frac{8}{3}e^*_1\stackrel{\Phi^{-1}}{\mapsto}\left( \frac{8}{3},0 \right)\]
  Ähnlich:
  \[F^*\left( \Psi(x) \right)=\left( (\alpha,\beta) \mapsto \int^1_{-1}\alpha x+\beta x^2+\alpha x^3\md x=\frac{2}{3}\beta\right)\]
  und
  \[F^*\left( \Psi(x^2) \right)=\left( (\alpha,\beta) \mapsto \int^1_{-1}\alpha x^2+\beta x^3+\alpha x^4\md x=\frac{2}{3}\alpha + \frac{2}{5}\alpha=\frac{16}{15}\alpha\right)\]
  d.h.
  \begin{align*}
    F^{ad}(1)&=\left( \frac{8}{3},0 \right)\\
    F^{ad}(x)&=\left( 0,\frac{2}{3} \right)\\
    F^{ad}(x^2)&=\left( \frac{16}{15},0 \right)
  \end{align*}
  \[F^{ad}\left( a+bx+cx^2 \right)=\left( \frac{8}{3}a+\frac{16}{5}c,\frac{2}{3}b \right)\]
  Check:
  \[\int^1_{-1}\left( \alpha+\beta x+\alpha x^2 \right)\left( a+bx+cx^2 \right)\md x \stackrel{?}{=} \left\langle \left( \alpha,\beta \right),\left( \frac{8}{3}a+\frac{16}{15}c,\frac{2}{3}b \right) \right\rangle \]
  Skalarprodukt:
  \[\alpha\left( \frac{8}{3}a+\frac{16}{15}c \right)+\beta\left( \frac{2}{3}b \right)\]
  Integral:
  \begin{gather*}
    \int^1_{-1}a\alpha+(b\alpha+c\beta)x+(c\alpha+b\beta+a\alpha)x^2+\left( c\beta+\alpha b \right)x^3+c\alpha x^4\md x=\\
    =2a\alpha+\frac{2}{3}\left( a\alpha+b\beta+c\alpha \right)+\frac{2}{5}c\alpha
  \end{gather*}
  stimmt.
\end{Bsp}
\begin{Bem}
  Wir könnten $F^{ad}$ auch durch die Wahl einer orthonormalen Basis von $W$ berechnen.
  \[\frac{1}{\sqrt{2}},\sqrt{\frac{3}{2}}x,\sqrt{\frac{5}{2}}\sqrt{\frac{-1+3x^2}{2}}\]
  Dann:
  \[A=\Mx{\frac{4}{3}\sqrt{2}&0\\0&\sqrt{\frac{2}{3}}\\\frac{2}{3}\sqrt{\frac{2}{3}}}\]
  und so
  \[A^t=\Mx{\frac{4}{3}&0&\frac{2}{3}\sqrt{\frac{2}{3}}\\0&\sqrt{\frac{2}{3}}&0}\to F^{ad}\]
\end{Bem}
