\begin{Bem}
  Jetzt betrachen wir den Fall $W=V$, also $b:V\times V\to\mb{K}$ eine Bilinearform. Da $b'$ und $b''$ genau durch das ``Umtauschen'' von $V$ und $W$ unterschieden
  \begin{gather*}
    b':V\to V^*,\ v\mapsto\left( w\mapsto b(v,w) \right)\\
    b'':V\to V^*,\ w\mapsto\left( v\mapsto b(v,w) \right)\\
  \end{gather*}
  haben wir Interpretation von Bedingungen über $b$:
  \begin{itemize}
    \item $b$ symmetrisch $\Lra$ $b''=b'$
    \item $b$ schiefsymmetrisch $\Lra$ $b''=-b'$
  \end{itemize}
\end{Bem}
\begin{Bem}
  Sei jetzt $\dim_\mb{K}<\infty$. Dann haben wir $b''=(b')^*$ in folgendem Sinne: Dual zu $b':V\to V^*$ ist
  \[\begindc{\commdiag}[50]
  \obj(0,1)[V**]{$V^{**}$}
  \obj(0,0)[V]{$V$}
  \obj(1,1)[V*]{$V^*$}
  \mor{V}{V**}{$\sim$}
  \mor{V**}{V*}{$(b')^*$}
  \enddc
  \ \ \
  \begindc{\commdiag}[50]
  \obj(0,1)[ev]{$ev_V$}
  \obj(0,0)[v]{$v$}
  \obj(1,1)[evc]{$ev_V\circ b'$}
  \mor{v}{ev}{$ $}[0,6]
  \mor{ev}{evc}{$ $}[0,6]
  \enddc\]
  ($ev_V$ = Auswertungsabbildung an $v\in V$)\\ Wobei:
  \[ev_V\circ b'(v')=ev_V\left( w\mapsto b(v',w) \right)=b(v',v)\]
  Da
  \[b'':V\to V^*\]
  durch
  \[ v\mapsto \left( v'\mapsto b(v',v) \right)\]
  definiert ist, haben wir Gleichheit.
\end{Bem}
\begin{Eig}
  \[b\ \text{symmetrisch}\iff b''=b'\iff (b')^*=b'\]
  \[b\ \text{schiefsymmetrisch}\iff b''=-b'\iff (b')^*=-b'\]
\end{Eig}
\begin{Bem}
  Falls $\dim_\mb{K}V<\infty$, $s$ symmetrisch und nicht ausgeartet führt zu
  \[b'(=b'')\ V\xrightarrow{\sim}V^*\]
\end{Bem}
\begin{Bem}{Spezialfall}
  $\mb{K}=\mb{R}$, $V$ euklidisch, dann ist des gerade das, was $\Psi$ hiess:\\
  Untevektorraum $U\subset V, U^\perp\subset V$ sowie $U^0\subset V^*$
  \[\begindc{\commdiag}[40]
  \obj(0,0)[Up]{$U^\perp$}
  \obj(0,1)[V]{$V$}
  \obj(1,0)[U0]{$U^0$}
  \obj(1,1)[V*]{$V^*$}
  \mor{Up}{V}{$\cup$}[0,2]
  \mor{U0}{V*}{$\cup$}[0,2]
  \mor{V}{V*}{$\sim$}
  \enddc\]
\end{Bem}
\begin{Bem}
  Was passiert, falls $K=\mb{C}$? Dann sind wir an sesquilinearen Abbildungen interessiert.
  \[s:V\times W\to\mb{C}\]
  Dann gibt es immer noch eine Abbildung
  \[s'':W\to V^*\ w\mapsto\left( v\mapsto s(v,w) \right)\]
  aber diese ist nicht mehr linear.
\end{Bem}
\begin{Bsp}
  $V=\mb{C}[x],s:V\times V\to\mb{C}$
  \[s(f,g)=\int^1_0f(x)\overline{g(x)}\md x\]
  Dann z.B.
  \[1\xmapsto{s''}\left( f\mapsto \int_0^1f(x)\md x \right)\]
  also: Durchschnittswert auf $[0,1]$\\
  aber:
  \[i\xmapsto{s''}\left( f\mapsto -i\int_0^1f(x)\md x \right)\]
  also: $(-i)\cdot$ Durchschnittswert auf $[0,1]$\\
  Damit ist $s''$ semilinear.
\end{Bsp}
\begin{Def}{semilinear}
  Eine Abbildung $t:V\to W$ zwischen $\mb{C}$-Vektorräumen heisst semilinear, falls:
  \begin{itemize}
    \item $t(v+v')=t(v)+t(v')$ $\forall v,v'\in V$
    \item $t(\lambda v)=\bar\lambda(v)$ $\forall v\in V,\ \lambda\in \mb{C}$
  \end{itemize}
\end{Def}
\begin{Def}{kanonischer Semi-Isomorphismus}
  Falls $V$ ein unitärer Vektorraum ist, mit Skalarprodukt
  \[s:V\times V\to\mb{C}\]
  so erhalten wir (was oben $s''$ heisst, nennen wir hier $\Psi$)
  \[\Psi:V\to V^*\ \text{kanonischer $\underbrace{\text{Semi}}_{\text{semilinear}}$-$\underbrace{\text{Isomorphismus}}_{\text{bijektiv}}$}\]
\end{Def}
\begin{Bem}
  Wie vorher haben wir zu einem Endomorphismus
  \[F:V\to V\]
  den adjugierten Endomorphismus
  \[F^{ad}:V\to V\]
  gegeben durch
    \[F^{ad}:=\Psi^{-1}\circ F^*\circ \Psi\]
\end{Bem}
\begin{Eig}{adjugierter Endomorphismus}
  \begin{itemize}
    \item 
      \[s\left( F(v),w \right)=s\left( v,F^{ad}(w) \right)\ \forall v,w\in V\]
    \item 
      \[\Im F^{ad}=\left( \Ker F \right)^\perp\]
    \item
      \[\Ker F^{ad}=<\left( \Im F \right)^\perp\]
    \item Ist $B$ eine Orthonormalbasis von $V$ und $A$ die darstellende Matrix von $F$ bezüglich $B$, dann ist $\bar A^t$ die darstellende Matrix von $F^{ad}$
  \end{itemize}
\end{Eig}
\begin{Sat}
  \[F\ \text{ist unitär diagonalisierbar} \iff F\circ F^{ad}=F^{ad}\circ F\]
\end{Sat}
\begin{Def}{$F$ normal}
  $F$ heisst normal, falls
  \[F\circ F^{ad}=F^{ad}\circ F\]
\end{Def}
\subsection{Anwendung des Dualraums}
das duale Polytop $\mb{K}=\mb{R}, V=\mb{R}^n$
\begin{Def}{konvexe Menge}
 $S\subset\mb{R}^n$ mit der Eigenschaft $\forall s,t\in S$: $s$ und $t$ sind wegzusammenhängend.
\end{Def}
\begin{Def}{konvexe Hülle}
  konvexe Hülle von $\Gamma\subset\mb{R}^n$ ist
  \[\cap_{S\subset\mb{R}^n, S\ \text{konvex},\ \Gamma\subset S} S\]
  ``kleinste konvexe Menge, in der $\Gamma$ enthalten ist''
\end{Def}
\begin{Def}{konvexes Polytop}
  die konvexe Hülle von einer endlichen Menge in $\mb{R}^n$
\end{Def}
\begin{Def}{innerer Punkt}
  Das Polytop $P$ hat $O\in \mb{R}$ als inneren Punkt falls:
  \begin{itemize}
    \item $O\in\mb{R}$
    \item $\exists \varepsilon>0:B_\varepsilon(0)\subset P$
  \end{itemize}
\end{Def}
\begin{Def}
  Ist ein Polytop mit $O\in\mb{R}$ als inneren Punkt, so definieren wir
  \[P^*=\left\{ \phi\in\left( \mb{R}^n \right)^*:\phi(v)\leq 1\ \forall v\in P \right\}\]
\end{Def}
\begin{Def}{duales Polytop}
  $P^*$ ist ein konvexes Polytop $O\in\left( \mb{R}^n \right)^*$ als innerem Punkt. $P^*$ heisst duales Polytop.
\end{Def}
\begin{Bem}
  \[P^**=P\]
\end{Bem}
\begin{Bem}
  Konstruktion des dualen Polytops:
  $l$ Hyperebene, so dass $P$ auf einer Seite von $l$ liegt (inklusive $l$ selbst) $\rsa$ Gleichung von $l$ schreiben als
  \[\alpha_1x_1+\cdots+\alpha_nx_n=1\]
  \[P\subset \left\{ \left( x_1,\cdots,x_n \right)|\alpha_1x_1+\cdots+\alpha_nx_n\leq 1 \right\}\]
  $\rsa$
  \[\left( \alpha_1,\cdots,\alpha_n \right)\in P^*\]
  (Skizze (Freiwilliger gesucht \}:-) )
  \[\text{Facetten von}\ P\ \rsa\ \text{Ecken in}\ P^*\]
  $(P^*)$ konvexe Hülle
\end{Bem}
\begin{Bsp}
  \begin{itemize}
    \item Der Tetraeder ist selbstdual.
    \item Der Würfel dual zum Oktaeder.
    \item Der Dodekaeder ist dual zum Ikosaeder.
  \end{itemize}
\end{Bsp}
