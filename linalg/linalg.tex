% headers by Alexander Berthold van der Bourg / Pirmin Weigele 

%= Document-Class ==================================================================================
\documentclass[10pt,a4paper]{article}

%= Packages ========================================================================================
\usepackage[utf8]{inputenc}
\usepackage{ngerman,amsmath,amssymb,amsfonts,mathrsfs}
\usepackage{amsthm}
\usepackage{bbm}
\usepackage{ulsy}
\usepackage{colortbl}
\usepackage{graphicx}
\usepackage{pictexwd,dcpic}
%\usepackage{dcpic}
\usepackage{makeidx}
\usepackage{fancyhdr}
\usepackage{latexsym}
\usepackage{psfrag}
\usepackage{enumerate}
\usepackage{float}
%\usepackage{mathtext}
\usepackage{dsfont}
\pagestyle{fancy}
\usepackage{multirow, bigdelim, bigstrut}
\usepackage{rotating}
\usepackage{ifthen}
\usepackage{boxedminipage}
\usepackage{chemarrow,stmaryrd}
\usepackage{mathtools}

%= Seiten-Layout =========================================================================
\voffset-22mm \textheight715pt 

%Seitenbreite==============================================================

%\oddsidemargin=0.4in
%\evensidemargin=0.2in
%\textwidth=5.2in
%\headwidth=5.2in

%= Index-Befehle ========================================================================
\renewcommand{\indexname}{Stichwortverzeichnis}
\makeindex

%= Befehl-Overwriting =======================================================================
\makeatletter
\makeatother

%= Strings ================================================================
\newcommand{\mainfold}{.}

%= Eigene Befehle ==========================================================================
\DeclareMathOperator{\id}{Id}
\DeclareMathOperator{\arccot}{arccot}
\DeclareMathOperator{\arsinh}{arsinh}
\DeclareMathOperator{\arcosh}{arcosh}
\DeclareMathOperator{\artanh}{artanh}
\DeclareMathOperator{\md}{d}
\DeclareMathOperator{\Grad}{grad}
\DeclareMathOperator{\Ker}{Ker}
\DeclareMathOperator{\Span}{span}
\DeclareMathOperator{\eig}{Eig}
\DeclareMathOperator{\diag}{diag}
\DeclareMathOperator{\disc}{disc}
\DeclareMathOperator{\Hom}{Hom}
\DeclareMathOperator{\rang}{rang}

\newcommand{\Diff}[2]{\displaystyle\frac{\mathrm{d}#1}{\mathrm{d}#2}}
\newcommand{\End}{\hfill{\hbox{$\Box$}}\par\vspace{2mm}}
\newcommand{\eps}{\varepsilon}
\newcommand{\ePic}[1]{\input{\mainfold/graphics/\prefix#1.eepic}}
\newcommand{\pst}[1]{\input{\mainfold/graphics/\prefix#1.pst}}
\newcommand{\pic}[1]{\input{\mainfold/graphics/\prefix#1.pic}}
\newcommand{\Mx}[1]{\begin{pmatrix}#1\end{pmatrix}}
\newcommand{\im}[1]{\operatorname{Im}(#1)}
%\newcommand{\Include}[4]{\rhead{#2.#3.20#4}\input{\mainfold/lectures/#1-#4-#3-#2.tex}}
\newcommand{\Index}[1]{\emph{#1}\index{#1}}
\newcommand{\Int}[4]{\displaystyle\int\limits_{#1}^{#2}#3\,\mathrm{d}#4}
\newcommand{\diff}[1]{\operatorname{d}\!#1}
\newcommand{\Limi}[1]{\displaystyle\lim_{#1\rightarrow\infty}}
\newcommand{\Limo}[1]{\displaystyle\lim_{#1\rightarrow0}}
\newcommand{\mb}[1]{\mathbb{#1}}
\newcommand{\ds}{\displaystyle}
\newcommand{\ol}[1]{\overline{#1}}
\newcommand{\Part}[2]{\dfrac{\partial #1}{\partial #2}}
\newcommand{\QED}{\hfill{\hbox{(QED)}}\par\vspace{2mm}}
\newcommand{\re}[1]{\operatorname{Re}(#1)}
\newcommand{\s}{\hspace{2mm}}
\newcommand{\vsa}{\vspace{1mm} \\}
\newcommand{\vsb}{\vspace{2mm} \\}
\newcommand{\vsc}{\vspace{3mm} \\}
% \newcommand{\tr}[1]{\textrm{#1}}
\newcommand{\tr}[1]{\text{#1}}
\newcommand{\ra}{\rightarrow}
\newcommand{\Ra}{\Rightarrow}
\newcommand{\Lra}{\Leftrightarrow}
\newcommand{\lra}{\leftrightarrow}
\newcommand{\La}{\Leftarrow}
\newcommand{\ul}[1]{\underline{#1}}
\newcommand{\rsa}{\rightsquigarrow}
\newcommand{\ara}[2]{\autorightarrow{\ensuremath{#1}}{\ensuremath{#2}}} 
\newcommand{\Mat}[1]{\ensuremath{(n\times n, \mb{#1})}}
\newcommand{\dcp}[2]{\begindc{\commdiag}[#1] #2 \enddc}

%\newcommand{\detmx}{\left| \begin{array} #1 \end{array} \right|}

\newcommand{\grad}[1]{\Grad(#1)}
\newcommand{\fr}[2]{\displaystyle\frac{#1}{#2}} % fertiger bullshit, daf�r gibts \dfrac{}{}
\renewcommand{\Re}{\operatorname{Re}}
\renewcommand{\Im}{\operatorname{Im}}

% ---- DELIMITER PAIRS ----
\def\floor#1{\lfloor #1 \rfloor}
\def\ceil#1{\lceil #1 \rceil}
\def\seq#1{\langle #1 \rangle}
\def\set#1{\{ #1 \}}
\def\abs#1{\mathopen| #1 \mathclose|}	% use instead of $|x|$ 
\def\norm#1{\mathopen\| #1 \mathclose\|}% use instead of $\|x\|$ 

% --- Self-scaling delmiter pairs ---
\def\Floor#1{\left\lfloor #1 \right\rfloor}
\def\Ceil#1{\left\lceil #1 \right\rceil}
\def\Seq#1{\left\langle #1 \right\rangle}
\def\Set#1{\left\{ #1 \right\}}
\def\Abs#1{\left| #1 \right|}
\def\Norm#1{\left\| #1 \right\|}

%Adrians Abbildungs-Environment ==============================================

\newcommand{\Sidein}{\begin{rotate}{90}\small$\in$\end{rotate}}

\newcommand{\Abb}[5][]{\ensuremath{
    \begin{array}{lc}
      \ifthenelse{\equal{#1}{}}{}{#1:}\;\; & 
      \begin{xy}
        \xymatrixrowsep{1em}\xymatrixcolsep{2em}%
        \xymatrix{ #2 \ar[r] \ar@{}[d]^<<<<{\hspace{0.001em} \Sidein}
          & #3  \ar@{}[d]^<<<<{\hspace{0.001em} \Sidein} \\
          #4 \ar@{|->}[r] & #5} \end{xy}
    \end{array}
  }%
}

%= Environments ========================================================================
\def\thechapter{\Roman{chapter}}
\def\thesection{\arabic{section}}
\newtheorem{Bew}{Beweis}
\newtheorem{Lem}{Lemma}
\newtheorem{Kor}{Korollar}
\newtheorem{Sat}{Satz}
\newtheorem{Prop}{Proposition}
\theoremstyle{definition}
\newtheorem{Bsp}{Beispiel}
\newtheorem{Def}{Definition}
\newtheorem{Prob}{Problem}
\theoremstyle{remark}
\newtheorem{Bem}{Bemerkung}
\newtheorem{Eig}{Eigenschaften}
\newtheorem{Faz}{Fazit}
\newtheorem{Not}{Not}

\def\pstexInput#1{%
  \begin{center}
    \begin{picture}(0,0)%
      \special{psfile=\mainfold/graphics/A2-#1.pstex}%
    \end{picture}%
    \input{\mainfold/graphics/A2-#1.pstex_t}%
  \end{center}
}

%= Titelseite ===========================================================================
\begin{document}
\headheight15pt
\begin{titlepage}
\hfill
\vspace{20mm}
\pagenumbering{roman}
\begin{center}
{\LARGE Lineare Algebra I - Vorlesungs-Script} \vskip 3em {\large Prof. Andrew Kresch} \vskip 1.5em
{\large Basisjahr 08/09 Semester II}\vspace{30mm}\\
{\large {\bf Mitschrift:} \vspace{2mm}\\
Simon Hafner}\vspace{5mm}\\ %30mm
%{\large {\bf Graphics:} \vspace{2mm}\\
%Pirmin Weigele }\vspace{30mm}\\ %30mm
\end{center}
\vfill

\end{titlepage}


%= Inhaltsverzeichnis ==========================================================================
\lhead{}
\rhead{}
\tableofcontents
\newpage
\pagenumbering{arabic}
\setcounter{page}{1}

%= Vorlesung-Skripts ==========================================================================
\cfoot{\thepage}
\fancyhead[L]{\nouppercase{\leftmark}}
\newpage

%= LinAlg I & & II ==========================================================================

%LinAlg I
\input{09-02-16}
\begin{Faz}
  \begin{align*}
    \Seq{.,.}& \mb{R}^n\times\mb{R}^n\to\mb{R}^n&\text{bilinear form}\\
    \Norm{.}& \mb{R}^n\to \mb{R}_{\geq 0}& \text{Norm}\\
    d(.,.)&\mb{R}^n\times\mb{R}^n\to\mb{R}_{>0}&\text{Metrik}
  \end{align*}
  \begin{align*}
    \Norm{x}&=\sqrt{\Seq{x,x}}\\
    d(x,y)&=\Norm{y-x}\\
    \Seq{x,y}&=\frac{\Norm{x}^2+\Norm{y}^2-\Norm{y-x}^2}{2}
  \end{align*}
\end{Faz}
\subsection{Vektorprodukt in $\mb{R}^3$}
  \begin{align*}
    \mb{R}^3\times\mb{R}^3&\to&\mb{R}^3\\
    (x,y)&\mapsto&x\times y
  \end{align*}
  für $y=(y_1,y_2,y_3)$ und $y=(y_1,y_2,y_3)$ ist 
  \[x\times y=(x_2y_3-x_3y_2, x_3y_1-x_1y_2,x_1y_2-x_2y_1)\]
  oder:
  \[x\times y=\det\Mx{e_1&e_2&e_3\\x_1&x_2&x_3\\y_1&y_2&y_3}\]
  wobei $(e_1,e_2,e_3)$ die Standardbasis ist. Es ist deshalb klar, dass
  \[0=\det\Mx{x_1&x_2&x_3\\x_1&x_2&x_3\\y_1&y_2&y_3}=\Seq{x,x\times y}\]
  \[0=\det\Mx{y_1&y_2&y_3\\x_1&x_2&x_3\\y_1&y_2&y_3}=\Seq{y,x\times y}\]
  $x\times y$ liegt auf der Gerade von Vektoren senkrecht zu $x$ und $y$.
  weiter:
  \[\det\Mx{w_1&w_2&w_3\\x_1&x_2&x_3\\y_1&y_2&y_3}=\Seq{x\times y,x\times y}\]
  \[=\Norm{x\times y}^2=(x_2y_3-x_3y_2)^2+(x_3y_1-x_1y_3)^2+(x_1y_2-x_2y_1)^2\]
  \[=\Norm{x}^2\Norm{y}^2-\Seq{x,y}^2=\Norm{x}^2\Norm{y}^2\left( 1-\frac{\Seq{x,y}^2}{\Norm{x}^2\Norm{y}^2} \right)\]
  \[=\Norm{x}^2\Norm{y}^2(1-\cos^2\angle (x,y) = \Norm{x}^2\Norm{y}^2\sin^2\angle (x,y)\]
\begin{Faz}
  Wenn das Ergebnis $=0$, folgt daraus, dass $x$ und $y$ linear abhängig sind. Falls $x$ und $y$ linear unabhängig sind, dann folgt dass $(x\times y,x,y)$ zu derselben Orientierungsklasse gehört wie $(e_1,e_2,e_3)$. Insgesamt bedeutet dies, dass $x\times y$ folgende Eigenschaften hat:
  \begin{itemize}
    \item ist senkrecht zu $x$ und $y$
    \item ist 0 $\Lra$ $x$ und $y$ sind linear abhängig
    \item hat Länge $\Norm{x}\Norm{y}\sin \angle (x,y)$
    \item und hat die Richtung, die mit $x$ und $y$ die gleiche Orientierungsklassse wie die Standardbasis hat.
  \end{itemize}
\end{Faz}
\subsection{Skalarprodukt über $\mb{C}^n$}
Sei $z=(z_1,\cdots,z_n)$ und $w=(w_1,\cdots,w_n)\in\mb{C}^n$
\begin{Bem}
  Der Ausdruck macht Sinn.
  \begin{align*}
    \Seq{z,w}&:=z_1w_1+\cdots+z_nw_n\\
    \Seq{z,z}&:=z_1^2+\cdots+z_n^2\\
  \end{align*}
  Dann kann die Länge nicht mehr interpretiert werden, z.B. für $z=(1,i,0,\cdots,0)$ haben wir $\Seq{z,z}=1^2+i^2=0$. Isotropische Untervektorräume von $\mb{C}^n$ werden nicht in in diesem Kurs behandelt. ($V\subset\mb{C}^n$ s.d. $\Seq{v,w}=0 \ \forall v,w\in V$). Für die Physik, die Geometrie usw. ist eine Interpretation in Zusammenhang mit Länge wichtig, deshalb brauchen wir eine neue Definition.
\end{Bem}
\begin{Def}[Das kanonische Skalarprodukt]
  von $\mb{C}^n$ ist gegeben durch
  \begin{align*}
    \Seq{.,.}_c:&\ \mb{C}^n\mb{C}^n\to\mb{C}\\
    &\ (z,w)\mapsto z_1\bar{w_1}+\cdots+z_n\bar{w_n}
  \end{align*}
\end{Def}
\begin{Eig}[von $\Seq{.,}_c$]
  \begin{align*}
    \Seq{z+z',w}&= \Seq{z,w}_c+\Seq{z',w}_c\\
    \Seq{\lambda z,w}_c&= \lambda\Seq{z,w}_c\\
    \Seq{z,w+w'}_c&= \Seq{z,w}_c+\Seq{z,w'}_c\\
    \Seq{z,\lambda w}_c&= \bar{\lambda}\Seq{z,w}_c
  \end{align*}
  für $z,z',w,w'\in\mb{C}^n$, $\lambda\in\mb{C}$\\
  $\Seq{.,.}_c$ ist sesquilinear
  \begin{align*}
    \Seq{w,z}_c&= \overline{\Seq{z,w}_c}& \text{hermitesch}\\
    \Seq{z,z}_c&\in\mb{R}_{\geq 0}& \text{positiv definit}\\
    \Seq{z,z}=0 &\Lra z=0
  \end{align*}
\end{Eig}
\begin{Faz}
  $\Seq{.,.}_c$ ist sesquilinear, hermitesch und positiv definit.
\end{Faz}
\begin{Bew}
  Bei Bedarf sonstwo nachschauen (Zu viele Zeichen und zu wenig Sinn). Es läuft auf eine Sammlung von Quadraten heraus.
\end{Bew}
\begin{Def}[Norm von $\mb{C}^n$]
  \[\Norm{z}=\sqrt{\Seq{z,z}_c}\]
\end{Def}
\begin{Bem}
  Sei $w=(x_1'+xy_1',\cdots,x_n'+iy_n')$. Dann:
  \begin{align*}
    \Seq{z,w}_c=(x_1+iy_1)(x_1'-iy_1')+\cdots+(x_n+iy_n)(x_n'-iy_n')\\
    =(x_1x_1'+y_1y_1'+\cdots+x_nx_n'+y_ny_n')+i(x_1'y_1-x_1y_1'+\cdots+x_n'x_y-x_ny_n')
  \end{align*}
  Auf diese Weise ist $\Seq{.,.}_c$ eine Erweiterung von reellen Skalarprodukt.
  \begin{align*}
    \mb{R}^{2n}&\xrightarrow{~}\mb{C}^n&\mb{R}\text{-linear}\\
    e_1\mapsto&(1,0,\cdots,0)\\
    e_2\mapsto&(i,0,\cdots,0\\
    \cdots\\
    e_{2n}&\mapsto(0,\cdots,0,i)
  \end{align*}
  \[\Seq{.,.}_c=\left( \Seq{.,.} \text{von}\ \mb{R}^{2n} \right) + i(\text{neues})\]
  $\Re\Seq{.,.}_c=\Seq{.,.}$ von $\mb{R}^{2n}$ unter diesem Isomorpismus.\\
  Sei $\omega:=\Im\Seq{.,.}$:
  \begin{align*}
    \omega:&\mb{C}^n\times \mb{C}^n\to\mb{R}\\
    \text{oder}\ & \mb{R}^{2n}\times\mb{R}^{2n}\to\mb{R}
  \end{align*}
\end{Bem}
\begin{Eig}[von $\omega$ (Imaginärteil des kanonischen Skalarproduktes)]
  \begin{description}
    \item[bilinear]
    \item[schiefsymmetrisch] $\omega(w,z)=-\omega(z,w)$
    \item[] $\omega(z,z)=0\ \forall z\in\mb{C}^n$ (oder $\mb{R}^{2n}$)
  \end{description}
\end{Eig}
\subsection{Bilinearform}
Sei $K$ ein Körper und $V$ ein $K$-Vektorraum.
\begin{Def}[Bilinearform]
  Eine bilineare Form auf $V$ ist eine Abbildung
  \[s:V\times V\to K\]
  so dass:
  \begin{align*}
    s(v+v',w)&=s(v,w)+s(v',w)\\
    s(\lambda v,w)&=\lambda s(v,w)\\
    s(v,w+w')&=s(v,w)+s(v,w')\\
    s(v,\lambda w)&= \lambda s(v,w)
  \end{align*}
  $\forall v,v',w,w'\in V, \lambda \in K$\\
  Und: $s$ heisst \underline{symmetrisch}, falls $s(w,v)=s(v,w)$ und \underline{schiefsymmetrisch}, falls $s(w,v)=-s(v,w)$.
\end{Def}
\begin{Bsp}
  \begin{itemize}
    \item $\Seq{.,.}:=\mb{R}^n\times \mb{R}^n\to\mb{R}$ ist eine symmetrische bilineare Form
    \item $\omega$ ist eine schiefsymmetrisch bilineare Form
    \item ($\Seq{.,.}_c$ nicht)
    \item $V=\{\text{stetige Abbildung} [0,1] \to\mb{R}\}$ über $\mb{R}$: $f,g\in V$
      \[s(f,g)=\int^1_0 f(x)g(x)dx\]
      ist eine symmetrisch bilineare Form auf $V$
  \end{itemize}
\end{Bsp}

\input{09-02-23}
\begin{Def}{hermitesch}
  Eine sesquilineare Form heisst hermitesch, falls
  \[s(w,v)=s(v,w)\ \forall v,w\in V\]
\end{Def}
\begin{Bsp}
  $\Seq{.,.}_c$ auf $\mb{C}^n$ ist hermitesch.
\end{Bsp}
\begin{Bem}{hermitesche Form}
  Man spricht von hermiteschen Form, diese sind immer sesquilinear
\end{Bem}
\begin{Def}{darstellende Matrix}
  Sei $\dim_\mb{C} V < \infty$, und $B:=(V_i)_{1\leq i \leq n}$ eine Basis. Ist $s$ eine sesquilineare Form, so definieren wir
  \[M_B(s):=\left( s(v_i,v_j) \right)_{1\leq i \leq n}\]
  die darstellende Matrix. Sind $z,w\in V$
  \begin{align*}
    z&= z_1v_1+\cdots+z_nv_n\\
    w&= w_1v_1+\cdots+w_nv_n
  \end{align*}
  dann haben wir
  \begin{align*}
    s(z,w)&=\sum^n_{i,j=1}z_i\bar{w_j}a_{ij} \text{wobei} a_{ij} = s(v_i,v_j)\\
    &= (z_1\cdots z_n)\Mx{a_{11}&\cdots&a_{1n}\\ \vdots & & \vdots \\ a_{n1} & \cdots & a_{nn}} \Mx{\bar{w_1} \\ \vdots \\ \bar{w_n}}
    &=  z^t M_B(s)\cdot \bar{w}
  \end{align*}
\end{Def}
\begin{Prop}
  Sei $V$ ein endlich dim. $\mb{C}$ Vektorraum und $B=(v_i)_{1\leq i \leq n}$. Wir haben eine Bijektion
  \[\{\text{sesquilineare Form auf $V$}\}\ \lra\ M(n\times n, \mb{C})\]
  Unter dieser Bijektion haben wir:
  \[\{\text{hermitesche Formen}\}\ \lra\ \{A\in M(n\times n, \mb{C}): A^t=\bar{A}\}\]
  Man sagt: eine Matrix $A\in M(n\times n, \mb{C})$ mit $A^t=\bar{A}$ ist \underline{hermitesch}.
\end{Prop}
\begin{Sat}{Transformationsformel}
  Sei $A=(u_1,\cdots,u_n)$ eine andere Basis mit Transformationsmatrix $T$:
  \[\text{TODO: hier einfügen}\]
  Dann gilt: 
  \[M_A(s)=T^t\cdot M_B(s)\cdot \bar{T}\]
  Mit $g(v):=s(v,v)$ gilt die Polarisierungsformel
  \begin{align*}
    s(v,w)=\frac{1}{4}\left( q(v+w)-q(v-w)+iq(v+iw)-iq(v-iw) \right)
  \end{align*}
\end{Sat}
\begin{Def}{positiv definit}
  Sei $K=\mb{R}$ oder $\mb{C}$, $V$ ein $K$-Vektorraum und $s:V\times V\to K$ eine Billinearform $\begin{cases}
    \text{symmetrisch} & K= \mb{R} \\
    \text{hermitesch} & K=\mb{C}
  \end{cases}$
  heisst positiv definit, falls $s(v;v)>0$ $\forall 0 \neq v\in V$
\end{Def}
\begin{Bsp}
  $\Seq{.,.}$ ist positiv definit auf $\mb{R}^n$\\
  $\Seq{.,.}_c$ ist psoitiv definit auf $\mb{C}^n$
\end{Bsp}
\begin{Def}{Skalarprodukt}
  \\Ein Skalarprodukt ist $ \begin{cases}
    \text{positiv definite symetrische bilineare Form} & K=\mb{R}\\
    \text{eine positiv definite hermetische Form} & K=\mb{C}
  \end{cases}$ 
\end{Def}
\begin{Def}
  Skalarprodukt oft $\Seq{.,.}$, Norm $\Norm{v} := \sqrt{\Seq{v,v}}$
\end{Def}
\begin{Def}{Euklidischer Vektorraum}
  Vektorraum über $\mb{R}$ mit Skalarprodukt
\end{Def}
\begin{Def}{Untärer Vektorraum}
  Vektorraum über $\mb{C}$ mit Skalarprodukt
\end{Def}
\begin{Bsp}
  \begin{align*}
    V=\left\{ f:[0,1]\to\mb{R}\ \text{stetig} \right\} \text{mit} \Seq{f,g}=\int_0^1f(x)g(x)dx\\
    V=\left\{ f:[0,1]\to\mb{C}\ \text{stetig} \right\} \text{mit} \Seq{f,g}=\int_0^1f(x)\overline{g(x)}dx
  \end{align*}
  in beiden Fällen
  \[\Norm{f}=\sqrt{\int_0^1\Abs{f(x)}^2dx}\]
  ``$L^2$-Norm''
\end{Bsp}
\begin{Bem}
  In einem beliebigen euklidischen bzw. unitären Vektorraum gilt die Cauchy-Schwarz'sche Ungleichtung
  \[\Abs{\Seq{v,w}}\leq \Norm{v}\Norm{w}\ \forall v,w\in V\]
  mit $=$ genau dann, wenn $v$ und $w$ linear abhängig sind.
\end{Bem}
\begin{Bew}
  (Skizze) klar falls $v=0$ oder $w=0$, also nehmen wir an, dass $v\neq 0$ und $w\neq 0$
  1. Reduktion: zum Fall $\Norm{v}=\Norm{w}=1$.
  \begin{align*}
    v_1:=\frac{v}{\Norm{v}} & w_1:=\frac{w}{\Norm{w}}\\
    \Norm{v_1}=1 & \Norm{w_1}=1
  \end{align*}
  2. Reduktion:  Es reicht aus, zu zeigen: $\Re \Seq{v,w}\leq 1$ = genau dann wenn $V=W$
  \begin{align*}
    \Abs{\Seq{v,w}}&= \mu\Seq{v,w} & \mu\in \mb{C}, \Abs{\mu}=1 \\
    &= \Seq{\mu v,w}\in \mb{R}_\geq \\
    &= \Re \Seq{v',w} \text{wobei} v':=\mu v
  \end{align*}
  Cauchy-Schwarz'sche Ungleichung $\leq$, Gleichheit: $v,w$ linear unabhängig $\implies$ $v',w$ linear unabhängig $\implies$ $v'\neq w$
\end{Bew}
\begin{Eig}
  \begin{align*}
     && = \text{falls} \\
    \Seq{v-w,v-w} \geq 0 && v-w=0\\
    \Seq{v,v} - \Seq{v,w} - \Seq{w,v} + \Seq{w,w} \geq 0& & v=w\\
    1-\Seq{v,v}-\overline{\Seq{v,w}} + 1 \geq 0 && v=w
  \end{align*}
\end{Eig}
\begin{Bsp}
  Ist $T:V\to\mb{R}^n$ oder $T:V\to\mb{C}^n$ ein Isomorphismus, dann ist $s:V\times V\to\mb{R}$ (bzw. $s:V\times V\to\mb{C}$) gegeben durch
  \[s(x,y)=\Seq{T_x,T_y}\]
  bzw.
  \[s(x,y)=\Seq{T_x,T_y}_c\]
  ein Skalarprodukt.
\end{Bsp}
\begin{Def}
  Sei $V$ ein exklusiver, bzw. unitärer Vektorraum
  \begin{itemize}
    \item $v,w\in V$ heisst orthogonal, falls $\Seq{v,w}=0$
    \item $U,W\subset V$ heissen orthogonal (geschrieben $U\perp V$) falls $U\perp W$ $\forall u\in U$, $w\in W$
    \item $U\subset W$ das orthagonale Koplement ist $U^\perp=\left\{ v\in V: u\perp v \forall u\in U \right\}$
    \item $v_1,\cdots, v_n$ sind orthogonal, falls $v_i\perp v_j$ $\forall i\neq j$
    \item $v_1,\cdots, v_n$ sind orthonormal, falls $v_i\perp v_j$ $\forall i\neq j$ und $\Norm{v_i}=1$ $\forall i$
    \item $V$ ist orthagonale direkte Summe von Untervektorräumen $V_1,\cdots,V_r$ falls 
      \begin{align*}
        V=V_1\oplus \cdots \oplus V_r\\
        V_i\perp V_j\ \forall i\neq j
      \end{align*}
  \end{itemize}
  \[C([-1,1],\mb{R}):=\left\{ f:[-1;1]\to \mb{R} \text{stetig} \right\}\]
  dann ist $C([-1,1]\mb{R})$ die orthogonale direkte Summe von $C([-1,1]\mb{R})_\text{gerade}$ und $C([-1,1]\mb{R})_\text{ungerade}$. gerade: $f(-x)=f(x)$ und ungerade: $f(-x)=-f(x)$ % stimmt das so? TODO
  \[f(x)=\underbrace{\frac{f(x)+f(-x)}{2}}_\text{gerader Teil} + \underbrace{\frac{f(x)-f(-x)}{2}}_\text{ungerader Teil}\]
  $g$ gerade, $h$ ungerade $\implies$ $gh$ ungerade $\implies$ $\Seq{g,h} = \int_{-1}^1g(x)h(x)=0$
\end{Def}

\input{09-03-02}
\input{09-03-05}
\input{09-03-09}
\input{09-03-12}
\subsection{Selbstadjugierte Endomorphismen}
$V$,$\left\langle , \right\rangle$, $K$-Vektorraum mit Skalaprodukt. ($K$=$\mb{R}$ oder $\mb{C}$). Ist $F:V\to V$ ein Endomorphismus, so heisst $F^*:V\to V$ \underline{adjugierter Endomorpismus} falls 
\[\left\langle F(v),w \right\rangle = \left\langle v,F^*(w) \right\rangle\ \forall v,w\in V\]
\begin{Def}
  $F:V\to V$ ist \underline{adjugiert} falls
  \[\left\langle F(v),w \right\rangle=\left\langle v,F(w) \right\rangle\ \forall v,w\in V\]
\end{Def}
\begin{Eig}
  Falls $V=\mb{R}^n$ mit Standardskalarprodukt, so zu $F$ ist eine assoziierte Matrix $A\in M(n\times n,\mb{R})$, dann ist $A^t$ zu $F^*$ assoziiert.
  Falls $V=\mb{C}^n$, dann ist 
  \[F\lra A\in M(n\times n,\mb{C})\]
  \[F^*\lra \bar{A}^t\in M(n\times n,\mb{C})\]
\end{Eig}
\begin{Bew}
  \[\left\langle Av,w \right\rangle=(Av)^t\bar{w}=v^tA^t\bar{w}=v^t\bar{A}^tw=\left\langle v,\bar{w}^tw \right\rangle\]
\end{Bew}
\begin{Bem}
  $F^*$ ist eindeutig falls für $\tilde F^*$ gilt
  \[\left\langle F(v),w \right\rangle=\left\langle v,\tilde F^*(w) \right\rangle\]
  dann ist
  \[0=\left\langle v,\tilde F^*(w)-F^*(w) \right\rangle\]
  $\implies$
  \begin{align*}
    0=\left\langle \tilde F^*(w)-F^*(w),\tilde F^*(w)-F^*(w) \right\rangle\\
    =\Norm{\tilde F^*(w)-F^*(w)}^2\\
    \implies \tilde F^*(w)=F^*(w)
  \end{align*}
\end{Bem}
\begin{Faz}
  Im Fall $V=\mb{R}^n$ bzw. $\mb{C}^n$ mit Standardskalarprodukt ist ein selbstadjungierter Endomorphismus durch eine symmetrische bzw. hermitesche Matrix gegeben.
\end{Faz}
\begin{Lem}
  Jeder Eigenwert eines selbstadjugierten Endomorphismus ist reell.
\end{Lem}
\begin{Bew}
  Ist $F(v)=\lambda v$ mit $v\neq 0$, so gilt
  \[\lambda\left\langle v,v \right\rangle=\left\langle \lambda v,v \right\rangle=\left\langle F(v),v \right\rangle=\left\langle v,F(v) \right\rangle=\left\langle v,\lambda v \right\rangle=\bar \lambda\left\langle v,v \right\rangle \implies \lambda=\bar\lambda \]
\end{Bew}
\begin{Bem}{Prä-Hilbertraum}
  bezeichnet einen $K$-Vektorraum ($K$=$\mb{R}$ oder $\mb{C}$) mit Skalarprodukt. Euklidische bzw. unitäre Vektorräume sind endlichdimensional,
\end{Bem}
\begin{Prop}
  Sei $V$ ein euklidischer bzw. unitärer Vektorraum und $F:V\to V$ ein selbstadjugierter Endomorphismus. Dann gibt es eine orthonormale Basis von Eigenvektoren.
\end{Prop}
\begin{Bew}
  Falls $V$ ein unitärer Vektorraum ist: durch Induktion nach $\dim V$, $\exists$ Eigenwert $\lambda$, Eigenvektor $v$, oBdA haben wir $\Norm{v}=1$. Wir behaupten:
  \[F(v^\perp)\in V^\perp\]
  \[\left\langle v,w \right\rangle =0\implies \left\langle v,F(w) \right\rangle =\left\langle F(v),w \right\rangle =\left\langle \lambda v,w \right\rangle =\lambda\left\langle v,w \right\rangle =0\]
  IA$\implies$ $\exists$ orthonormale Basis von $v^\perp$. Dies, zusammen mit $v$, gibt eine Basis von $V$. Fall eines euklidischen Vektorraums: Das gleiche Argumente ist gültig, sobald wir wissen, dass $F$ einen Eigenwert besitzt. Man wählt eine Basis von $V$, so:
  \[F\lra A\in M(n\times n,\mb{R})\ \ [n=\dim V]\]
  mit $A=A^t$. Wir betrachten $A$ als komplexe Matrix, so dass 
  \[A=\bar A\implies \bar A^t=A^t=A \implies A\ \text{ist hermetisch}\]
  Sei $\lambda$ ein (komplexer) Eigenwert von $A$. Weil $A$ hermetisch ist, haben wir $\lambda\in \mb{R}$. Wir haben 
  \[\det(A-\lambda E_n)=0\]
  Dann:
  \[\det(F-\lambda id_V)=0\]
  also $\lambda$ ist Eigenwert von $F$.
\end{Bew}
\begin{Kor}
  Sei $A\in M(n\times n,\mb{R})$ symmetrisch. Dann $\exists$ $S\in O(n)$ mit
  \[S^tAS=\diag (\lambda_1,\cdots,\lambda_n),\ \lambda_1,\cdots,\lambda_n\in\mb{R}\]
  Sei $A\in M(n\times n,\mb{C})$ hermetisch. Dann $\exists$ $S\in U(n)$ mit
  \[\bar S^tAS=\diag (\lambda_1,\cdots,\lambda_n),\ \lambda_1,\cdots,\lambda_n\in\mb{R}\]
\end{Kor}
\begin{Kor}
  Sei $F:V\to V$ wie in der Proposition oben. Dann ist $V$ die orthogonale direkte Summe von diesen Eigenräumen:
  \[V=\bigoplus_{\text{Eigenwerte} \lambda}\eig(F;\lambda)\]
\end{Kor}
\begin{Faz}
  $\rsa$ Praktisches Verfahren: $A$ symmetrisch bzw. hermetische Matrix\\
  $\hookrightarrow$ berechnen $\eig(A;\lambda)$\\
  $\hookrightarrow$ wählen von jedem eine orthonormale Basis
\end{Faz}
\begin{Bsp}
  \begin{align*}
    A=\Mx{5&3&3+3i\\3&5&-3-3i\\3-3i&-3+3i&2}\\
    P_A(t)=\det(tE_3-A)=(t-5)^2(t-2)+\cdots = t^3-12t^2+256=(t+4)(t-8)^2\\
    \eig(A;-4)=\Ker\Mx{9&3&3+3i\\3&9&-3-3i\\3-3i&-3+3i&6}=\Span \left\{ \Mx{1\\-1\\-1+i} \right\} \\
    \eig(A;\delta)=\Ker\Mx{-3&3&3+3i\\3&-3&-3+3i\\3-3i&-3+3i&-6}=\Span\left\{ \Mx{1\\1\\0} ,\Mx{2\\0\\1-i} \right\} \rsa \Mx{1\\1\\0},\Mx{1\\-1\\1-i}
  \end{align*}
  bzw.
  \[\Mx{\frac{\sqrt{2}}{2}&\frac{\sqrt{2}}{2}&0},\Mx{\frac{1}{2}&-\frac{1}{2}&\frac{1-i}{2}}\]
  Wir bekommen:
  \[S:=\Mx{\frac{1}{2}&\frac{\sqrt{2}}{2}&\frac{1}{2}\\ -\frac{1}{2}&\frac{\sqrt{2}}{2}&-\frac{1}{2}\\ \frac{-1+i}{2}&0&\frac{1-i}{2}}\]
  dann:
  \[\bar S^tAS=\diag(-4,8,8)\]
\end{Bsp}
\begin{Bem}
  Das Resultat von der Proposition oben im Fall eines euklidischen Vektorraums ist klar, auch aus geometrischem Grund.
  \[\text{symm. Matrizen}\setminus \mb{R}\ \rsa\ \text{quadratische Formen}\]
  (Prop aktuelle-7) $S^tAS$ aus der Transformationsformel.\\
  \ldots und man kann auch einen alternativen Beweis in diesem Fall geben.
  \[A\in M(n\times n,\mb{R}), A^t=A\ \rsa\ q:\mb{R}^n\to\mb{R},q(v):=v^tAv\]
  (Faktum aus der Analysis)
  \[\exists x\in\mb{R}^n, \Norm{x}=1\ \text{mit} q(x)\geq g(x')\ \forall x'\in\mb{R}n^n, \Norm{x'}=1\]
  Dann für $v\in\mb{R}^n, v\perp x$ haben wir $Av\perp x$. In der Tat haben wir 
  \[(Av-q(x)v)\perp x\ \forall v\in\mb{R}^n\]
  denn
  \[\left\langle Av-q(x)v, v\right\rangle +2\lambda\left\langle Av-q(x)v,x \right\rangle =(v+\lambda x)^t(A-q(x)E_n)(v+\lambda x)\leq 0\ \forall \lambda\in\mb{R}\]
  (Details im Buch, 5.6.4)
\end{Bem}

\input{09-03-19}
\input{09-03-23}
\input{09-03-26}
\input{09-03-30}
\section{Multilineare Algebra}
\begin{Def}{Dualvektorraum / Linearformen}
  Sei $\mb{K}$ ein Körper und $V$ ein $\mb{K}$-Vektorraum. Der Dualvektorraum ist $V^*:=\Hom(V,\mb{K})$. Elemente vo $V^*$ heissen Linearformen. $V^*$ ist ein $\mb{K}$-Vektorraum, mit Addition von Abbildungen und Multiplikation durch Skalare.
\end{Def}
\begin{Eig}
  Sei $B=(v_i)_{i\in I}$ eine Basis von $V$.
  \begin{itemize}
    \item Koeffizient von $v_i$
      \[v_i^*:v=\sum_{j\in I}a_jv_j\mapsto a_i\]
    \item Summe von Koeffizienten
      \[\sum v_i^*:v=\sum_{j\in I}a_jv_j\mapsto \sum a_i\]
      wohldefiniert, weil nur endlich viele $a_j$ sind $\neq 0$
    \item Operationen auf Funktionsräumen, z.B. $\left\{ f:\mb{R}\to \mb{R}\ \text{stetig} \right\}$
    \item Standardkoordinaten: $n\in\mb{N}_{>0}$, $V=\mb{K}^n$, Standardbasis $e_1,\cdots,e_n$
      \[e_i^*:(x_1,\cdots,x_n)\mapsto x_i\]
  \end{itemize}
\end{Eig}
\begin{Bem}
  Ist $B=(v_1,\cdots,v_n)$ eine Basis von $V$, so ist $B^*=(v_1^*,\cdots,v_n^*)$ eine Basis von $V^*$. Denn zu $f:V\to \mb{K}$ haben wir $c_i:=f(v_i)$, dann \[f\ \text{linear} \implies f(\sum_{i=1}^na_iv_i)=\sum^n_{i=1}c_ia_i\]
  Das zeigt, dass $V^*$ ist von $v_1^*,\cdots,v_n^*$ aufgespannt. Lineare Unabhängigkeit von $v_1^*,\cdots,v_n^*$ ist klar. Deshalb haben wir einen Isomorphismus $V\to V^*$, gegeben durch $v_i\mapsto v_i^*\ \forall i$. Falls $\dim V=\infty$ mit Basis $(v_i)_{i\in I}$, dann ist $V^*$ nicht von den $v_i^*, i\in I$ aufgespannt, z.B.
  \[\sum_{i\in I}\not\in\Span(v_i^*)_{i\in I}\]
  $\phi:V\to\mb{K}$ mit $\phi(v_i)\neq 0$ nur für endlich viele $i\in I$
\end{Bem}
\begin{Bsp}
  $V=\mb{K}$ mit Standardbasis $(e_1,\cdots,e_n)$. Dann hat $V^*$ die Standardbasis $(e_1^*,\cdots,e_n^*)$ und wir haben den Isomorphismus
  \begin{align*}
    &\mb{K}^n\to(\mb{K})^*\\
    &e_i\mapsto e^*_i\ \forall i
  \end{align*}
\end{Bsp}
\begin{Bem}
  Es ist nicht überraschend, dass der Isomorphismus $V\to V^*$ assoziert zu einer Basis $B=(v_1,\cdots,v_n)$ abhängig von der Basis ist.
\end{Bem}
\begin{Bem}
  Sei $V\subset\mb{K}^n$ ein Untervektorraum. $V$ kann durch eine Basis gegeben werden, oder durch Gleichungen.
  \[V=\Span\left( \Mx{1\\1\\0},\Mx{0\\1\\1} \right)=\left\{ \Mx{x\\y\\z}\Big| x-y+z=0 \right\}\]
  ist eine Linearform auf $\mb{K}^n$
\end{Bem}
\begin{Def}{orthogonaler Raum}
  Sei $W$ ein $\mb{K}$-Vektorraum und $V\subset W$ ein Untervektorraum. Der Untervektorraum
  \[V^0=\left\{ \phi\in W^*:\phi(v)=0\ \forall v\in V \right\}\subset W^*\]
  heisst der zu $V$ orthogonale Raum. Falls $\dim V<\infty$, dann haben wir $\dim V^0=\dim W - \dim V$. Basis von \[W^* =\overbrace{\underbrace{W_1,\cdots,W_d}_{\text{von }V},\cdots,W_n}^{W}\]
  Dann:
  \[V^0=\Span\left( w_{d+1}^*,\cdots,w_n^* \right)\]
\end{Def}
\begin{Def}{duale Abbildung}
  Sei $V\to W$ eine lineare Abbildung von $\mb{K}$-Vektorräumen. Dann gibt es eine lineare Abbildung $F^*:W^*\to V^*$, die duale Abbildung, gegeben durch Komposition mit $F$
  \[\psi:V\to K\mapsto F^*(\psi):=\psi \circ F\]
  Dann
  \[V^0=\ker\left( W^*\to V^* \right)\]
  Aus der Dimensionsformel bekommt man nochmals
  \[\dim V^0=\dim W^*-\dim V^*\]
\end{Def}
\begin{Eig}{duale Abbildung}
  \begin{itemize}
    \item falls $W=V$, gilt $\left( id_V \right)^*=\id_{V^*}$
    \item Ist auch $G:U\to V$ gegeben, so haben wir \[G^*F^*\psi=\left( F\circ G \right)^*\psi\]
  \end{itemize}
  Das nennt man Funktorialität.
\end{Eig}
\begin{Bem}
  Man kann zeigen, dass zu $U\subset V$ bekommt man eine surjektive duale Abbildung $V^*\to U^*$ \[\left( \psi:V\to \mb{K} \right)\mapsto \psi|_U\]
\end{Bem}
\begin{Prop}
  Seien $V$ und $W$ endlich dimensionale $\mb{K}$-Vektorräume mit Basen $A=(v_1,\cdots,v_n)$ und $B=(w_1,\cdots,w_m)$. Sei $F:V\to W$ eine lineare Abbildung mit darstellender Matrix $M$. Dann ist $F^*:W^*\to V^*$ bezüglich der dualen Basen $A^*=(v_1^*,\cdots,v_n^*), B^*=(w^*_1,\cdots,w_m^*)$ durch die Matrix $M^t$ dargestellt. Wir schreiben $M=(a_{ij})$. Das bedeutet:
  \[F(v_j)=\sum_{i=1}^ma_{ij}w_i\]
  Es folgt 
  \[F^*(w_i^*)(v_j)=\ \text{$i$-te Komonent von } F(v_j)=a_{ij}\]
  Das ist zu sagen, die darstellende Matrix von $F^*$ ist die Matrix $(a_{ji})$
  \[F^*(w_i^*)=\sum^m_{j=1}a_{ij}v^*_j\]
\end{Prop}

\begin{Eig}{$F:V\to W$}
  \begin{itemize}
    \item \[\underbrace{\left( \Im F \right)^0}_{\text{alle}\ W\xrightarrow{\phi}\mb{K}\ \text{mit}\ \phi|_{\Im F}=0}=\underbrace{\Ker F^*}_{\text{alle}\ W\xrightarrow{\phi}\mb{K}\ \text{mit}\ \phi\circ F=0}\]
      Da \[\phi|_{\Im F}=0\ \Lra\ \phi \circ F = 0\]
      haben wir die Gleichung.
    \item \[\left( \Ker \right)^0 = \Im\left( F^* \right)\]
      $\supset$ offensichtlich\\
      $\subset$ folgt aus der Surjektivität von $W^*\to \left( \Im F \right)^*$
  \end{itemize}
  Wir betrachten $\phi:V\to \mb{K}$ mit $\phi|_{\Ker F}=0$
  \[w\to W, w=F(v)\ \text{für ein}\ v\in V\]
  \[w\rsa \bar\phi(w)=\phi(v)\]
  Ist
  \[w=F(v')\]
  dann ist
  \[v'-v\in\Ker F\]
  und
  \[\phi(v')-\phi(v)=\phi(v'-v)=0\]
  \[\begindc{\commdiag}[50]
  \obj(0,1){$V$}
  \obj(2,1){$K$}
  \obj(1,0)[F]{$\Im F$}
  \mor{$V$}{$K$}{$\phi$}
  \mor{$V$}{F}{}
  \mor{F}{$K$}{$\bar\phi$}
  \enddc\]
  \[\exists \underbrace{\psi}_{\in W^*}\mapsto \underbrace{\bar\phi}_{\in (\Im F)^*}\]
  d.h.
  \[\psi:W\to \mb{K}\]
  mit
  \[\psi|_{\Im F}=\bar\phi\]
  Das zeigt:
  \[F^*(\psi)=\phi\]
\end{Eig}
\begin{Bem}
  An dem Diagramm haben wir eine Bijektion zwischen $\phi\in V^*$ mit $\phi|_{\Ker F}=0$ und $\bar\phi\in \left(\Im F \right)^*$
  \[\xRightarrow{\dim W<\infty}\ \dim(\Im F)= \dim (\Im F)^* = \dim (\Ker F)^0=\dim \Im (F^*)\]
  \[\xRightarrow{\dim V,\dim W<\infty} \rang(F)=\rang(F^*)\]
  Keine Überraschung! $\rang(A)=\rang(A^t)$
\end{Bem}
\begin{Bsp}
  \[\mb{R}[x]^{\leq 2}\xrightarrow{(ev_{-1},ev_1)}\mb{R}\]
  surjektiv 
  \[\implies (\Im F)^0=0\]
  Interpretation:
  \[\alpha f(-1)+\beta f(1)=0\ \forall f\in \mb{R}[x]^{\leq 2}\ \Lra\ \alpha=\beta=0\]
  \[\Ker\left( (\alpha, \beta) \mapsto \left( f\mapsto \alpha f(-1)+\beta f(1) \right) \right)\]
  \begin{gather*}
    \Ker (ev_{-1},ev_1)=\Span (x^2-1)\\
    \implies \Ker(ev_{-1},ev_1)^0=\left\{ \mb{R}[x]^{\leq 2}\xrightarrow{\phi}\mb{R},\ \phi(x^2-1)=0 \right\}
    =\Span \left( \frac{1}{2}ev''_0+ev_0,ev_0' \right)
  \end{gather*}
  und 
  \[=\Im\left( ev_{-1},ev_1 \right)^*=\Span (ev_{-1},ev_1)\]
  weil
  \begin{gather*}
    ev_{-1}=\frac{1}{2}ev_0'' -ev_0'+ev_0\\
    ev_1=\frac{1}{2}ev_0'' + ev_0' + ev_0
  \end{gather*}
\end{Bsp}
\subsection{Der Bidualraum $V\rsa V^*\rsa V^{**}$}
\begin{Def}{kanonische lineare Abbildung}
  $\dim V<\infty$ $\implies$ ein Isomorphismus $V\to V^*$ wird durch die Auswahl einer Basis bestimmt. Dagegen haben wir eine Abbildung $V\to V^{**}$ unabhängig von der Basis, so:
  \[v\mapsto \Mx{V^*\xrightarrow{ev_v}\mb{K}\\ (\phi: V\to\mb{K})\mapsto \phi(v)}\]
  Dies heisst kanonische lineare Abbildung und ist ein Isomorphismus falls $\dim V<\infty$
\end{Def}
\begin{Bem}
  Im Allgemeinen ist die kanonische Abbildung $V\to V^{**}$ injektiv: 
  \begin{align*}
    \left[ \text{Sei}\ v\in V\ \text{mit}\ v\neq 0 \right] \stackrel{\Span(v)\subset V}{\rsa}&V^*\twoheadrightarrow \Span(V)^*\\
    & \phi\mapsto \psi:v\mapsto 1\ (\text{d.h.}\ \phi(v)=1)\\
    \implies ev_1(\phi)\neq 0
  \end{align*}
  \[\dim V <\infty\implies \dim V= \dim V^*=\dim V^{**}\]
  Dann:
  \[\implies V\to V^{**}\ \text{injektiv}\ \Lra \text{bijektiv}\]
  Oft schreibt man $V=V^{**}$ für $V$ ein Vektorraum mit $\dim V<\infty$. Das bedeutet immer, dass $V$ und $V^{**}$ identifiziert wird, durch den kanonischen Isomorphismus.
\end{Bem}
\begin{Bsp}
  $V=\mb{K}^n$ mit Standardbasis $e_1,\cdots,e_n$.\\
  $V^*=(\mb{K}^n)^*$ hat die duale Basis $e_1^*,\cdots,e_n^*$\\
  $V\to V^{**}$ mit einer Abbildung \[e_i\mapsto \Mx{\phi:(\mb{K}^n)^*\to\mb{K}\mapsto \phi(e_i)\\ e_j^* \mapsto \delta_{ij}}=e^{**}_i\]
\end{Bsp}
\begin{Bem}
  Sei $F:V\to W$ eine lineare Abbildung von endlichdimensionalen Vektorräumen. Dann ist $F^**=F$, im folgenden Sinn:
  \[\begindc{\commdiag}[40]
  \obj(0,1){$V$}
  \obj(0,0){$V^{**}$}
  \obj(1,0){$W^{**}$}
  \obj(1,1){$W$}
  \obj(2,1){$W^*$}
  \obj(3,1){$V^*$}
  \mor{$V$}{$V^{**}$}{$\sim$}
  \mor{$V$}{$W$}{$F$}
  \mor{$V^{**}$}{$W^{**}$}{$F^{**}$}
  \mor{$W$}{$W^{**}$}{$\sim$}
  \mor{$W$}{$W^*$}{$ $}[1,\dasharrow]
  \mor{$W^*$}{$W^{**}$}{$ $}[1,\dasharrow]
  \mor{$W^*$}{$V^*$}{$F^*$}
  \enddc\]
  Wobei $\sim$ einen kanonischen Isomorphismus darstellt.\\
  Daraus folgt, dass
  \[V\xrightarrow{F} W\rsa W^*\xrightarrow{F^*}V^*\rsa V^{**}\xrightarrow{F^{**}}W^{**}\]
  kommutativ ist.
\end{Bem}
\begin{Bem}
  \[\begindc{\commdiag}[5]
  \obj(00,05){$\phi$}
  \obj(10,05)[phi]{$\phi(v)\in V^{**}$}
  \obj(10,10)[V**]{$V^{**}$}
  \obj(10,20)[V]{$V$}
  \obj(00,20)[v]{$v\in V$}
  \obj(20,20)[W]{$W$}
  \obj(30,20)[F]{$F(v)\in V$}
  \obj(20,10)[W**]{$W^{**}$}
  \obj(30,10)[psi1]{$\psi$}
  \obj(40,10)[psiF]{$\psi(F(v))$}
  \obj(30,05)[psi2]{$\psi$}
  \obj(40,05)[Fpsi]{$F^*(\psi)(v)$}
  \cmor((0,22)(15,25)(30,22)) \pright(22,26){$\mapsto$}[0]
  \mor{V}{W}{}
  \mor{V}{V**}{}
  \mor{V**}{W**}{}
  \mor{W}{W**}{}
  \cmor((10,3)(25,0)(40,03)) \pright(25,1){$\mapsto$}
  \mor{$\phi$}{phi}{ }[1,\aplicationarrow]
  \mor{psi1}{psiF}{ }[1,\aplicationarrow]
  \mor{psi2}{Fpsi}{}[1,\aplicationarrow]
  \mor{Fpsi}{psiF}{$=$}[1,\solidline]
  \mor{F}{psi1}{}[1,\aplicationarrow]
  \enddc\]
  Falls $\dim W<\infty$ und $V\subset W$, dann haben wir $V^{00}=V$ im folgenden Sinn
  \[\overbrace{V^0}^{\dim =\dim W-\dim V}\subset W^*\]
  \[\overbrace{V^{00}}^{\dim = \dim W-(\dim W-\dim V)\dim V}\subset W^{**}\xleftarrow{\sim} W\supset V\]
  Das Bild von $V$ unter dem kanonischen Isomorphismus ist $V^{00}$.\\
  Sei $v\in V$ $ev_1\in V^{00}$
  \[\phi \in W^*,\ \phi(v)0\ \forall \phi \in V\implies \phi(v)=0\]
\end{Bem}
\begin{Bsp}
  \begin{gather*}
    W=\mb{R}^3\\
    V=\Span\left( \Mx{1\\0\\1},\Mx{0\\-1\\1} \right)\implies V^0=\Span(e_1^*-e_2^*-e_3^*)
    =\Span (e_1+e_3,-e_2+e_3)\\
    V^{00}=\Span(e^{**}_1+e^{**}_2,e_1^{**}+e_3^{**})
  \end{gather*}
\end{Bsp}

\subsection{Zusammenhang zwischen Dualraum und bilinearen Abbildungen}
\begin{itemize}
  \item schon gesehen, z.B. bei der Definition ``nicht ausgeartet''
  \item jetzt explizit
\end{itemize}
\begin{Def}{bilineare Abbildung}
  Sei $\mb{K}$ ein Körper, $v$ und $W$ Vektorräume über $\mb{K}$. Eine Abbildung $b:V\times W\to \mb{K}$ heisst bilinear falls 
  \[w\mapsto b(v,w)\ \text{ist linear}\  \forall v\in V\]
  und
  \[v\mapsto b(v,w)\ \text{ist linear}\  \forall w\in W\]
  \[\left[ (w\mapsto b(v,w))\in W^* \right]\]
\end{Def}
\begin{Bem}
  Im Fall $W=V$ ist dies genau zu sagen, dass $b$ eine Bilinearform ist. Also haben wir Abbildungen
  \[b':V\to W^*\]
  und
  \[b'':W\to V^*\]
  Aus der Definition folgt, dass $b'$ und $b''$ sind linear.
\end{Bem}
\begin{Def}{nicht ausgeartet}
  Eine bilineare Abbildung $b:V\times W\to\mb{K}$ ist nicht ausgeartet, falls $b':V\to W^*$ und $b'':W\to V^*$ injektiv sind.
\end{Def}
\begin{Bem}
  Falls $V$ und $W$ endlich dimensional sind, ist es nur möglich, eine nicht ausgeartete bilineare Abbildung zu haben, wenn $\dim V=\dim W$. Falls $\dim V=\dim W:$ ``injektiv'' oben ist äquivalent zu ``bijektiv''.
\end{Bem}
\begin{Bsp}
  \begin{itemize}
    \item $V$ beliebig, dann ist
      \begin{align*}
        V\times V^*&\to\mb{K}\\
        (v,\phi)&\mapsto \phi(v)
      \end{align*}
      stets nicht ausgeartet.
      \begin{align*}
        b':V&\to V^{**}\\
        v&\mapsto e v_v
      \end{align*}
      ist die kanonische Abbildung, ist injektiv
      \begin{align*}
        b'':V^*&\to V^*\\
        \phi&\mapsto\phi
      \end{align*}
      ist $id_{V^*}$ ist ein Isomorpismus
    \item $\mb{K}=\mb{R}$, $\left\langle , \right\rangle $ Skalarprodukt auf $V$.
      \[b(v,w):=\left\langle v,w \right\rangle\]
      $b'$ und $b''$ sind gleich, definiert als $\Psi$
      \[\rsa \Psi:V\to V^*\]
      injektiv\\
  \end{itemize}
\end{Bsp}
\begin{Bem}
  Das zeigt, dass jedes Skalarprodukt nicht ausgeartet ist. Und: falls $\dim_\mb{R}V<\infty$ ist $\Psi$ ein Isomorphismus. $\Psi$ heisst kanonisch. (kanonische Abbildung bzw. kanonischer Isomorphismus)
\end{Bem}
\begin{Eig}{$V$, $\dim V = n$, mit Skalarprodukt, kanonischer Isomorphismus $\Psi$}
  \begin{itemize}
    \item Für $U\subset V$ Untervektorraum gilt 
      \[\Psi(U^\perp)=U^0\]
    \item Für $B=(v_1,\cdots,v_n)$ eine orthonormal Basis haben wir 
      \[\Psi(v_i)=V^*_i\]
      für $i=1,\cdots,n$, wobei $(v_1^*,\cdots,v_n^*)$ die duale Basis ist.\\
      zeigen:
      \[\underbrace{\Psi(U^\perp)}_{\dim = \dim V-\dim U}\subset \underbrace{U^0}_{\dim=\dim v-\dim U}\ \text{klar}\]
      \[\left\langle v_i,\sum^n_{j=1}a_jv_j \right\rangle =a_i\]
      \[v_i^*\left( \sum^n_{j=1}a_jv_j \right)=a_i\]
  \end{itemize}
\end{Eig}
\begin{Bsp}
  Graphiker gesucht ;)
\end{Bsp}
\begin{Bem}
  Wir haben zwei kanonische Abbildungen:
  \[V\to V^{**}\ \text{für beliebiges $V$ / $\mb{K}$}\]
  \[V\xrightarrow{\Psi} V^*\ \text{für $V/\mb{R}$ mit Skalarprodukt}\]
\end{Bem}
\begin{Def}{adjugierte Abbildung}
  $V$, $W$ euklidische Vektorräume
  \[F:V\to W\ \text{lineare Abbildung}\]
  adjugiert: $F^{ad}:W\to V$ ist adjugiert zu $F$ falls gilt
  \[\left\langle F(v),w \right\rangle =\left\langle v,F^{ad}(w) \right\rangle \ \forall v\in V, w\in W\]
\end{Def}
\begin{Bem}
  \[\begindc{\commdiag}[60]
  \obj(0,1)[V]{$V$}
  \obj(1,0)[W*]{$W^*$}
  \obj(0,0)[V*]{$V^*$}
  \obj(1,1)[W]{$W$}
  \mor{W}{V}{$F^{ad}$}
  \mor{V}{V*}{$\Phi$}
  \mor{W*}{V*}{$F^*$}
  \mor{W}{W*}{$\Psi$}
  \enddc\]
  \begin{gather*}
    F^*(\Psi(w))(v)=\Psi(w)\left( F(v) \right)=\Phi\left( F^{ad}(w) \right)(v)\\
    \implies F^*\left( \Phi(w) \right)=\Phi\left( F^{ad}(w) \right)\ \text{in}\ V^*    
  \end{gather*}
  Daraus folgt, dass das Diagramm kommutiert
\end{Bem}
\begin{Bem}
  Seien $v_1,\cdots,v_n$ orthonormale Basen von $V$, $w_1,\cdots,w_m$ für $W$ $\rsa$ duale Basen $v_1^*,\cdots,v_n^*$ und $w_1^*,\cdots,w_m^*$
  Bezüglich orthonormaler Basen ist $F^{ad}$ durch die transponierte Matrix gegeben:
  Sei
  \[F\ \lra\ A\in M(m\times n,\mb{R})\]
  dann, aus Prop 49 (13?):
  \[F^*\ \lra\ A^t\in M(m\times n,\mb{R})\implies F^{ad}\ \lra A^t\ \text{weil} \Phi(v_i)=v_i^*,\ \Psi(w_i)=w_i^*\ \forall i\]
\end{Bem}
\begin{Bsp}
  $V=\mb{R}^2$ mit Skalarprodukt, $W=\mb{R}[x]^{\leq 2}$ mit \[\left\langle f,g \right\rangle =\int_{-1}^1f(x)g(x)\md x\]
  \begin{align*}
    F:V&\to W\\
    (\alpha,\beta)&\mapsto \alpha+\beta x+\alpha x^2
  \end{align*}
  Basis von $W$ $1,x,x^2$
  \[V^*=(\mb{R}^2)^*\ \text{mit Basis}\ e_1^*,e_2^*\]
  %TODO ergänzen
  \[\begindc{\commdiag}[60]
  \obj(0,1)[V]{$V$}
  \obj(1,0)[W*]{$W^*$}
  \obj(0,0)[V*]{$V^*$}
  \obj(1,1)[W]{$W$}
  \mor{W}{V}{$F^{ad}$}
  \mor{V}{V*}{$\Phi$}
  \mor{W*}{V*}{$F^*$}
  \mor{W}{W*}{$\Psi$}
  \enddc\]
  \begin{align*}
    V&\xrightarrow{\Psi}V^*\\
    e_1&\mapsto e_1^*\\
    e_2&\mapsto e_2^*
  \end{align*}
  $W^*$ hat die duale Basis $1^*,x^*,x^{2^*}$.\\
  Wir berechnen $\Psi$ explizit:
  \begin{align*}
    \Psi(1)=\left( f\mapsto \int^1_{-1}f(x)\md x \right)
  \end{align*}
  \begin{align*}
    W&\xrightarrow{\Phi}W^*\\
    1&\mapsto 2(1^*)+\frac{2}{3}\left( x^{2^*} \right)\\
    x&\mapsto \cdots\\
    x^2&\mapsto \cdots
  \end{align*}
  Dann:
  \[F^*\left( \psi(1) \right)=\left(\left( \alpha,\beta \right)\int^1_{-1}\alpha+\beta x+\alpha x^2\md x=2\alpha+\frac{2}{3}\alpha=\frac{8}{3}\alpha\right)\]
  d.h.
  \[\frac{8}{3}e^*_1\stackrel{\Phi^{-1}}{\mapsto}\left( \frac{8}{3},0 \right)\]
  Ähnlich:
  \[F^*\left( \Psi(x) \right)=\left( (\alpha,\beta) \mapsto \int^1_{-1}\alpha x+\beta x^2+\alpha x^3\md x=\frac{2}{3}\beta\right)\]
  und
  \[F^*\left( \Psi(x^2) \right)=\left( (\alpha,\beta) \mapsto \int^1_{-1}\alpha x^2+\beta x^3+\alpha x^4\md x=\frac{2}{3}\alpha + \frac{2}{5}\alpha=\frac{16}{15}\alpha\right)\]
  d.h.
  \begin{align*}
    F^{ad}(1)&=\left( \frac{8}{3},0 \right)\\
    F^{ad}(x)&=\left( 0,\frac{2}{3} \right)\\
    F^{ad}(x^2)&=\left( \frac{16}{15},0 \right)
  \end{align*}
  \[F^{ad}\left( a+bx+cx^2 \right)=\left( \frac{8}{3}a+\frac{16}{5}c,\frac{2}{3}b \right)\]
  Check:
  \[\int^1_{-1}\left( \alpha+\beta x+\alpha x^2 \right)\left( a+bx+cx^2 \right)\md x \stackrel{?}{=} \left\langle \left( \alpha,\beta \right),\left( \frac{8}{3}a+\frac{16}{15}c,\frac{2}{3}b \right) \right\rangle \]
  Skalarprodukt:
  \[\alpha\left( \frac{8}{3}a+\frac{16}{15}c \right)+\beta\left( \frac{2}{3}b \right)\]
  Integral:
  \begin{gather*}
    \int^1_{-1}a\alpha+(b\alpha+c\beta)x+(c\alpha+b\beta+a\alpha)x^2+\left( c\beta+\alpha b \right)x^3+c\alpha x^4\md x=\\
    =2a\alpha+\frac{2}{3}\left( a\alpha+b\beta+c\alpha \right)+\frac{2}{5}c\alpha
  \end{gather*}
  stimmt.
\end{Bsp}
\begin{Bem}
  Wir könnten $F^{ad}$ auch durch die Wahl einer orthonormalen Basis von $W$ berechnen.
  \[\frac{1}{\sqrt{2}},\sqrt{\frac{3}{2}}x,\sqrt{\frac{5}{2}}\sqrt{\frac{-1+3x^2}{2}}\]
  Dann:
  \[A=\Mx{\frac{4}{3}\sqrt{2}&0\\0&\sqrt{\frac{2}{3}}\\\frac{2}{3}\sqrt{\frac{2}{3}}}\]
  und so
  \[A^t=\Mx{\frac{4}{3}&0&\frac{2}{3}\sqrt{\frac{2}{3}}\\0&\sqrt{\frac{2}{3}}&0}\to F^{ad}\]
\end{Bem}

\begin{Bem}
  Jetzt betrachen wir den Fall $W=V$, also $b:V\times V\to\mb{K}$ eine Bilinearform. Da $b'$ und $b''$ genau durch das ``Umtauschen'' von $V$ und $W$ unterschieden
  \begin{gather*}
    b':V\to V^*,\ v\mapsto\left( w\mapsto b(v,w) \right)\\
    b'':V\to V^*,\ w\mapsto\left( v\mapsto b(v,w) \right)\\
  \end{gather*}
  haben wir Interpretation von Bedingungen über $b$:
  \begin{itemize}
    \item $b$ symmetrisch $\Lra$ $b''=b'$
    \item $b$ schiefsymmetrisch $\Lra$ $b''=-b'$
  \end{itemize}
\end{Bem}
\begin{Bem}
  Sei jetzt $\dim_\mb{K}<\infty$. Dann haben wir $b''=(b')^*$ in folgendem Sinne: Dual zu $b':V\to V^*$ ist
  \[\begindc{\commdiag}[50]
  \obj(0,1)[V**]{$V^{**}$}
  \obj(0,0)[V]{$V$}
  \obj(1,1)[V*]{$V^*$}
  \mor{V}{V**}{$\sim$}
  \mor{V**}{V*}{$(b')^*$}
  \enddc
  \ \ \
  \begindc{\commdiag}[50]
  \obj(0,1)[ev]{$ev_V$}
  \obj(0,0)[v]{$v$}
  \obj(1,1)[evc]{$ev_V\circ b'$}
  \mor{v}{ev}{$ $}[0,6]
  \mor{ev}{evc}{$ $}[0,6]
  \enddc\]
  ($ev_V$ = Auswertungsabbildung an $v\in V$)\\ Wobei:
  \[ev_V\circ b'(v')=ev_V\left( w\mapsto b(v',w) \right)=b(v',v)\]
  Da
  \[b'':V\to V^*\]
  durch
  \[ v\mapsto \left( v'\mapsto b(v',v) \right)\]
  definiert ist, haben wir Gleichheit.
\end{Bem}
\begin{Eig}
  \[b\ \text{symmetrisch}\iff b''=b'\iff (b')^*=b'\]
  \[b\ \text{schiefsymmetrisch}\iff b''=-b'\iff (b')^*=-b'\]
\end{Eig}
\begin{Bem}
  Falls $\dim_\mb{K}V<\infty$, $s$ symmetrisch und nicht ausgeartet führt zu
  \[b'(=b'')\ V\xrightarrow{\sim}V^*\]
\end{Bem}
\begin{Bem}{Spezialfall}
  $\mb{K}=\mb{R}$, $V$ euklidisch, dann ist des gerade das, was $\Psi$ hiess:\\
  Untevektorraum $U\subset V, U^\perp\subset V$ sowie $U^0\subset V^*$
  \[\begindc{\commdiag}[40]
  \obj(0,0)[Up]{$U^\perp$}
  \obj(0,1)[V]{$V$}
  \obj(1,0)[U0]{$U^0$}
  \obj(1,1)[V*]{$V^*$}
  \mor{Up}{V}{$\cup$}[0,2]
  \mor{U0}{V*}{$\cup$}[0,2]
  \mor{V}{V*}{$\sim$}
  \enddc\]
\end{Bem}
\begin{Bem}
  Was passiert, falls $K=\mb{C}$? Dann sind wir an sesquilinearen Abbildungen interessiert.
  \[s:V\times W\to\mb{C}\]
  Dann gibt es immer noch eine Abbildung
  \[s'':W\to V^*\ w\mapsto\left( v\mapsto s(v,w) \right)\]
  aber diese ist nicht mehr linear.
\end{Bem}
\begin{Bsp}
  $V=\mb{C}[x],s:V\times V\to\mb{C}$
  \[s(f,g)=\int^1_0f(x)\overline{g(x)}\md x\]
  Dann z.B.
  \[1\xmapsto{s''}\left( f\mapsto \int_0^1f(x)\md x \right)\]
  also: Durchschnittswert auf $[0,1]$\\
  aber:
  \[i\xmapsto{s''}\left( f\mapsto -i\int_0^1f(x)\md x \right)\]
  also: $(-i)\cdot$ Durchschnittswert auf $[0,1]$\\
  Damit ist $s''$ semilinear.
\end{Bsp}
\begin{Def}{semilinear}
  Eine Abbildung $t:V\to W$ zwischen $\mb{C}$-Vektorräumen heisst semilinear, falls:
  \begin{itemize}
    \item $t(v+v')=t(v)+t(v')$ $\forall v,v'\in V$
    \item $t(\lambda v)=\bar\lambda(v)$ $\forall v\in V,\ \lambda\in \mb{C}$
  \end{itemize}
\end{Def}
\begin{Def}{kanonischer Semi-Isomorphismus}
  Falls $V$ ein unitärer Vektorraum ist, mit Skalarprodukt
  \[s:V\times V\to\mb{C}\]
  so erhalten wir (was oben $s''$ heisst, nennen wir hier $\Psi$)
  \[\Psi:V\to V^*\ \text{kanonischer $\underbrace{\text{Semi}}_{\text{semilinear}}$-$\underbrace{\text{Isomorphismus}}_{\text{bijektiv}}$}\]
\end{Def}
\begin{Bem}
  Wie vorher haben wir zu einem Endomorphismus
  \[F:V\to V\]
  den adjugierten Endomorphismus
  \[F^{ad}:V\to V\]
  gegeben durch
    \[F^{ad}:=\Psi^{-1}\circ F^*\circ \Psi\]
\end{Bem}
\begin{Eig}{adjugierter Endomorphismus}
  \begin{itemize}
    \item 
      \[s\left( F(v),w \right)=s\left( v,F^{ad}(w) \right)\ \forall v,w\in V\]
    \item 
      \[\Im F^{ad}=\left( \Ker F \right)^\perp\]
    \item
      \[\Ker F^{ad}=<\left( \Im F \right)^\perp\]
    \item Ist $B$ eine Orthonormalbasis von $V$ und $A$ die darstellende Matrix von $F$ bezüglich $B$, dann ist $\bar A^t$ die darstellende Matrix von $F^{ad}$
  \end{itemize}
\end{Eig}
\begin{Sat}
  \[F\ \text{ist unitär diagonalisierbar} \iff F\circ F^{ad}=F^{ad}\circ F\]
\end{Sat}
\begin{Def}{$F$ normal}
  $F$ heisst normal, falls
  \[F\circ F^{ad}=F^{ad}\circ F\]
\end{Def}
\subsection{Anwendung des Dualraums}
das duale Polytop $\mb{K}=\mb{R}, V=\mb{R}^n$
\begin{Def}{konvexe Menge}
 $S\subset\mb{R}^n$ mit der Eigenschaft $\forall s,t\in S$: $s$ und $t$ sind wegzusammenhängend.
\end{Def}
\begin{Def}{konvexe Hülle}
  konvexe Hülle von $\Gamma\subset\mb{R}^n$ ist
  \[\cap_{S\subset\mb{R}^n, S\ \text{konvex},\ \Gamma\subset S} S\]
  ``kleinste konvexe Menge, in der $\Gamma$ enthalten ist''
\end{Def}
\begin{Def}{konvexes Polytop}
  die konvexe Hülle von einer endlichen Menge in $\mb{R}^n$
\end{Def}
\begin{Def}{innerer Punkt}
  Das Polytop $P$ hat $O\in \mb{R}$ als inneren Punkt falls:
  \begin{itemize}
    \item $O\in\mb{R}$
    \item $\exists \varepsilon>0:B_\varepsilon(0)\subset P$
  \end{itemize}
\end{Def}
\begin{Def}
  Ist ein Polytop mit $O\in\mb{R}$ als inneren Punkt, so definieren wir
  \[P^*=\left\{ \phi\in\left( \mb{R}^n \right)^*:\phi(v)\leq 1\ \forall v\in P \right\}\]
\end{Def}
\begin{Def}{duales Polytop}
  $P^*$ ist ein konvexes Polytop $O\in\left( \mb{R}^n \right)^*$ als innerem Punkt. $P^*$ heisst duales Polytop.
\end{Def}
\begin{Bem}
  \[P^**=P\]
\end{Bem}
\begin{Bem}
  Konstruktion des dualen Polytops:
  $l$ Hyperebene, so dass $P$ auf einer Seite von $l$ liegt (inklusive $l$ selbst) $\rsa$ Gleichung von $l$ schreiben als
  \[\alpha_1x_1+\cdots+\alpha_nx_n=1\]
  \[P\subset \left\{ \left( x_1,\cdots,x_n \right)|\alpha_1x_1+\cdots+\alpha_nx_n\leq 1 \right\}\]
  $\rsa$
  \[\left( \alpha_1,\cdots,\alpha_n \right)\in P^*\]
  (Skizze (Freiwilliger gesucht \}:-) )
  \[\text{Facetten von}\ P\ \rsa\ \text{Ecken in}\ P^*\]
  $(P^*)$ konvexe Hülle
\end{Bem}
\begin{Bsp}
  \begin{itemize}
    \item Der Tetraeder ist selbstdual.
    \item Der Würfel dual zum Oktaeder.
    \item Der Dodekaeder ist dual zum Ikosaeder.
  \end{itemize}
\end{Bsp}

\subsection{Das Tensorprodukt}
\[V\ \rsa\ V^*\ \text{Linearformen}\]
Wir möchten eine Redeweise haben, genug flexibel für solche Situationen. Eine nützliche Konstruktion dafür ist das Tensorprodukt.
\begin{Def}{Tensorprodukt}
  Sei $\mb{K}$ ein Körper und $V$ und $W$ Vektorräume über $\mb{K}$. Ein $\mb{K}$-Vektorraum heisst Tensorprodukt von $V$ und $W$, geschrieben $V\otimes W$, falls, eine bilineare Abbildung
  \[\eta:V\times W\to V\otimes W\]
  gegeben ist, die folgende universelle Eigenschaften erfüllt:\\
  \item Zu jedem $\mb{K}$-Vektorraum $U$ mit bilinearer Abbildung
    \[\xi:V\times W\to U\]
    gibt es eine eindeutige lineare Abbildung 
    \[\xi':V\otimes W\to U\]
    so dass das Diagramm
    \[\begindc{\commdiag}[60]
      \obj(0,0)[ot]{$V\otimes W$}
      \obj(0,1)[t]{$V\times W$}
      \obj(1,0)[U]{$U$}
      \mor{t}{U}{$\xi$}
      \mor{ot}{U}{$\xi'$}
      \mor{t}{ot}{$\eta$}
    \enddc\]
    kommutiert.
\end{Def}
\begin{Bem}
  Es ist noch unklar, wie die Elemente von $V\otimes W$ aussehen, oder ob $V\otimes W$ gar existiert. Auch unklar: warum.
\end{Bem}
\begin{Kor}
  Aus der Definition folgt: $V\otimes W$, falls es existiert, ist eindeutig bis auf Isomorphismus.
\end{Kor}
\begin{Bem}
  Seien
  \[\eta: V\times W\to V\otimes W\]
  und
  \[\tilde\eta: V\times W\to \widetilde{V\otimes W}\]
  gegeben, beide erfüllen die universellen Eigenschaften.
  Wir wenden die universelle Eigenschaft an, mit $U:=\widetilde{V\otimes W}$, so dass das folgende Diagramm kommutiert:
  \[\begindc{\commdiag}[60]
    \obj(0,0)[ot]{$V\otimes W$}
    \obj(1,0)[ti]{$\widetilde{V\otimes W}$}
    \obj(0,1)[t]{$V\times W$}
    \mor{t}{ti}{$\tilde\eta$}
    \mor{ot}{ti}{$\zeta$}
    \mor{t}{ot}{$\eta$}
  \enddc\]
  Wir wenden die universelle Eigenschaft an mit $U:=V\otimes W$, so dass das folgende Diagramm kommutiert:
  \[\begindc{\commdiag}[60]
    \obj(1,0)[ot]{$V\otimes W$}
    \obj(0,0)[ti]{$\widetilde{V\otimes W}$}
    \obj(0,1)[t]{$V\times W$}
    \mor{ti}{ot}{$\tilde\zeta$}
    \mor{t}{ti}{$\tilde\eta$}
    \mor{t}{ot}{$\eta$}
  \enddc\]
  Wir wenden die universelle Eigenschaft an mit $U:=V\otimes W$
  \[\begindc{\commdiag}[60]
    \obj(1,0)[ot]{$V\otimes W$}
    \obj(0,0)[ot2]{$V\otimes W$}
    \obj(0,1)[t]{$V\times W$}
    \mor{t}{ot}{$\eta$}
    \mor{t}{ot2}{$\eta$}
    \mor{ot2}{ot}{$\tilde\zeta\circ\zeta$}
  \enddc
  \ \ \ 
  \begindc{\commdiag}[60]
    \obj(1,0)[ot]{$V\otimes W$}
    \obj(0,0)[ot2]{$V\otimes W$}
    \obj(0,1)[t]{$V\times W$}
    \mor{t}{ot}{$\eta$}
    \mor{t}{ot2}{$\eta$}
    \mor{ot2}{ot}{$1_{V\otimes W}$}
  \enddc\]
  Beide Diagramme kommutieren
  \[\tilde\zeta\circ\zeta\circ\eta=\tilde\zeta\circ\tilde\eta=\eta\]
  \[1_{V\otimes W}\circ\eta=\eta\]
  Es folgt aus der universellen Eigenschaft, dass
  \[\tilde\zeta\circ\zeta=1_{V\otimes W}\]
  Wir wenden die universelle Eigenschaft an, mit $U:=\widetilde{V\otimes W}$
  \[\begindc{\commdiag}[60]
    \obj(1,0)[ot]{$V\otimes W$}
    \obj(0,0)[ot2]{$\widetilde{V\otimes W}$}
    \obj(0,1)[t]{$\tilde{V\times W}$}
    \mor{t}{ot}{$\tilde\eta$}
    \mor{t}{ot2}{$\tilde\eta$}
    \mor{ot2}{ot}{$\tilde\zeta\circ\zeta$}
  \enddc
  \ \ \ 
  \begindc{\commdiag}[60]
    \obj(1,0)[ot]{$\widetilde{V\otimes W}$}
    \obj(0,0)[ot2]{$\widetilde{V\otimes W}$}
    \obj(0,1)[t]{$V\times W$}
    \mor{t}{ot}{$\tilde\eta$}
    \mor{t}{ot2}{$\tilde\eta$}
    \mor{ot2}{ot}{$1_{\widetilde{V\otimes W}}$}
  \enddc\]
  Beide Diagramme kommutieren
  \[\zeta\circ\tilde\zeta\circ\tilde\eta=\zeta\circ\eta=\tilde\eta\]
  \[1_{\widetilde{V\otimes W}}\circ\tilde\eta=\tilde\eta\]
  Es folgt aus der universellen Eigenschaft, dass
  \[\zeta\circ\tilde\zeta=1_{\widetilde{V\otimes W}}\]
  Ergebnis:
  \[V\otimes W\xrightarrow{\zeta}\widetilde{V\otimes W}\]
  ist Isomorphismus, inverse zu
  \[\tilde\zeta:\widetilde{V\otimes W}\to V\otimes W\]
\end{Bem}
\begin{Faz}
  universelle Eigenschaft $\rsa$ Eindeutigkeit bis auf Isomorphismus
\end{Faz}
\begin{Not}{Tensorprodukt}
  $V\otimes W$ oder $V\otimes_\mb{K} W$
\end{Not}
\subsubsection{Existenz vom Tensorprodukt}
\begin{Bem}
  Es gibt zwei Methoden
  \begin{itemize}
    \item Durch Auswahl von Basen
    \item Beschreibung als Quotientenvektorraum
  \end{itemize}
  Heute: Methode 1
  \begin{itemize}
    \item braucht die Existenz von Basen
    \item klar fall $\dim_\mb{K} V< \infty$
  \end{itemize}
  oder im Allgemeinen für die, die das Auswahlaxiom gesehen haben.
\end{Bem}
\begin{Prop}
  Sei $(v_i)_{i\in I}$ eine Basis von $V$ und $(w_j)_{j\in J}$ eine Basis von $W$. Dann existiert das Tensorprodukt $V\otimes W$, mit Basis $(v_i\otimes w_j)_{(i,j)\in I\times J}$ und 
  \[\eta:V\times W\to V \otimes W\]
  \[\left( \sum_{i\in I}a_iv_i \sum_{j\in J}b_jw_j \right)\mapsto \sum_{(i,i)\in I\times J}a_ib_j\left( v_i\otimes w_j \right)\]
  mit $a_i\neq 0$ für endlich viele $i$ und $b_j\neq 0$ für endlich viele $j$\\
  Das bedeutet, dass die Elemente von $V\otimes W$
  \[\sum_{(i,j)\in I\times J}c_{ij}\left( v_i\otimes w_j \right)\ c_{ij}\in\mb{K}\]
  nur endlich viele $c_{ij}\neq 0$.
\end{Prop}
\begin{Bew}
  Zu verifizieren:
  \begin{itemize}
    \item dass $\eta$ bilinear ist
    \item und erfüllt die universelle Eigenschaft
  \end{itemize}
  $\eta$ ist bilinear:
  \begin{gather*}
    \eta\left( \sum_{i\in I} a_iv_i,\ \sum_{j\in J} b_jw_j \right)+\eta\left( \sum_{i\in I}a_i'v_i,\ \sum_{j\in J} b_jw_j \right)=\\
    =\sum_{i\in I} a_ib_j\left( v_i\otimes w_j \right)+\sum_{\left( i,\ j \right)\in I\times J}a_i'b_j\left( v_i\otimes w_j \right)=\\
    =\sum_{\left( i,\ j \right)\in I\times J}\left( a_ib_j+a_i'b_j \right)\left( v_i\otimes w_j \right)=\\
    =\sum_{\left( i,\ j \right)\in I\times J}\left( a_i+a_i' \right)b_j\left( v_i\otimes w_j \right)=\\
    =\eta\left( \sum_{i\in I}(a_i1a_i')v_i,\ \sum_{j\in J}b_jw_j \right)\\
    =\eta\left( \sum_{i\in I}a_iv_i+\sum_{i\in I}a_i'v_j,\ \sum_{j\in J}b_jw_j \right)
  \end{gather*}
\end{Bew}
\begin{Bsp}
  $V=\mb{K}^2$, $W=\mb{K}[t]$\\
  $V$ hat die Standardbasis $(e_1,e_2)$\\
  $W$ hat die Basis $(1,t,t^2,\cdots)$\\
  $\implies$ $V\otimes W$ hat die Basis $e_1\otimes 1,e_1\otimes t,\cdots$, $w_2\otimes1, e_2\otimes t,\cdots$
  \begin{align*}
    \eta:\mb{K}^2\times K[t]&\to K^2\otimes K[t]\\
    \left( (1,0),t^2 \right)&\mapsto e_1\otimes t^2\\
    \left( (0,1),t^3 \right)&\mapsto e_2\otimes t^3\\
    \left( (2,3),t^4 \right)&\mapsto 2e_1\otimes t^4+3e_2\otimes t^4
  \end{align*}
  Typisches Element von $\mb{K}^2\otimes \mb{K}[t]$
  \[e_1\otimes t^2+e_2\otimes t^3+2e_1\otimes t^4+3e_2\otimes t^4\]
  Mit anderer Schreibweise:
  \[(t^2,0)+(0,t^3)+(2t^4,0)+(0,3t^4)\]
  oder:
  \[(t^2+2t^4,t^3+3t^4)\]
\end{Bsp}
\begin{Def}
  Sei $U$ ein $\mb{K}$-Vektorraum und 
  \[\xi:V\times W\to U\]
  eine bilineare Abbildung. Wir definieren
  \[\begindc{\commdiag}[60]
    \obj(0,0)[ot]{$V\otimes W$}
    \obj(0,1)[t]{$V\times W$}
    \obj(1,0)[U]{$U$}
    \mor{t}{U}{$\xi$}
    \mor{ot}{U}{$\xi'$}
    \mor{t}{ot}{$\eta$}
  \enddc\]
  \[\xi':V\otimes W\to U\]
  durch
  \[\xi'(v_i \otimes w_j):=\xi(v_i,w_j)\]
  für $i\in I$, $j\in J$, und deshalb:
  \[\xi'\left( \sum_{(i,j)\in I\times J} c_{ij}v_i\otimes w_j \right)= \sum_{(i,j)\in I\times J} c_{ij}(v_i,w_j)\]
  Das Diagram kommutiert (aus der Bilinearität von $\xi$)\\
  Die Eindeutigkeit ist durch die identischen Basenvektoren gegeben.
\end{Def}

\begin{Bem}
  \[\dim V\otimes W=\left( \dim V \right)\left( \dim W \right)\]
\end{Bem}
\begin{Not}
  Für $v\in V$, $w\in W$ schreibt man oft $v\otimes w$ für $\eta\left( v,w \right)$
\end{Not}
\begin{Bem}
  Weil
  \[\left( v,w \right)\mapsto v\otimes w:=\eta\left( v,w \right)\]
  eine bilineare Abbildung ist, haben wir
  \begin{align*}
    v\otimes w+v'\otimes w&=\left( v\times v'\right)\otimes w\\
    v\otimes w+v\otimes w'&=v\otimes \left(w'\times w\right)\\
    \left( \lambda v \right)\otimes w=\lambda\left( v\otimes w \right)=v\otimes\left( \lambda w) \right)
  \end{align*}
  Rechenregeln für Tensoren
\end{Bem}
\begin{Not}
  Letztes Mal: für Basiselemente $v_i$, $w_j$ bezeichnet $v_i\otimes w_j$ ein Basiselement von $V\otimes W$
\end{Not}
\begin{Bem}
  Die Abbildung lässt sich schreiben als
  \begin{align*}
    V\times W&\xrightarrow{\eta}V\otimes W\\
    \left( v_i,w_j \right)&\mapsto v_i\otimes w_j
  \end{align*}
\end{Bem}
\begin{Bem}
  Jetzt für beliebige $v\in V$, $w\in W$ bezeichnet $v\otimes w$ das Element
  \[\eta\left( v,w \right)\in V\otimes W\]
  Weil in der Konstruktion wir $\eta$ durch
  \[\eta\left( v_i,w_j \right):=v_i\otimes w_j\]
  definiert haben, stimmen die beiden Bedeutungen von $v_i\otimes w_j$ überein.
\end{Bem}
\begin{Bsp}{Isomorphismus von Vektorräumen über $\mb{R}$}
  \begin{align*}
    \mb{R}^2\otimes_\mb{R}\mb{R}[t]&\xrightarrow{\sim}\mb{R}[t]\oplus\mb{R}[t]\\
    e_1\otimes t^j&\mapsto\left( t^j,0 \right)\\
    e_2\otimes t^j&\mapsto\left( 0,t^j \right)
  \end{align*}
  Ganz ähnlich
  \begin{align*}
    \mb{C}\otimes_\mb{R}\mb{R}[t]&\xrightarrow{\sim}\mb{C}[t]\\
    1\otimes t^j&\mapsto t^j\\
    i\otimes t^j&\mapsto it^j
  \end{align*}
  $\implies$ für $\gamma\in \mb{C}$ gilt:
  \[\gamma\otimes t^j\mapsto \gamma t^j\]
  weil: 
  \[\gamma=\alpha+\beta i\ \alpha,\beta\in \mb{R}\]
  haben wir
  \begin{align*}
    \gamma \otimes t'&=\left( \alpha+\beta i \right)\otimes t^j\\
    &=\alpha\otimes t^j+\beta i\otimes t^j\\
    &=\alpha\left( 1\otimes t^j \right)+\beta\left( i\otimes t^j \right)\\
    &\mapsto \alpha\left( t^j \right)+\beta(it^j)\\
    &=\left( \alpha+\beta i \right)t^j\\
    &=\gamma t^j
  \end{align*}
\end{Bsp}
\begin{Prop}
  Sei $\mb{K}\to\mb{L}$ eine Körpererweiterung und $V$ ein $\mb{K}$-Vektorraum. Dann hat $K\otimes_\mb{K}V$ die Struktur von $L$-Vektorraum, mit
  \[\alpha\left( \beta\otimes v \right)=\left( \alpha\beta \right)\otimes v\ \ \text{für $\alpha,\beta\in L$ und $v\in V$}\]
\end{Prop}
\begin{Bew}
  Wir müssen verifizieren, dass
  \[\left( \alpha,\beta\otimes v \right)\mapsto \left( \alpha\beta \right)\otimes v\]
  eine Abbildung von $L\times\left( L\otimes_\mb{K}V \right)$ nach $L\otimes_\mb{K}V$ beschreibt. D.h. für jedes $\alpha\in L$, $\exists$ eine Abbildung $L\otimes V\to L\otimes V$
  \[\begindc{\commdiag}[100]
  \obj(0,0)[LoV1]{$L\otimes V$}
  \obj(1,0)[LoV2]{$L\otimes V$}
  \obj(0,1)[LtV1]{$L\times V$}
  \obj(1,1)[LtV2]{$L\times V$}
  \mor{LtV1}{LtV2}{$\left( \beta,\alpha \right)\mapsto \left( \alpha\beta,v \right)$}
  \mor{LtV1}{LoV1}{$\eta$}
  \mor{LtV1}{LoV2}{$\left( \alpha,\beta \right)\mapsto \left( \alpha\beta \right)\otimes v$}[1,0]
  \mor{LtV2}{LoV2}{$\eta$}
  \mor{LoV1}{LoV2}{Aus der u.E.}[-1,1]
  \enddc\]
  \[\left( \alpha,\beta \right)\mapsto \left( \alpha\beta \right)\otimes v\]
  ist bilinear:
  \begin{align*}
    \left( \beta+\beta',v \right)\mapsto\left( \alpha\left( \beta+\beta' \right) \right)\otimes v=&\\
    &=\left( \alpha\beta+\alpha\beta' \right)\otimes v&\text{Körpereigenschaft}\\
    &=\alpha\beta\otimes v+\alpha\beta'\otimes v&\text{Rechenregeln für Tensoren}
  \end{align*}
  So bekommen wir
  \begin{align*}
    L\times \left( L\otimes V \right)&\to L\otimes V\\
    \left( \alpha,\beta\otimes v \right)&\mapsto \left( \alpha\beta \right)\otimes v
  \end{align*}
  Wir müssen auch die Axiome für den Vektorraum über $L$ verifizieren, d.h.:
  \begin{align*}
    \alpha\left( w+w' \right)&=\alpha w+\alpha w'&\text{für $\alpha\in L$, $w,w'\in L\otimes V$}\\
    \alpha\left( \alpha'w \right)&=\left( \alpha\alpha' \right)w&\text{für $\alpha,\alpha'\in L$, $w\in L\otimes V$}
  \end{align*}
  Die erste Gleichung gilt weil $L\otimes V\dashrightarrow L\otimes V$ über $\mb{K}$-linear ist. Die zweite folgt aus der ersten, falls wir nur den Fall verifizieren, wobei $w=\beta\otimes v$. Dafür benutzen wir
  \begin{itemize}
    \item $L\otimes_\mb{K} V$ ist von Elementen $\beta\otimes v$ ($\beta\in L$, $v\in V$) aufgespannt, als $\mb{K}$-Vektorraum (Klar von der Konstruktion)
    \item Dann können wir schreiben
      \begin{align*}
        w&=\beta_1\otimes v_1+\cdots+\beta_\gamma\otimes v_\gamma&\gamma\in\mb{N}
      \end{align*}
  \end{itemize}
  \begin{align*}
    \alpha\left( \alpha'\left( \beta_1\otimes v_1+\cdots+ \beta_\gamma\otimes v_\gamma\right) \right)&=\alpha\left( \alpha'\left( \beta_1 \otimes v_1\right)+\cdots+\alpha'\left( \beta_\gamma \otimes v_\gamma\right) \right)\\
    &=\alpha\left( \alpha'\left( \beta_1\otimes v_1 \right) \right)+\alpha\left( \alpha'\left( \beta_\gamma \right) \right)\\
    &=\left( \alpha\alpha' \right)\left( \beta_1\otimes v_1 \right)+\cdots+\left( \alpha\alpha' \right)\left( \beta_\gamma\otimes v_\gamma \right)\\
    &=\left( \alpha\alpha' \right)\left( \beta_1\otimes v_1 +\cdots+ \beta_\gamma\otimes v_\gamma\right)
  \end{align*}
  für $w:=\beta\otimes v$:
  \begin{align*}
    \alpha\left( \alpha'\left( \beta\otimes v \right) \right)&=\alpha\left( \left( \alpha'\beta \right)\otimes v\right)\\
    &=\left( \alpha\left( \alpha'\beta \right) \right)\otimes v\\
    &=\left( \left( \alpha\alpha' \right)\beta \right)\otimes v\\
    &=\left( \alpha\alpha' \right)\left( \beta\otimes v \right)
  \end{align*}
\end{Bew}
\begin{Sat}
  \[\underbrace{\mb{C}\otimes_\mb{R}\mb{R}[t]}_{\mb{C}-\text{Vektorraum}}\to\mb{C}[t]\]
  Beh: dies ist ein Isomorphismus von $\mb{C}$-Vektorräumen. Nur noch zu verifzieren: die $\mb{C}$-Linearität. Im allgemeinen haben wir
  \[L\otimes_\mb{K}\mb{K}[t]\xrightarrow{\sim}L[t]\]
  $L$-linearer Isomorphismus
\end{Sat}
\begin{Bew}
  Sei $\gamma\in \mb{C}$
  \[\begindc{\commdiag}[60]
  \obj(0,0)[CR2]{$\mb{C}\otimes_\mb{R}\mb{R}[t]$}
  \obj(0,1)[CR1]{$\mb{C}\otimes_\mb{R}\mb{R}[t]$}
  \obj(1,0)[C2]{$\mb{C}[t]$}
  \obj(1,1)[C1]{$\mb{C}[t]$}
  \mor{CR1}{C1}{$\sim$}
  \mor{CR2}{C2}{$\sim$}
  \mor{CR1}{CR2}{$\gamma\cdot\left( \ \right)$}
  \mor{C1}{C2}{$\gamma\cdot\left( \  \right)$}
  \enddc\]
\end{Bew}
\begin{Bem}
  Ganz ähnlich
  \[L\otimes \mb{K}^n\xrightarrow{\sim}L^n\]
  $L$-linearer Isomorphimsus. ($n\in\mb{N}$) Insbesondere:
  \[\mb{C}\otimes_\mb{R}\mb{R}^n\xrightarrow{\sim}\mb{C}^n\]
\end{Bem}
\begin{Def}{Komplexifizierung}
  Ist $V$ ein $\mb{R}$-Vektorraum, so heisst der $\mb{C}$-Vektorraum $\mb{C}\otimes_\mb{R}V$ die Komplexifizierung von $V$
\end{Def}
\subsubsection{Tensorprodukt von linearen Abbildungen}
\begin{Def}{Tensorprodukt von linearen Abbildungen}
  Sei $V$, $W$, $V\otimes_\mb{K}W$, $V\times W\xrightarrow{\eta}V\otimes_\mb{K}W$ und $V'$, $W'$, $V'\otimes_\mb{K}W'$, $V'\times W'\xrightarrow{\eta'}V'\otimes_\mb{K}W'$ gegeben, mit linearen Abbildungen
  \begin{gather*}
    V\xrightarrow{\phi}V'\\
    W\xrightarrow{\psi}W'
  \end{gather*}
  Dann haben wir:
  \[\dcp{40}{
  \obj(0,1)[VW]{$V\times W$}
  \obj(0,0)[VoW]{$V\otimes W$}
  \obj(2,0)[V'o]{$V'\otimes W'$}
  \obj(2,1)[V't]{$V'\times W'$}
  \mor{VW}{VoW}{$\eta$}
  \mor{VW}{V't}{$\phi\times\psi$}
  \mor{V't}{V'o}{$\eta'$}
  \mor{VoW}{V'o}{neu}[1,1]
  }\]
  Beh: die Komposition ist bilinear
  \[\left( v,w \right)\mapsto \psi(v)\otimes \psi(w)\]
  neu aus der universellen Eigenschaft das Tensorprodukt von linearen Abbildungen
\end{Def}
\begin{Not}{Tensorprodukt von linearen Abbildungen}
  \[\phi\otimes \psi\]
\end{Not}


\newpage

%= Stichwortverzeichnis ======================================================================
\rhead{}
\addcontentsline{toc}{section}{Stichwortverzeichnis}
\printindex

\end{document}
